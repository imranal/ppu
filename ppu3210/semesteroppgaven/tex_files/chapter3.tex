\documentclass[main.tex]{subfiles} 
\begin{document}

\section*{Refleksjon}
\label{sec:3}
Hvordan bidro utforskende samtaler til å skape god begrepsforståelse i naturfagstimen for 8. klassen, 
og hvordan kunne undervisningsopplegget forbedres?
\newline
\newline
% Manger (2013) beskriver motivasjon som noe mer enn en trengsel for å ha lyst på noe eller ønske om å 
% utføre en aktivitet.
% \begin{displayquote}
% Det handler om den mentale innsatsen til eleven. Å lese ein tekst ti gonger kan indikera at eleven 
% held ut, men læringsmotivasjon viser seg mellom anna gjennom meir aktive studiestrategiar, slik som 
% oppsummeringar, refleksjon over dei grunnleggjande ideane i faget og sammenfattingar av ideane med 
% eigne ord. (\citeNP[s. 162]{mang13})
% \end{displayquote}
% Det vil si en form for indre motivasjon, der eleven reflekterer over begrepene i naturfag og uttrykker
% de med sine egne ord. Altså en form for assimilasjon, og hvis det oppstår en kognitiv konflikt, 
% akkommodasjon. Gjennom gruppesamtalene forsøkte en del elever å beskive begreper med sine egne ord. 
% I blant viste de svakheter eller misforståelser i sine formuleringer, men det faktum at de brukte eget 
% språk kan antyde at de var villige til å lære og forstå begrepene. Hos Vygotsky 
% (\citeNP[s. 130]{bta98}) består motivasjon i å skape meningsfulle læringsbetingelser. Dette kan oppnås 
% ved å tilrettelegge undervisningen som passer elevens nærmeste utviklingssonen, og ved å tydeliggjøre 
% nytteverdien av det gitte lærestoffet. Så fra lærer perspektivet kan vi oppfatte motivasjon som en 
% ytre faktor, dvs. ytre motivasjon. Indre og ytre motivasjon trenger ikke å være motpoler på en skala. 
% Et fag kan inneholde både indre og ytre element. For eksempel en lærer gjennom levende fortellingsevne 
% og engasjerende diskusjoner med elever skaper grunnlag for indre motivasjon. Samtidig kan elevenes 
% bevissthet rundt hva gode skoleresultater har å si for valg av utdanning og yrke, støtte opp under 
% læring som har grunnlag i en interessant undervisning (\citeNP[s. 148]{mang13}).
% \newline
% \newline
\citeA[s. 136]{klet13} beskriver en god undervisningsseksens der lærere klarer å balansere mellom 
tilegnelses-, utprøvings-, og konsolideringssituasjoner. Ifølge Klette har norske klasserom ensidige 
tendenser i bruken av varierte arbeidsmåter. Slik det kan ses fra figur \ref{fig:odeg10}, er det for 
eksempel lite konsolideringssituasjoner. Lærernes metalæringsaktiviteter regnes som særlig 
avgjørende for å sikre elevenes læring (\citeNP[s. 186]{klet13}). Å bruke dette som et fast
organiserende prinsipp, blir derimot sjelden gjennomført (\citeNP[s. 26]{odeg10}). Gjennom egen 
praksiserfaring har mine timer inkludert aktivering av forkunnskaper, gjennom repitisjon og 
gjenbruk av begreper og gjennomgang av lekser, bæret preg av konsolideringssituasjoner. Derimot
har timen som har blitt analysert ikke hatt noen appetittvekkere. Dette er noe som kunne ha blitt 
inkludert. For eksempel ved introduksjonen av encellede organismer kunne en kort videosnutt ha blitt
inkludert slik at elevene kunne se noen ``levende'' mikroorganismer.\footnote[3]{Til tross for at 
elevene ville ha observert slike organismer i labøvelsen i den tredje timen.}  
\newline
\newline
I helklassesamtalene ble elevene spurt om det de har hatt til lekse.
Siden de blir engasjert i samtaler rundt lekser de skal ha utført, har de forutsetning for å kunne 
respondere på lærerinitiativ. Det er ønskelig å få bekreftet at elevene innehar en overordnet 
forståelse. Det kan derfor være nødvendig å utpeke noen elever som ikke viser aktiv deltagelse i 
timen og frembringe deres respons. Gjennom min egen praksiserfaring brukte jeg navnekort
til å engasjere vilkårlige elever. Noen elever viste således svakheter i deres forståelse.
I blant valgte jeg å hjelpe eleven korrigere sin kunnskap ved å hjelpe de til å komme
fram til en korrekt oppfatning, og noen ganger valgte jeg å la andre elever slippe inn i dialogen
slik at de kunne bidra med egen oppfatning.
\newline
\newline
Hvis elevene ikke klarer å respondere på lærer initiativ, kan utspørringen av elevene vise hull 
i deres kunnskap. Derimot har utpeking av elever også noen negative implikasjoner. For eksempel 
vil noen elever føle ubehag av å bli utpekt (\citeNP[s. 82]{weos01}). Det er ønskelig å trene 
elevene i å aktivt delta i undervisningen, men det er også lurt å ikke forsterke negative 
assosiasjoner til slik deltagelse. Hvis svake elever blir engasjert, bør de få muligheten til å 
kunne demonstrere sin mestring av temaer de er fortrolig og godt kjent med (\citeNP{spr02}). 
Uansett er det nødvendig at elevene får autentiske mestringssituasjoner slik at deres forventninger
om mestring øker (\citeNP[s. 163]{mang13}). Hvis en av intensjonene med helklassesamtalen var å 
finne hull i elevenes kunnskapsnivå, ble gruppesamtaler en bedre arena for meg.
\newline
\newline
Øvelsen med tokolonnnenotatet hadde flere styrker, men den hadde noen organisatoriske svakheter. 
Det ble brukt for mye tid til å fordele elever i grupper, dette kunne gjerne ha blitt planlagt på 
forhånd. Dessuten var instruksjonene ikke helt klare, tydelighet i instruksjoner ville ha spart tid 
som elevene da kunne ha brukt i faglig aktivitet. Av faktorer som har direkte effekt på elevenes 
læring, fremhever \citeA[s. 189]{klet13} et gjennomtenkt undervisningsopplegg som muliggjør at 
de bruker minimalt tid på ikke-faglige aktiviteter. For tokolonnenotatet er det også viktig å være 
bevisst på hvor mange frihetsgrader elever skal få (\citeNP[s. 29]{knai11}). Jo flere beslutninger eleven 
må ta selv, jo åpnere er oppgaven. For eksempel i den første del av øvelsen jobbet noen elever 
så og si selvstendig. Og senere var det noen elever som forsøkte å skrive av hverandres notater.
Dette kunne ha blitt tatt tak i hvis elevene fikk tydeligere begrensninger på hva de kan og ikke
kan gjøre, og hva hensikten med øvelsen er. Med andre ord, tydeligere intensjoner og læringsmål
kan forsikre at elevene oppnår tiltenkt læringsutbytte. 
% Når naturfag rettferdiggjøres som et fag i skolen bruker man ofte to typer argumenter, som blir
% omtalt som produkt-argumentet og prosess argumentet, \citeA[s. ~351]{sjob04}. Produkt-argumentet går 
% ut på at naturfaglige kunnskaper, begreper og teorier er viktige både for eleven i skolehverdagen og 
% senere i arbeidslivet. Prosess-argumentet går ut på at det er naturvitenskapens prosesser, 
% arbeidsmåter og metoder som rettferdiggjør fagets plass i skolen. \iffalse \citeauthor{sjob04} \else 
% Sjøberg \fi skriver at selv om det er noe ``pedagogisk tidsmessig og tiltrekkende'' ved det synet 
% at det er prosessene som er det vesentlige, må det understrekes at produktorientert syn trenger ikke 
% å medføre \emph{en autoritær og doserende metodisk tilnærming} når dette produktet skal formidles 
% til elevene. \iffalse \citeauthor{sjob04} \else Han \fi skriver videre at det er viktig at 
% vitenskapens egenart ikke automatisk dikterer en metodisk tilnærming, eller motsatt, at man lar et 
% syn på læring definere hva som skal oppfattes som vitenskapens egenart. Undervisningsopplegget har 
% hatt en preg av begge disse syn på vitenskapens vesen. Innføring av nye begreper har styrket 
% elevenes syn på naturfag som et produkt, mens deres observasjoner i laboratoriet og skriving av 
% rapport har forsterket deres syn på naturfag som en prosess. 
% \newline
% \newline
% En overordnet ramme for arbeid med Forskerspiren er at elevene skal praktisere en vitenskapelig 
% metode. På 1960-tallet i USA og England kom læreplaner som blir omtalt
% for \emph{discover-learning}, \citeA[s. ~31]{knai11}. Her skulle elevene lære naturvitenskapelig 
% kunnskap gjennom aktiviteter som skulle ligne naturvitenskapelig forskning. I følge \iffalse 
% \citeauthor{knai11} \else Knain \fi  er det flere svakheter ved denne retningen. En av dem var 
% tanken at barn lærer naturfaglig begrepskunnskap gjennom induksjon, det vil si ved å trekke 
% sluttninger fra erfaringer. \iffalse \citeauthor{knai11} \else Knain \fi skriver videre at
% \begin{displayquote}
% Som Hodson påpeker:
% \begin{displayquote}
% Du kan ikke oppdage noe som du mangler begreper om. Du vet ikke hvor du skal se, hvordan du skal se 
% eller hvordan du skal gjenkjenne det når du har funnet det (Hodson 1996, s. 118).
% \end{displayquote}
% \end{displayquote}
% Selv om kognitiv forståelse alltid vil være avgjørende viktig i naturfag, så kan vi her spørre om det ikke nettopp er økt evne til deltagelse som bør være 
% begrunnelsen for å utvikle elevenes begrepsforståelse. Fokuset på forståelse må derfor suppleres med vektlegging av lesing for at naturfaget skal forberede til 
% demokratisk deltagelse \citeA[s. ~72]{kols09}.
% \citeA[s. 67]{roen15} referer til Kolstø når han argumenter om elever som aktive samfunnsborgere.
% \begin{displayquote}
% Kolstø (2009) peker også på at forståelse av begrepene vil være nødvendig med tanke på deltakelse 
% i samfunnet for øvrig. Slike begreper må være forstått for å kunne ta del i diskusjoner og argumentasjon, 
% og er dermed en del av en naturfaglig allmenndannelse.
% \end{displayquote}
% Demokrati er avhengig av 
\newline
\newline
Utforskende samtaler må først innlæres i en klasse slik at elevene kan få mest mulig
utbytte av sine felles diskusjoner og samtaler. \citeA[s. 57]{meli07} beskriver dette som kjernen i 
praksisen:
\begin{displayquote}
At the heart of the approach is the negotiation by each teacher and class of a set of ``ground
rules'' for talking and working together. These ground rules then become established as a set of 
principles for how the children will collaborate in groups.
\end{displayquote}
Slike regler bør derfor etableres ved et tidlig stadie for en gitt klasse, noe \citeA[s. 151]{ogd09}
også understøtter. Elevene bør rutineres i å tillate rom for alternative løsninger, uten å true 
gruppens solidaritet eller individets identitet. Disse reglene kan innøves gjennom flere 
anledninger: helklassesamtaler, gruppesamtaler, og parsamtaler. Sistnevnte anledning er passende
for 8. klassen, siden alle elever har en læringspartner. Ved å bruke bordplasseringen som allerede 
er på plass frigjør dette organiseringstid som isteden kan brukes mot fagrettet læring. 
\citeA{klet13} legger vekt på effektive instrukser som bidrar til mer fagrettet undervisning og 
større fokus på kognitive utfordringer.
\newline
\newline
%Til den neste praksisperioden er jeg dessuten interessert i å utprøve hvordan grupper med forskjellige 
%permutasjoner kan dannes.  For eksempel kan elever som viser tydelige leder egenskaper jevnt fordeles i 
%forskjellige grupper. Dette kan bidra til å skape tydelige roller i grupper. Det som er hensikten med denne 
%tankegangen er at alle elever i en gruppe bør føle at de har en unik rolle i gruppen, og at de sammen kan 
%utforme et fellesprodukt.
%\newline
%\newline
For å oppsummere denne oppgaven vil jeg trekke frem de viktigste punktene. Det var ikke tydelige 
nok instrukser for første del av tokolonnenotat-øvelsen. Dette kan ha ført til at noen elever 
samarbeidet isteden for å kollaborere. Design av gruppeoppgaven bør utformes slik at elevene er 
nødt til å jobbe sammen. Oppgaven bør ikke være så enkel at elevene kan jobbe individuelt med 
deloppgavene, slik at det ikke er noen nødvendighet for elevene å jobbe sammen. Tilsvarende 
bør oppgaven ikke ha så høy vanskelighetsgrad slik at de ikke klarer å danne forståelse eller mening. 
En gruppeoppgave er da en oppgave som individet ikke klarer å utføre alene og som 
krever kollaborasjon. Åpne oppgaver er bedre egnet enn lukkede hvor fokuset er å finne 
en riktig svar. Dette er kanskje grunnen til at en sterk elev kan dominere samtalen 
(\citeNP[s. 31]{meli07}). Først og fremst er villigheten til deltagerne til å dele sin forståelse 
og ideer, og forsette med dette til tross for uenigheter i mellom, en faktor for en vellykket 
utforskende samtale. Positive relasjoner mellom elever er derfor avgjørende for å skape et støttende 
klima for kollaborasjon. \citeA{klet13} kategoriserer dette som en underkategori i 
undervisningsmessig støtte: støttende klima - et klassemiljø preget av respekt, toleranse og 
engasjement (\citeNP[s. 191]{klet13}). Ifølge \citeA{ungd15} virker støttende klima til å være 
godt ivaretatt i norske klasserom. Men kanskje den aller største svakheten i øvelsen var at det 
var ingen konsolideringsmuligheter for tokolonnenotat øvelsen.\footnote[5]{Praksisveileder valgte 
å utføre dette etter endt praksisperiode.} Den neste timen skulle ha inkludert gjennomgang av 
tokolonnenotat-øvelsen gjennom for eksempel helklassesamtale. Det ville også ha vært passende
å instruere elevene i å levere et sluttprodukt på for eksempel OneNote. Dette ville ha gitt meg
muligheten til å se nærmere på elevenes begrepsforståelse fra et annet perspektiv enn kun deres 
muntlige formidling.  

\end{document}
