\documentclass[12pt,twoside,onecolumn,norsk]{article}

\usepackage{a4}
\usepackage[margin=0.9in]{geometry}
\usepackage{pdfpages}
\usepackage{apacite}
\usepackage{jneurosci}
\usepackage[utf8]{inputenc}
\usepackage[T1]{fontenc}
\usepackage[norsk]{babel} 
\usepackage{url}
\usepackage{amsmath} % math package
\usepackage{relsize} % to use larger math symbols
\usepackage{amssymb} % for using blackboard letters (e.g R for real numbers)
\usepackage{tocbibind} % for adding the reference section in table of contents
\usepackage{fixltx2e} % for text in sub script
\usepackage{perpage} % for resetting footnote counter at each page
\usepackage{csquotes}
\usepackage{epigraph}
\usepackage{ragged2e}

\MakePerPage{footnote} %the perpage package command

\addto\captionsenglish{%
  \renewcommand{\figurename}{Figur}
}

\makeatletter
\def\@documentnocite#1{\@bsphack
  \@for\@citeb:=#1\do{%
    \edef\@citeb{\expandafter\@firstofone\@citeb}%
    \if@filesw\immediate\write\@auxout{\string\citation{\@citeb}}\fi
    \@ifundefined{b@\@citeb}{\G@refundefinedtrue
      \@latex@warning{Citation `\@citeb' undefined}}{}}%
  \@esphack}
\AtBeginDocument{\let\nocite\@documentnocite}
\makeatother

\begin{document}

\section*{Del 1 - Observasjon}
\subsection*{Sammendrag av undervisningssituasjonen}
Videocasen viser en undervisningssituasjon i en KRLE-time. Elevene sitter i grupper 
på 4 rundt runde bord med retning mot tavlen. Underveis i sekvensen benytter læreren 
av tavle og lystavle. Timen starter med introduksjon av nytt tema, ``Hellige skrifter'', 
der mål for dagens time er kristendommens hellige skrifter. Først instruerer læreren 
elevene til å diskutere sammen om det de vet allerede om bibelen. Deretter gjennomgår 
læreren elevenes forslag på tavlen. Ved slutten av (video)sekvensen starter læreren 
en PPT presentasjon, ``Fakta om bibelen'', hvor h*n instruerer elevene om å notere ned 
underveis. Jeg kommer nå til å gjennomgå noen flere detaljer fra hele videocasen, og 
diskutere sekvensen i lys av relevant teori.

\subsection*{Teoridrevet observasjon}
I videocasen starter læreren sekvensen med mål for timen, hvor hun redegjør tema 
elevene skal jobbe med i løpet av timen. Deretter initierer læreren elevene i en kort 
dialog, hvor hun spør elevene om hva den hellige skriften i kristendommen heter. 
Elevene rekker opp hånd for å svare. Helklassesamtalen har preg av IRE/F-metoden 
(\citeNP[s. 175]{klet13}; \citeNP[s. 82]{weos01}; \citeNP[s. 29]{merc95}), 
dvs. lærer tar initiativ(I), elev responderer(R) og responsen blir evaluert(E) 
og/eller kommentert(F). Til denne sekvensen rakk elevene opp hånda for å respondere. 
I videocasen var det mange elever som rakk opp en hånd. Ofte er ikke dette tilfellet, 
hvor det er få elever som er villige til å svare. Dette kan være uheldig siden 
brorparten av elevene ikke er aktive. I slike samtaler stiller læreren spørsmål og 
elevene svarer, evaluerer læreren svarene ut fra hvilke svar hun/han ønsker. Lærere kan 
ha tendens til snevre videre inn spørsmålene og hinte til elevene, mao. det 
litteraturen kaller ``cued elicitation'' (\citeNP[s. 26]{merc95}). Det fører til at 
elevene prøver å gjette hvilke svar læreren ønsker. Dette er ikke kommunikasjon i tråd
med læreplan (\citeNP[s.110]{olma15}). IRE/F-metoden dominerer matematikklasserommene 
(\citeNP[s.117]{olma15}). For å få gode samtaler som setter i gang tankeprosessene hos 
elevene, er det viktig å stille spørsmål av høyere orden, som oftere innledes med 
hvorfor, hvordan og på hvilke måter. Elevene må begrunne svarene. I videocasen stilte 
læreren en lukket spørsmål, hvor elevrespons kunne ikke utdypes videre. Det er også 
viktig å skape en kultur for å våge å si noe og å akseptere feilsvar i klassen. I 
videocasen viser elevene aktiv deltagelse og det er ingen indikasjon på at elevene har 
problemer med å uttrykke seg selv.
\newline
\newline
\citeA[s. 181]{klet13} viser til viktigheten av at lærere bruker metalæringssituasjoner til å 
fremme elevers læring. Dette inkluderer blant annet ``aktiviteter for å mobilisere elevenes 
eksisterende kunnskap innen et gitt område, eller aktiviteter med fokus på å reflektere og oppsummere læringsaktiviteten''. Lærernes metalæringsaktiviteter regnes som særlig avgjørende 
for å sikre elevenes læring (\citeNP[s. 186]{klet13}). Å bruke dette som et fast 
organiserende prinsipp, blir derimot sjelden gjennomført (\citeNP[s. 26]{odeg10}). 
I videocasen starter timen med tydelige mål og læringsintensjoner, og elevene får reflektert 
over temaet de skal undervises i. Av faktorer som har direkte effekt på elevenes læring, 
fremhever \citeA[s. 189]{klet13} et gjennomtenkt undervisningsopplegg som muliggjør at de bruker 
minimalt tid på ikke-faglige aktiviteter. Tydeligere intensjoner og læringsmål kan forsikre 
at elevene oppnår tiltenkt læringsutbytte.
\newline
\newline
Etter utspørringen blir elevene instruert i å jobbe i grupper, hvor de blir bedt om å 
snakke sammen om det de vet fra før om bibelen. Gjennom hele denne sekvensen går læreren 
rundt og observerer forskjellige grupper snakke sammen. I blant initierer læreren en 
gruppe i en dialog hvor h*n gjentar hensikten med øvelsen. En del av responsen fra 
elevene er stikkord eller begreper, og deres respons har ofte lite substans. Dialogen 
har preg av  IRE/F-metoden. I en av slike respons sier en elev: ``Er det ikke sånn at 
muslimene kan også bruke det gamle testamentet?'' Læreren gir en bekreftende kommentar 
og går så videre til andre grupper. Her kunne det gjerne tenkes at læreren brukte litt 
mer tid til å få eleven til å utdype hva h*n mener med dette, gjennom for eksempel bruk 
av ``revoicing'', til å gjenta og forsterke elevens forslag. Ifølge \citeA{klet13}, 
viser fravær av slike eksplisitte innramminger fra lærerens side at eleven blir 
sittende med et uklart kunnskapsinnhold og i verste fall feil begrepsforståelse 
(\citeNP[s. 175 - 176]{klet13}). For å kunne bruke revoicing mest mulig effektivt, må 
læreren raskt kunne bestemme om elevens respons har validitet og er relevant. Gjennom 
egen praksiserfaring har revoicing vært vanskelig å utføre. Ved å forutse elevsvar 
før elever i klassen blir initiert, kan misforståelser som ofte oppstår bli redegjort 
av læreren, og respons som ofte opptrer kan derfor tas stilling til. Dette krever 
imidlertid en god del erfaring fra læreren sin side. I \citeA[s. 401]{bta98} 
klassifiseres dette som "knowledge of content and students, KCS". Over tid vil en 
lærer danne omfattende KCS og dette kan dermed bidra til å øke kvaliteten på samtalene. 
Revoicing kan også brukes i andre sammenhenger, for eksempel i neste del av timen hvor 
læreren oppsummerer gruppeøvelsen på tavlen.
\newline
\newline
Etter at læreren har hørt på forskjellige grupper, går h*n tilbake til tavlen og 
starter neste del av øvelsen, som er et tankekart. Her fører læreren opp elevenes 
forslag, og h*n bruker aktivt gruppene og enkelt elevene h*n har hørt gjennom 
gruppeøvelsen. Det virker til å være en viss mønster i hvordan læreren klumper 
sammen forslag. For eksempel ligger relevante begrep og personligheter nærmere 
i forhold til hverandre, og noen forslag forgreiner seg i henhold til deres 
relevans. Dermed fremviser læreren en logisk oversikt over elevrespons. Det kunne 
igjen tenkes at elevrespons kunne utdypes videre gjennom for eksempel bruk av 
revoicing.
\newline
\newline
Gruppeøvelsen og fellesgjennomgang på tavlen kan karakteriseres som I-G-P 
metoden (\citeNP{heoi14}). I denne undervisningssekvensen kan overgangen fra 
lærerinitiert spørsmål til gruppearbeid anses som individ-delen av metoden, 
mens gruppearbeidet og fellesgjennomgang/plenum utgjør resten av aktiviteten.
Det kan derimot argumenteres at siden tilstrekkelig tid ikke var avsatt for 
refleksjon, var individperspektivet fraværende. Uansett ble metoden brukt 
til en viss grad for å lokke fram elevenes forkunnskaper. Før videocasen 
avslutter starter læreren en ny sekvens med en PPT presentasjon av temaet 
``Fakta om bibelen''. Elevene har nå en god forutsetning for å følge med 
gjennom presentasjonen og dermed få en større utbytte. Læreren har derfor 
klart å aktivere elevenes forkunnskaper og elevene har blitt varmet opp 
kognitivt.
\newline
\newline
Gjennom hele undervisningssekvensen har læreren demonstrert bruk av positive 
tilbakemeldinger, og ved et tilfelle viste h*n interesse i en elevs 
burdsdagsfeiring. Ifølge \citeA[s. 142 - 144]{klet13} kjennetegner god 
undervisning lærerens støttende rolle i kategorien emosjonell støtte. Hun 
framhever viktigheten av lærerens respons og verdsetting av elevinitiativ 
og bruk av ros/irettesettinger som avgjørende for kvaliteten på opplæringen. 
Gode sosiale relasjoner er med på å danne en omgivelse av respekt og læring. 
Ifølge \citeA{ungd15} virker støttende klima til å være godt ivaretatt 
i norske klasserom.

\section*{Del 2 - Analyse}
\subsection*{Problemstilling}
\citeA[s. 136]{klet13} beskriver en god undervisningssekvens der lærere 
klarer å balansere mellom tilegnelses-, utprøvings-, og 
konsolideringssituasjoner. Ifølge Klette har norske klasserom ensidige 
tendenser i bruken av varierte arbeidsmåter. Norsk 
matematikkundervisning fokuserer i stor grad på teoretisk gjennomgang, 
kombinert med individuell oppgaveløsning. Konsekvensen av dette kan 
bli at lite tid avsettes til muntlige aktiviteter som for eksempel det 
å forklare sine svar (\citeNP[s. 110]{olma15}). Det er derfor 
et relativt ensidig repertoar - enten helklasseundervisning og 
gjennomgang på tavla, eller arbeid med matematikkoppgaver individuelt. 
I den sosiokulturelle tradisjonen rettes fokus mot læring i felleskap 
før kunnskap blir internalisert på individnivå (\citeNP[s. 90]{rogs13}). 
Blant annet inkluderer dette arbeid i grupper. Gruppearbeid er lite 
brukt i matematikkundervisningen. Studier viser også liten bruk av 
medelevene som læringsressurser i matematikkfaget, og tilsvarende er 
det også lite samarbeidslæring i matematikktimene (\citeNP[s. 182]{klet13}). 
Gjennom videocasen i en undervisningssekvens ble gruppe samtaler 
brukt til å aktivisere elevene. Det kan derfor tenkes at et 
tilsvarende opplegg kan brukes i en matematikktime der det blir satt 
fokus på muntlige ferdigheter i matematikk.
\newline
\newline
Ifølge læreplanen for matematikk innebærer muntlige ferdigheter ``vere med i
samtalar, kommunisere idear og drøfte matematiske problem, løysingar og 
strategiar med andre'' (Læreplan for matematikk, K06). Bruk av samtale og 
kommunikasjon har derfor god støtte i læreplanens læringssyn. Ifølge 
\citeA[s. 15]{ole93} er elevens individuelle matematiske erfaringer i 
veldig stor grad knyttet til verbal kommunikasjon. I videocasen hadde elevenes 
samtaler varierende kvalitet. Noen elever brukte stikkord uten å gå dypere inn 
i begrepene, andre elever fremsatte fakta om visse begreper. En gruppe som 
jeg spesielt la merke til tok opp et kompleks tema (jamfør elevens henvendelse 
til læreren om muslimenes forhold til gamle testamentet). Samtalekvaliteten på 
gruppearbeid hadde derfor et stort spenn. \citeA[s. 58-59]{meli07} 
definerer tre distinkte klassifiseringer for slike samtaler: ``Disputational'', 
``Cumulative'' og ``Exploratory''. Den sist nevnte klassikasjonen, også kalt 
utforskende samtaler, utgjør gruppearbeid som har preg av kollaborasjon og 
dermed regnes som den mest ønskelige samtaleformen.
\newline
\newline
Min problemstilling er følgende:
\newline
\newline
\textbf{Hvordan bidrar utforskende samtaler til å øke samtalekvaliteten i en 
matematikktime?}

\section*{Læringsteorier}
Den sosiokulturell læringsteori blant annet understreker lærerens viktige 
rolle i undervisningen og betydningen av de sosiale rammene rundt våre 
handlinger. Den sosiokulturelle teorien har utgangspunkt i Lev Vygotsky 
sine perspektiver på læring og utvikling (\citeNP[s. 13]{meli07}; 
\citeNP[s. 123]{bta98}; \citeNP[s. 87]{rogs13}). Vygotskys
mente at barns intellektuelle utvikling er formet utfra tilegnelse av språk, 
fordi språk muliggjør dialog mellom mennesker (\citeNP[s. 5]{meli07}), 
en tanke Jerome Bruner også ville ha støttet. Bruner argumenterer at språk øker 
barns evne til å håndtere abstrakte konsepter (\citeNP{mcl08}) og dermed deres evne 
til å delta i faglige samtaler. Det sosiokulturelle synet fokuserer mer på 
spørsmål som gjelder kvaliteten på elevenes deltaking i læringsaktivitetene. I 
motsetning legger det konstruktivistiske synet mer vekt på om elevene forstår 
begrepene innenfor et fagområde og om de kan bruke de metodene og strategiene 
som er nyttige for å løse problemene i faget.
\newline
\newline
I kognitiv konstruktivisme utgjør elevenes erfaringer og kunnskaper det de 
kan møte nye utfordringer med Piaget kaller delstrukturer som utgjør disse 
kognitive strukturene for (mentale) skjemaer (\citeNP[s. 78]{solv92}). Piaget 
mente at elevenes tenkning ble utviklet gjennom to forskjellige typer 
prosesser: ``assimilasjon'' og ``akkomodasjon'' (\citeNP[s. 65]{rogs13}). 
I den første prosessen utvider elevene sine skjemaer, men det skjer ingen 
endring. Når elevene møter noe som ikke stemmer overens med de eksisterende 
skjemaer, endres skjemaene, altså akkomodasjon. Det er gjennom en kombinasjon 
av begge prosessene at elevene anskaffer den ønskelige type kunnskap, 
operativ kunnskap, i motsetning til figurativ kunnskap. Utforskende samtaler 
bør derfor utformes slik at de bidrar til å utvide elevenes skjemaer.


\section*{Utforskende samtaler}
Utforskende samtaler må først innlæres i en klasse slik at elevene kan få mest 
mulig utbytte av sine felles diskusjoner og samtaler. \citeA[s. 57]{meli07} 
beskriver dette som kjernen i praksisen:
\begin{displayquote}
At the heart of the approach is the negotiation by each teacher and class of a set of ``ground
rules'' for talking and working together. These ground rules then become established as a set of 
principles for how the children will collaborate in groups.
\end{displayquote}
Slike regler bør derfor etableres ved et tidlig stadie for en gitt klasse, noe 
\citeA[s. 151]{ogd09} også understøtter. Elevene bør rutineres i å tillate rom 
for alternative løsninger, uten å true gruppens solidaritet eller individets 
identitet. Disse reglene kan innøves gjennom ere anledninger: helklassesamtaler, 
gruppesamtaler, og parsamtaler. Gruppesamtaler er passende for klassen som blir 
observert i videocasen, siden alle elever har en plasseringer som tilrettelegger 
dette. Ved å bruke bordplasseringen som allerede er på plass frigjør dette 
organiseringstid som isteden kan brukes mot fagrettet læring.  \citeA{klet13} 
legger vekt på effektive instrukser som bidrar til mer fagrettet undervisning 
og større fokus på kognitive utfordringer. Gjennom helklassesamtaler er det ofte 
et fåtall elever som er aktive i lærerinitiert dialog. Dette var uheldig siden 
elevenes styrker og svakheter ikke blir tilstrekkelig avdekket. Gruppesamtaler 
kan derfpr være en god plattform for å avdekke hull og svakheter i elevenes 
begrepsbruk.
\newline
\newline
I videocasen var det noen elever som arbeidet så og si selvstendig, tiltross for 
at de var i en gruppe. Sluttresultatet var således ikke basert på et felles 
grunnlag. Dermed fikk de en annen type utbytte fra gruppearbeidet enn det som var 
tiltenkt, hva \citeA[s. 25]{meli07} definerer som ``group-sense
or feeling of a shared endeavour''. Med andre ord ble det et samarbeid og ikke en 
kollaborasjon, som \citeA{meli07}  respektivt kaller ``interacting vs. 
interthinking''. Gruppen endte opp med et felles produkt (i videocasen: punkter 
som læreren kan føre opp på tavlen), men det er ikke basert på en kollaborasjon 
mellom elevene. Målet med et samarbeid er å ende opp med et sluttprodukt. I en 
kollaborasjon frembringer individer egne ideer til gruppen, og hver ide blir da 
vurdert og diskutert felles i gruppen: enten blir den akseptert eller så forkastes 
den. En viktig del av den sosiale utprøvingen av ideer og begreper innebærer å 
sammenlikne egne forestillinger med andres forestillinger (\citeNP{odeg10}; 
\citeNP{dals94}).
\newline
\newline
Rollen til læreren ligger i å veilede elevene i den nærmeste utviklingssonen. 
Den \emph{nærmeste utviklingssonen} beskriver en sone som ligger i mellom en elevs 
kognitive ferdigheter, dvs. hva de kan oppnå selvstendig uten hjelp, og elevens 
potensielle utvikling, dvs. hva en elev kan få til eller forstå gjennom enten 
veiledning eller kollaborasjon (\citeNP[s. 14]{meli07}; \citeNP[s. 125]{bta98}; 
\citeNP[s. 75]{rogs13}). Ved bruk av scaffolding eller 
stillasbygging (\citeNP{bta98}; \citeNP[s. 71]{math15}) kan elevenes begrepsbruk 
knyttes til deres forkunnskaper og hverdagsoppfatninger. Gode fagsentrerte 
samtaler mellom elever (eller faglige samtaler med lærer) hvor elever bruker 
egne erfaringer og språk for å oppnå faglig forståelse hjelper til å skape bro 
mellom praksis og teori (\citeNP{odeg10}).
\newline
\newline
Design av gruppeoppgaven bør utformes slik at elevene er nødt til å jobbe sammen. 
Oppgaven bør ikke være så enkel at elevene kan jobbe individuelt med deloppgavene, 
slik at det ikke er noen nødvendighet for elevene å jobbe sammen. Tilsvarende bør 
oppgaven ikke ha så høy vanskelighetsgrad slik at de ikke klarer å danne forståelse 
eller mening. En gruppeoppgave er da en oppgave som individet ikke klarer å utføre 
alene og som krever kollaborasjon. Åpne oppgaver er bedre egnet enn lukkede hvor 
fokuset er å finne en riktig svar. Dette er kanskje grunnen til at en sterk elev 
kan dominere samtalen (\citeNP[s. 31]{meli07}).  En fremgangsmåte kan være at 
elever drøfter med problemet med hverandre i gruppa først for å få klarhet i hva 
oppgaven består i, deretter kan de sette seg inn i problemet og få ideer fra 
hverandre. Elever skal deretter begrunne sine løsningsforslag med hverandre. De må 
først forsto selv hva de har gjort for så etter at de er fortrolig med løsningen, 
forklare den for andre. Andre eksempler på bruk av samtaler og diskusjon i 
matematikkundervisningen kan være (\citeNP[s. 111]{olma15}):
\begin{enumerate}
\item Bekrefte eller avsanne påstander (eksempel : Diskuter påstanden - å multiplisere 
et tall med 10 er samme som å legge til 0 til tallet).
\item Forklare aktiviteter (eksempel : Forklar hvordan du konstruerer en likesidet 
trekant der to vinkler er 45 grader).
\item Forklare eller tolke informasjon som ligger i for eksempel tabeller og 
diagrammer.
\item Lærerledet undervisning der elevene veiledes mot muntlige oppgaver og utledning 
av fremgangsmåter.
\item Lek og konkurranser i grupper.
\end{enumerate}
Ved design av utforskende samtaler, en annen viktig ting å ta hensyn til er det 
\citeA[389 og 393]{cuoc96}) beskriver som erke typer blant matematikere. Det er 
viktig at elevene kan bruke
den typen de assosierer seg med: "algebraikere" eller "geometere". Det er kanskje 
interessant å undersøke hvordan slike elever samarbeider i grupper.
\newline
\newline
Det er først og fremst er villigheten til deltagerne til å dele sin forståelse og 
ideer, og forsette med dette til tross for uenigheter imellom, en faktor for en 
vellykket utforskende samtale. Positive relasjoner mellom elever er derfor 
avgjørende for å skape et støttende klima for kollaborasjon. \citeA{klet13} 
kategoriserer dette som en underkategori i undervisningsmessig støtte: støttende 
klima et klassemiljø preget av respekt, toleranse og engasjement 
(\citeNP[s. 191]{klet13}).

\bibliographystyle{apacite}
\bibliography{database}

\end{document}
