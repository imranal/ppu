\documentclass[12pt,twoside,onecolumn,norsk]{article}

\usepackage{a4}
\usepackage[margin=0.9in]{geometry}
\usepackage{pdfpages}
\usepackage{apacite}
\usepackage{jneurosci}
\usepackage[utf8]{inputenc}
\usepackage[norsk]{babel} 
\usepackage{url}
\usepackage{amsmath} % math package
\usepackage{relsize} % to use larger math symbols
\usepackage{amssymb} % for using blackboard letters (e.g R for real numbers)
\usepackage{tocbibind} % for adding the reference section in table of contents
\usepackage{fixltx2e} % for text in sub script
\usepackage{perpage} % for resetting footnote counter at each page
\usepackage{csquotes}
\usepackage{epigraph}
\usepackage{ragged2e}

\MakePerPage{footnote} %the perpage package command

\addto\captionsenglish{%
  \renewcommand{\figurename}{Figur}
}

\makeatletter
\def\@documentnocite#1{\@bsphack
  \@for\@citeb:=#1\do{%
    \edef\@citeb{\expandafter\@firstofone\@citeb}%
    \if@filesw\immediate\write\@auxout{\string\citation{\@citeb}}\fi
    \@ifundefined{b@\@citeb}{\G@refundefinedtrue
      \@latex@warning{Citation `\@citeb' undefined}}{}}%
  \@esphack}
\AtBeginDocument{\let\nocite\@documentnocite}
\makeatother

\begin{document}


\section*{God undervisning}
%%%%% God undervisning %%%%%%%%%
\citeA[s. 136]{klet13} beskriver en god undervisningsseksens der lærere klarer å balansere mellom 
tilegnelses-, utprøvings-, og konsolideringssituasjoner. Ifølge Klette har norske klasserom ensidige 
tendenser i bruken av varierte arbeidsmåter. Lærernes metalæringsaktiviteter regnes som særlig 
avgjørende for å sikre elevenes læring (\citeNP[s. 186]{klet13}). Å bruke dette som et fast
organiserende prinsipp, blir derimot sjelden gjennomført (\citeNP[s. 26]{odeg10}). Gjennom egen 
praksiserfaring har mine timer inkludert aktivering av forkunnskaper, gjennom repitisjon og 
gjenbruk av begreper og gjennomgang av lekser, bæret preg av konsolideringssituasjoner.
\newline
\newline
Av faktorer som har direkte effekt på elevenes 
læring, fremhever \citeA[s. 189]{klet13} et gjennomtenkt undervisningsopplegg som muliggjør at 
de bruker minimalt tid på ikke-faglige aktiviteter. Tydeligere intensjoner og læringsmål
kan forsikre at elevene oppnår tiltenkt læringsutbytte. 
\newline
\newline
"Bilde av norsk matematikkundervisning blir avtegnet som i stor grad begrenser seg til 
teoretisk gjennomgang, kombinert med individuell oppgaveløsning. Konsekvensen av dette
kan bli at lite tid avsettes til muntlige aktiviteter som for eksempel det å forklare
sine svar" (\citeNP[s.110]{olma15}). "Matematikk er kjennetegnet med et relativt
ensidig repertoar - enten helklasseundervisning og gjennomgang på tavla, eller arbeid
med matematikkoppgaver individuelt. Gruppearbeid er lite brukt i matematikkundervisningen.
Studier viser også liten bruk av medelevene som læringsressurser i matematikkfaget, og
tilsvarende også lite samarbeidslæring i matematikktimene"(\citeNP[s. 182]{klet13}).

\section*{Helklassesamtale}
%%%%% Helklassesamtale %%%%%%%
Helklassesamtalene hadde preg av
IRE/F-metoden (\citeNP[s. 175]{klet13}; \citeNP[s. 82]{weos01}), dvs. lærer tar initiativ(I), elev 
responderer(R) og responsen blir evaluert(E) og/eller kommentert(F). Til denne sekvensen rakk 
elevene opp hånda for å respondere. Det viste seg at det var få elever som var villig til å svare. 
Dette var uheldig siden flerparten av elevene ikke var aktive. "I blant stiller læreren spørsmål og
elevene svarer, evalurer læreren svarene ut fra hvilkee svar hun/han ønsker. Læreren sneverer videre
inn spørsmålene og hinter til elevene. Det fører til at elevene prøver å gjette hvilke svar læreren
ønsker. Dette er ikke kommunikasjon i tråd med læreplan (\citeNP[s.110]{olma15})". IRE/F-metoden
dominerer matematikkklasserommene (\citeNP[s.117]{olma15}). "For å få gode samtaler som setter i
gang tankeprosessene hos elevene, er det viktig å stille spørsmål av høyere orden, som oftere
innledes med hvorfor, hvordan og på hvilke måter. Elevene må begrunne svarene." "Det er viktig
å skape en kultur for å våge å si noe og å akseptere feilsvar i klassen."
\newline
\newline
Det var ikke tilstrekkelig bruk av ``revoicing'' gjennom denne sekvensen, til å gjenta og 
forsterke elevenes forslag.  Ifølge Klette, viser fravær av slike eksplisitte 
innramminger fra lærerens side at eleven blir sittende med et uklart kunnskapsinnhold og i 
verste fall feil begrepsforståelse (\citeNP[s. 175-176]{klet13}). For å kunne bruke revoicing 
mest mulig effektivt, må læreren raskt kunne bestemme om elevens repons har validitet 
og er relevant. Gjennom egen praksiserfaring har revoicing vært vanskelig å utføre.
Ved å forutse elevsvar før elever i klassen blir initiert, kan misforståelser som ofte oppstår bli 
redegjort av læreren, og respons som ofte opptrer kan derfor tas stilling til. Dette krever imidlertid 
en god del erfaring fra læreren sin side. I \citeA[s. ~401]{batp08} klassifiseres dette som 
``knowledge of content and students, (KCS)''. Over tid vil en lærer danne omfattende KCS og
dette kan dermed bidra til å øke kvaliteten på helklassesamtalene. Revoicing kan også brukes
i andre sammenhenger, for eksempel i neste del av timen hvor jeg innførte et nytt tema. 
\newline
\newline
I helklassesamtalene ble elevene spurt om det de har hatt til lekse.
Siden de blir engasjert i samtaler rundt lekser de skal ha utført, har de forutsetning for å kunne 
respondere på lærerinitiativ. Det er ønskelig å få bekreftet at elevene innehar en overordnet 
forståelse. Det kan derfor være nødvendig å utpeke noen elever som ikke viser aktiv deltagelse i 
timen og frembringe deres respons. Gjennom min egen praksiserfaring brukte jeg navnekort
til å engasjere vilkårlige elever. Noen elever viste således svakheter i deres forståelse.
I blant valgte jeg å hjelpe eleven korrigere sin kunnskap ved å hjelpe de til å komme
fram til en korrekt oppfatning, og noen ganger valgte jeg å la andre elever slippe inn i dialogen
slik at de kunne bidra med egen oppfatning.
\newline
\newline
Hvis elevene ikke klarer å respondere på lærer initiativ, kan utspørringen av elevene vise hull 
i deres kunnskap. Derimot har utpeking av elever også noen negative implikasjoner. For eksempel 
vil noen elever føle ubehag av å bli utpekt (\citeNP[s. 82]{weos01}). Det er ønskelig å trene 
elevene i å aktivt delta i undervisningen, men det er også lurt å ikke forsterke negative 
assosiasjoner til slik deltagelse. Hvis svake elever blir engasjert, bør de få muligheten til å 
kunne demonstrere sin mestring av temaer de er fortrolig og godt kjent med (\citeNP{spr02}). 
Uansett er det nødvendig at elevene får autentiske mestringssituasjoner slik at deres forventninger
om mestring øker (\citeNP[s. 163]{mang13}). Hvis en av intensjonene med helklassesamtalen var å 
finne hull i elevenes kunnskapsnivå, ble gruppesamtaler en bedre arena for meg.

\section*{Læringsteorier}
%%%%% Læringsteorier %%%%%%%%
Den sosiokulturelle teorien har utgangspunkt 
i Lev Vygotsky sine perspektiver på læring og utvikling (\citeNP[s. 13]{meli07}; \citeNP[s. 123]{bta98}; 
\citeNP[s. 87]{rogs13}). Vygotskys mente at barns 
intellektuelle utvikling er formet utfra tilegnelse av språk, fordi språk muliggjør dialog mellom 
mennesker (\citeNP[s. 5]{meli07}), en tanke Jerome Bruner også ville ha støttet. Bruner argumenterer 
at språk øker barns evne til å håndtere abstrakte konsepter (\citeNP{mcl08}).
\newline
\newline
I kognitiv konstruktivisme utgjør elevenes erfaringer og kunnskaper det de kan møte nye 
utfordringer med. Piaget kaller delstrukturer som utgjør disse kognitive strukturene 
for (mentale) skjemaer (\citeNP[s. 78]{solv92}). Piaget mente at elevenes tenkning ble 
utviklet gjennom to forskjellige typer prosesser: ``assimilasjon'' og ``akkomodasjon'' 
(\citeNP[s. 65]{rogs13}). I den første prosessen utvider elevene sine skjemaer, men
det skjer ingen endring. Når elevene møter noe som ikke stemmer overens med de
eksisterende skjemaer, endres skjemaene, altså akkomodasjon. Det er gjennom en 
kombinasjon av begge prosessene at elevene anskaffer den ønskelige type kunnskap,
operativ kunnskap, i motsetning til figurativ kunnskap.

\section*{Matematikk - to type matematikere}
%%%% Matematikk - to type matematikere %%%%%%
"Geometric thinking is an absolute necessity in every branch of mathematics, and,
throughout history, the geometric point of view has provided exactly the right
insight for many investigations (e.g., complex analysis).6 Geometers (amateurs
and professionals) seem to have a special stash of tricks of the trade. 
\newline
\newline
Algebra is a language for expressing mathematical ideas (there are certainly
others), and, like any language, it consists of much more than a way to represent
objects with symbols. There are algebraic habits of mind that center around ways
to transform the symbols. For algebraists, the images of these transformations
are so strong and pervasive that the symbols take on a life of their own, until they
become objects that exist as tools for informing one about the nature of the
transformations. " (\citeNP[389 og 393]{cuoc96})

\section*{Muntlig matematikk}
%%%% Muntlig matematikk  %%%%%%%%
"Munnlege ferdigheiter i matematikk inneber å skape meining gjennom å lytte, tale og 
samtale om matematikk. Det inneber å gjere seg opp ei meining, stille spørsmål og 
argumentere ved hjelp av både eit uformelt språk, presis fagterminologi og omgrepsbruk. 
Det vil seie å vere med i samtalar, kommunisere idear og drøfte matematiske problem, 
løysingar og strategiar med andre. Utvikling i munnlege ferdigheiter i matematikk går 
frå å delta i samtalar om matematikk til å presentere og drøfte komplekse faglege emne. 
Vidare går utviklinga frå å bruke eit enkelt matematisk språk til å bruke presis 
fagterminologi og uttrykksmåte og presise omgrep." (Læreplan for matematikk, K06)
\newline
\newline
Bruk av samtale og kommunikasjon har derfor god støtte i læreplanens læringssyn.
"Elevens individuelle matematiske erfaringer er i veldig stor grad knyttet til verbal
kommunikasjon" (\citeNP[s. 15]{ole93}).
\newline
\newline
Drøfte med problemet med hverandre i gruppa først for å få klarhet i hva oppgaven
består i, sette seg inn i problemet og få ideer fra hverandre.
Elever skal deretter begrunne sine løsningsforslag med hverandre. De må først forsto
selv hva de har gjort for så etter at de er fortrolig med løsningen, forklare den 
for andre.
\newline
\newline
Eksempeler på bruk av samtaler og diskusjon i matematikkundervisningen :
1) Bekrefte eller avsanne påstander (eksempel : Diskuter påstanden - å multiplisere
et tall med 10 er samme som å legge til 0 til tallet).
2) Forklare aktiviteter (eksempel : Forklar hvordan du konstruerer en likesidet 
trekant der to vinkler er 45 grader).
3) Forklare eller tolke informasjon som ligger i for eksempel tabeller og 
diagrammer.
4) Lærerledet undervisning der elevene veiledes mot muntlige oppgaver og utledning
av fremgangsmåter.
5) Lek og konkurranser i grupper.

\section*{Utforskende samtaler}
%%%%% Utforskende samtaler %%%%%%%
I den sosiokulturelle tradisjonen rettes 
fokus mot læring i felleskap før kunnskap blir internalisert på individnivå (\citeNP[s. 90]{rogs13}). 
Blant annet inkluderer dette arbeid i grupper. Samtalekvaliteten på gruppearbeid kan ha et stort spenn. 
\citeA[s. 58-59]{meli07} definerer tre distinkte klassifiseringer for slike samtaler:
``Disputational'', ``Cumulative'' og ``Exploratory''. Den sist nevnte klassifikasjonen,
også kalt utforskende samtaler, utgjør gruppearbeid som har preg av kollaborasjon og dermed 
regnes som den mest ønskelige samtaleformen. 
\newline
\newline
Utforskende samtaler må først innlæres i en klasse slik at elevene kan få mest mulig
utbytte av sine felles diskusjoner og samtaler. \citeA[s. 57]{meli07} beskriver dette som kjernen i 
praksisen:
\begin{displayquote}
At the heart of the approach is the negotiation by each teacher and class of a set of ``ground
rules'' for talking and working together. These ground rules then become established as a set of 
principles for how the children will collaborate in groups.
\end{displayquote}
Slike regler bør derfor etableres ved et tidlig stadie for en gitt klasse, noe \citeA[s. 151]{ogd09}
også understøtter. Elevene bør rutineres i å tillate rom for alternative løsninger, uten å true 
gruppens solidaritet eller individets identitet. Disse reglene kan innøves gjennom flere 
anledninger: helklassesamtaler, gruppesamtaler, og parsamtaler. Sistnevnte anledning er passende
for 8. klassen, siden alle elever har en læringspartner. Ved å bruke bordplasseringen som allerede 
er på plass frigjør dette organiseringstid som isteden kan brukes mot fagrettet læring. 
\citeA{klet13} legger vekt på effektive instrukser som bidrar til mer fagrettet undervisning og 
større fokus på kognitive utfordringer.
\newline
\newline
I helklassesamtalen ved starten av timen var det bare et fåtall elever som var aktive i lærerinitiert 
dialog. Dette var uheldig siden elevenes styrker og svakheter ikke ble tilstrekkelig avdekket. 
Gruppesamtalene kan være en god plattform for å avdekke hull og svakheter i elevenes begrepsbruk.
\newline
\newline
Noen grupper viser akkumulative tendenser. Det vil si at elevene 
er villige til å akseptere hverandres bidrag uten å stille kritiske spørsmål og fremsette
alternative eller utfyllende forklaringer. Noen elever som jobbet i en gruppe, arbeidet så og si
selvstendig. Sluttresultet var
således ikke basert på et felles grunnlag. Dermed fikk de en annen type utbytte fra gruppearbeidet
enn det som var tiltenkt, hva \citeA[s. 25]{meli07}  definerer som ``groupsense or feeling of a 
shared endeavour''. Med andre ord ble det et samarbeid og ikke en kollaborasjon, som \citeA{meli07} 
respektivt kaller \emph{interacting vs. interthinking}. Gruppen endte opp med et felles produkt, 
men det var ikke basert på en kollaborasjon mellom elevene. Målet med et samarbeid er å ende opp 
med et sluttprodukt. I en kollaborasjon frembringer individer egne ideer til gruppen, og hver ide
blir da vurdert og diskutert felles i gruppen: enten blir den akseptert eller så forkastes den.
En viktig del av den sosiale utprøvingen av ideer og begreper innebærer å sammenlikne egne 
forestillinger med andres forestillinger i tillegg til naturvitenskapens forklaringer 
(\citeNP{odeg10}; \citeNP{dals94}).
\newline
\newline
Min rolle som lærer i denne øvelsen lå i å veilede elevene i den nærmeste utviklingssonen.
Den \emph{nærmeste utviklingssonen} beskriver en sone som ligger i mellom en elevs kognitive 
ferdigheter, dvs. hva de kan oppnå selvstendig uten hjelp, og elevens potensielle utvikling, dvs. 
hva en elev kan få til eller forstå gjennom enten veiledning eller kollaborasjon 
(\citeNP[s. 14]{meli07}; \citeNP[s. 125]{bta98}; \citeNP[s. 75]{rogs13}). Jeg brukte ``scaffolding'' 
eller stillasbygging (\citeNP{bta98}; \citeNP[s. 71]{math15}) for å knytte begrepene til elevenes 
forkunnskaper og hverdagsoppfatninger.  Gode fagsentrerte samtaler mellom elever (eller faglige 
samtaler med lærer) hvor elever bruker egne erfaringer og språk for å oppnå faglig forståelse 
hjelper til å skape bro mellom praksis og teori (\citeNP{odeg10}).
\newline
\newline
Design av gruppeoppgaven bør utformes slik at elevene er 
nødt til å jobbe sammen. Oppgaven bør ikke være så enkel at elevene kan jobbe individuelt med 
deloppgavene, slik at det ikke er noen nødvendighet for elevene å jobbe sammen. Tilsvarende 
bør oppgaven ikke ha så høy vanskelighetsgrad slik at de ikke klarer å danne forståelse eller mening. 
En gruppeoppgave er da en oppgave som individet ikke klarer å utføre alene og som 
krever kollaborasjon. Åpne oppgaver er bedre egnet enn lukkede hvor fokuset er å finne 
en riktig svar. Dette er kanskje grunnen til at en sterk elev kan dominere samtalen 
(\citeNP[s. 31]{meli07}). Først og fremst er villigheten til deltagerne til å dele sin forståelse 
og ideer, og forsette med dette til tross for uenigheter i mellom, en faktor for en vellykket 
utforskende samtale. Positive relasjoner mellom elever er derfor avgjørende for å skape et støttende 
klima for kollaborasjon. \citeA{klet13} kategoriserer dette som en underkategori i 
undervisningsmessig støtte: støttende klima - et klassemiljø preget av respekt, toleranse og 
engasjement (\citeNP[s. 191]{klet13}). Ifølge \citeA{ungd15} virker støttende klima til å være 
godt ivaretatt i norske klasserom.

\newpage\null
\bibliographystyle{apacite}
\bibliography{database}

\end{document}