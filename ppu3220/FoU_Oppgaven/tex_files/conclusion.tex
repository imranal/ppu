% !TEX spellcheck = en
\documentclass[main.tex]{subfiles} 

\begin{document}

\section*{Konklusjon}
\label{sec:6}
En kartleggingsprøve i sannsynlighetsregning ble designet og brukt i mitt vuderingsarbeid
gjennom praksis. Gjennom kartleggingsprøven kom elevenes misoppfattelser frem. Enda mer pressende 
informasjon jeg fikk gjennom kartleggingsprøven, var svakheter hos noen elever i grunnleggende 
ferdighet regning, som inkluderer tallforståelse. Jeg brukte kartleggingsprøven til å gi elever tilbakemeldinger 
og fremovermeldinger, både skriftlig og muntlig. Gjennom personlige elevsamtaler fikk jeg en bedre 
innblikk i elevenes tankegang. Jeg brukte elevsamtaler til å veilede elevene videre i deres faglig progresjon. 
Ved utforming av en kartleggingsprøve er det viktig å ikke lage spørsmål der elever kan svare ja eller nei. 
Det er nødvendig å bruke åpne spørsmål slik at elevenes arbeidsmåter og misoppfattelser kan evalueres. 
Gjennom utforming av gode diagnostiske oppgaver kan en underviser tilrettelegge sin undervisning etter 
informasjon som akkumuleres gjennom underveisvurderinger. Jeg vil videre benytte meg av kartleggingsprøver 
med større fokus på diagnostiske oppgaver slik at jeg kan veilede mine elever fra lav 
kompetansenivå til høyere kompetansenivå. I mitt videre arbeid vil jeg gjerne undersøke hva 
utgjør gode diagnosiske oppgaver og hvordan kan jeg bruke de i vurderingsarbeidet til å fremme 
læring.
\end{document}
