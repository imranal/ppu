\documentclass[main.tex]{subfiles} 
\begin{document}

\section*{Resultater \& Drøfting}
\label{sec:4}

Når elever skal vurderes så kan dette gjøres på flere måter:
\begin{itemize}
\item Normalfordeling og fast poengsum : da blir vurderingsgrunnlaget \emph{de andre elevenes prestasjoner}. 
Dette blir også referert som relativ vurdering. Vurdering av en individ avhenger da av de andre
elevenes prestasjoner. Ved fast poengsum innebærer det at det er etter poeng oppnåelse elevene blir vurdert. 
Da vil karakterene avgjøres ut ifra hvor mye peongsum eleven har klart å oppnå. Her vil ofte vanskelig oppgaver
bli vektlagt mer enn enkelere oppgaver.
\item Individrelatert kriterier : da vurderes eleven utelukkende i forhold til sine egne forutsetninger
og tidligere prestasjoner. I grunnskolen skal vurderingen uten karakterer i hovedsak være 
individrelatert  (\citeNP[s. 25]{hell07}).
\item Målrelatert kriterier : kvalitetsstandarden blir da en didaktisk kontretisering av kompetansemål.
I grunnskolen skal vurderingen med karakterer skje etter målrelaterte kriterier (\citeNP[s. 26]{hell07}).
\item Kompetansemål : da vurderes elevene utfra hvilket nivå eller trinn (i henhold til Bloom's taksonomi) 
                      de demonstrer i sin oppnåelse av kompetansemålene.
\end{itemize}
Jeg er nok enig i at bruken av individrelatert vurdering er en god vurderingsgrunnlag i situasjoner
der karakterer ikke brukes. Denne vurderingsformen oppfyller kriterier for god vurdering, siden den brukes til å 
fortelle eleven hvor hen befinner seg i sin faglig progresjon. Når lærer og elev sammen setter individuelle mål, både 
nærliggende og langsikte, da vil eleven gjennom et slikt vurderingsgrunnlag få konkete tilbakemeldinger og 
fremovermeldinger som fokuserer på nettopp elevens prestasjoner og målsettinger som hen har laget sammen med
læreren (\citeNP[s. 200]{olma15}). Det kan også være fordelsaktig å koble målsettingene til kompetansenivå eleven 
demonstrerer og jobbe mot høyre kompetansenivå. For eksempel diskutere er et høyt kompetansenivå, der eleven 
kan trekke sammnenhenger og redegjøre for sine tanker om en problemstilling. I motsetning er å beskrive et middels 
kompetansenivå (i Bloom's taksonomi).

Ved kartlegging av elevenes svakheter og styrker, da er det passende å bruke
individrelaterte kritier (\citeNP[s. 25]{hell07}) og koble inn kompetansemålene. Til en kartleggingsprøve så er det 
vanskelig å trekke inn individrelaterte kriterier med mindre lærer har godt kjennskap til eleven på forhånd. Jeg 
koblet dessverre ikke inn kompetansemålene heller, noe som det bør brukes mer av. Gjennom noen av mine samtaler 
med elever, oppdaget jeg fort at det var få som trakk forbindelsen mellom egen læring og koblingen til kompetansemålene. 
For undervisere regnes det som en god praksis at elevene er alltid bevisste om hvorfor de lærer det de lærer og hvor 
de er på vei. \citeA[s. 136]{klet13} beskriver en god undervisningsseksens der lærere klarer å balansere mellom 
tilegnelses-, utprøvings-, og konsolideringssituasjoner. Ifølge Klette har norske klasserom ensidige tendenser i bruken 
av varierte arbeidsmåter. Slik det kan ses fra figur \ref{fig:odeg10}, er det for eksempel lite 
konsolideringssituasjoner. Lærernes metalæringsaktiviteter regnes som særlig avgjørende for å sikre elevenes læring 
(\citeNP[s. 186]{klet13}). Å bruke dette som et fast organiserende prinsipp, blir derimot sjelden gjennomført 
(\citeNP[s. 26]{odeg10}).
\begin{figure}[h!]
\includegraphics[scale = 0.6]{../figures/undervisnings_aktivitet.png}
\caption{Oversikt over naturfaglærernes undervisningstilbud til elevene fra PISA+ studie. Kilde: 
\protect\citeA{odeg10}.}
\label{fig:odeg10}
\end{figure}
\newline

Det var på forhånd avtalt med elevene at kartleggingsprøven var ikke en del av summativ vurdering, men skulle brukes
som den del av formativ vurdering. Derimot, siden jeg hadde ikke valgt å gi elevene enn slutt vurdering, annet enn en 
poengsum, hadde det en mulig uheldig effekt. Fra mine observasjoner og samtaler med elevene kom det frem at noen av 
elevene anstrengte ikke like hardt som de ellers ville ha gjort hvis kartleggingsprøven var en "virkelig" prøve.
Gjennom elevsamtale med en av de sterke elevene, spurte jeg eleven hvorfor hen ikke hadde besvart en spesifikk 
oppgave og eleven svarte med å si at oppgaven var lett, men hen ``orket'' ikke å gå gjennom den. Grunnen hen oppga 
var at siden det ikke var en prøve, hadde det ikke så mye betyning. Dette samsvarer godt med det \citeA[s. 3]{brbl14} 
skriver : \emph{Vurdering kan ha en betydelig påvirkning på hvordan elever jobber, fordi de oppfatter det som 
vurderes som det eneste ``som teller''}.
Jeg hadde dessuten vektlagt de ``vanskelige'' oppgavene mye mer enn de ``enkle''. Nesten alle elever på tvers av nivå og 
ferdigheter hadde problemmer med å løse disse oppgavene og veldig få klarte å gå over en score på 5 ut av 10. Dermed 
fikk jeg ikke veldig mye informasjon om elevene gjennom poengsum. Ofte var det de over middelssterke elevene i klassen 
som tangerte mot 5 i poengsum. Kun en elev klarte å oppnå en score på 8.5. Videre nå vil jeg snakke om individuelle 
oppgaver fra kartleggingsprøven og diskutere elevenes feiltolninger. Det er viktig å merke at hvis dette en prøve så 
ville det ha vært viktig å skape situasjoner der elever kan vise mestring (\citeNP[s. 199]{olma15}). Siden dette var en 
kartleggingsprøve var fokuset isteden rettet mot å avklare elevens misoppfattelser og få en oversikt over hvor
elever ligger i sin læringsprossess. Her vil jeg derfor påstå at det er viktig å gjøre et slikt skille, men det
bør selvfølgelig skapes litt rom for å la elevene kjene på mestringsfølelsen. Dette vil jeg ta hensyn til videre i
vurderingsarbeidet.

\subsection*{Elevenes feiltolkninger}
Ifølge \citeA[s. 15]{brek02} og \citeA[s. 170]{olma15} kan diagnostiske oppgaver bli brukt til å identifisere og 
fremheve misoppfatninger som elevene har utviklet, gi læreren informasjon om elevenes løsningsstrategier og måle hvordan 
undervisningen har hjulpet elevene til å overvinne misoppfatningene. Gjennom blant annet kartleggingsprøver får elever 
muligheter til å utrykke sine skriftlige ferdigheter i matematikk. Å skrive matematikk regnes som en av grunnleggende 
ferdighetene. Det innebærer blant å beskrive og forklare egen tankgegang, å lage tegninger og skissere grafer. 
Skriving i matematikk blir sett på som et redskap for å utvikle egne tanker og egen læring (\citeNP{udirGF}). 

Pyskologene Daniel Kahneman og Amos Tversky har satt fram en teoretisk 
ramme for å undersøke læring av sannsynlighet og statistikk. Deres tese er at mennesker uten erfaring, refleksjon og 
innsikt i statistikk, bruker følgende strategier for å bedømme sannsynlighet (\citeNP{udir13}; \citeNP{evan17}):
\begin{itemize}
\item Representativitet : små utvalg skal representere den fordelingen som finnes i populasjonen
\item Tilgjengelighet : sannsynlighet bedømmes ut fra hvor lett det er å huske spesielle tilfeller
\item Resultatorientering : utfallet kan forutses, som ved en deterministisk prosess
\item Konjunksjonsfellen : sannsynligheten for at to hendelser inntreffer samtidig er mindre enn sannsynligheten
for at en av hendelsene inntreffer.
\item Generelt vanskeligheter med betinget sannsynlighet : dvs. vanskeligheter med sannsynligheter hvor et utfall
avhenger av foregående utfall. Imotsetning er utbetinget sannsynlighet utfall der en begivenhet forekommer uavhengig 
av tidligere utfall. 
\end{itemize}
Dette er ofte misoppfattelser som også ligger hos elever ved ulike faglignivå. For eksempel en vanlig feil
mange elever gjør er at de blander addisjonsregelen med muliplikasjonsregelen. Det kan godt hende at dette
skyldes at de bruker eksempeler de husker gjennom informasjon de har tilgjengelig. F.eks. hvis en oppgave minner
dem om et tilfelle der de brukte addisjonsregelen, vil de prøve å bruke den, selv når situasjonen krever bruk
av multiplikasjonsregelen. Dette har jeg observert både gjennom undervisning av sannsynlighetsregning og
gjennom kartleggingsprøven (se figure : \ref{fig:mohsin}).
\newline



\subsection*{Tilbakemeldinger}

William redegjør for hvorfor tilbakemeldinger noen ganger kan føre til senking 
i elvenes ytelse. Han referer til Kluger og DeNisi (1996), når han summerer opp 
\begin{displayquote}
\textelp{} feedback was least effective when it focused on the task in hand, 
and more effective when it focused on the details at hand, and most effective 
when it focused on the details of the task and involved goal-setting.
(\citeNP[s. 140]{will10})
\end{displayquote}



\end{document}
