\documentclass[main.tex]{subfiles} 
\begin{document}

\section*{Introduksjon}

Elevenes kunnskap om og forståelse av det de har lært, og anvendelse av det de har lært er viktig for å oppnå kompetanse. I utredningen \emph{NOU 2015:8 Fremtidens skole} anses det en tett forbindelse mellom kompetanse og dybdelæring.
\begin{displayquote}
Utvalget mener at mer dybdelæring i skolen vil bidra til at elevene behersker sentrale elementer i fagene bedre og lettere kan overføre læring fra ett fag til et annet. (Ludvigsen-utvalget 2015)
\end{displayquote}
For å fremme dybdelæring forutsetter det varierte arbeidsformer. I utredningen blir lærerenes arbeid knyttet til å gi elever tilstrekkelig tid til fordypning, utfordringer tilpasset den enkelte eleven og elevgruppensnivå, samt støtte og veiledning (\citeNP[s. 11]{ludv15}).
\newline\newline
Ifølge opplæringsloven skal alle elever få en opplæring tilpasset etter deres evner og forutsetninger. Gjennom min egen praksis har jeg undervist naturfag til 10. trinn på en ungdomsskole. Etter praksisperioden har jeg reflektert på min undervisningspraksis. Gjennom denne perioden har jeg innsett nettopp hvor vanskelig det kan være å tilby tilpasset opplæring for en elevgruppe hvor interesse, motivasjon, og faglig nivå varierer. Jeg har fundert over hvordan jeg kan forbedre min undervisningspraksis og bruke mine ideer og anbefalinger fra faglitteratur videre i mitt undervisningsarbeid. Derfor stiller jeg følgende spørsmål til min \mbox{problemstilling:}
\newline
\newline
\textbf{Hvordan kan bruk av representasjonsformer bidra til et variert undervisningsopplegg i en naturfagstime?}
\newline
\newline
Med begrepet representasjonsformer referer jeg til definisjon av \citeA[s. 41]{furb16} og \citeA[s. 61]{knai15}, hvor grafer, figurer, diagrammer, bilder og så videre brukes til å illustrere fagets begreper, fenomener og prosesser. Jeg har valgt å fokusere på dette i forbindelse med et undervisningsopplegg til en naturfagstime som er knyttet til kompetansemålet:
\begin{displayquote}
\emph{undersøke hydrokarboner, alkoholer, karboksylsyrer og karbohydrater, beskrive stoffene og gi eksempler på framstillingsmåter og bruksområder}\newline (\citeNP{udirLP})
\end{displayquote}
Dette er et kompetansemål jeg har selv operasjonalisert gjennom min praksisperiode. Jeg vil undersøke hvordan jeg kan bedre tilpasse opplæringen i en naturfagstime ved hjelp av tekster og illustrasjoner. Først vil jeg trekke inn teorien og undersøke hva litteraturen fremhever, og deretter vil jeg drøfte problemstillingen i lys av teori knyttet til pedagogikk og naturfagdidatikk. 
\end{document}