
\documentclass[main.tex]{subfiles} 
\begin{document}

\setlength{\epigraphwidth}{0.8\textwidth}
\epigraph{``\textelp{} part of the feedback given to pupils in class is like so many 
bottles thrown out to sea. No one can be sure that the message they contain will one day 
find a receiver.''}
{\textit{Philippe Perrenoud}}


\section{Random sitater}

\subsection*{Gruppeoppgaver og muntlig naturfag}
\citeA[s. 56]{froy10} beskriver hvordan forståelse kan nivådeles fra lav nivå 
til høy nivå gjennom følgende fire dimensjoner
\begin{itemize} 
\item å forstå et kunnskapsområde (kunnskap)
\item å forstå metodene som ble brukt for å komme fram til kunnskapen (metode)
\item å forstå meningen med kunnskapen og hva den har å gjøre med deg (hensikt)
\item å forstå hvordan du skal formidle kunnskapen til andre (form)
\end{itemize}
I den siste dimensjonen blir elever vurdert etter hvordan de presenterer sin kunnskap,
i hvilken grad de tar i bruk symbolsystemer og i hvilken de grad vidsr de hensyn til 
den situasjonen foregår i.
\newline\newline
Viktig å være bevisst på hvor mange frihetsgrader elever skal få (\citeNP[s. 29]{knai11}.
\newline\newline
Gode fagsentrerte samtaler mellom elever (eller faglige samtaler med lærer) hvor elever bruker egne erfaringer og språk for å oppnå faglig forståelse hjelper til å skape bro mellom
praksis og teori (\citeNP{odeg10})
\newline\newline
En viktig del av den sosiale utprøvingen av ideer og begreper innebærer å sammenlikne egne forestillinger med andres forestillinger i tillegg til naturvitenskapens forklaringer (\citeNP{odeg10}; \citeNP{dals94}).
\newline\newline
\citeA[s. 176]{klet13} viser til viktigheten av at lærere
legger til rette for ``systematisk trening, øvelse og bruk av naturfaglige begreper for å utvikle
elevenes naturfaglige forståelse''. Muntlige ferdigheter er en av grunneleggende ferdigheter i naturfag. 
I læreplanen står det blant annet: ``Utviklingen av muntlige ferdigheter i naturfag går
fra å kunne lytte og samtale om opplevelser og observasjoner til å kunne presentere og diskutere 
stadig mer komplekse emner''. I den sosiokulturelle tradisjonen rettes fokus mot læring i
felleskap før kunnskap blir internalisert på individnivå (\citeNP[s. 90]{salj13}). Blant annet inkluderer 
dette arbeid i grupper.
\newline\newline
Design av gruppeoppgaven bør utformes slik at
elevene er nødt til å jobbe sammen. Oppgaven bør ikke være så enkel at elevene kan jobbe
individuelt med deloppgavene, slik at det ikke er noen nødvendighet for elevene å jobbe sam-
men. Tilsvarende bør oppgaven ikke ha så høy vanskelighetsgrad slik at de ikke klarer å danne
forståelse eller mening. En gruppeoppgave er da en oppgave som individet ikke klarer å utføre
alene og som krever kollaborasjon. Åpne oppgaver er bedre egnet enn lukkede hvor fokuset er
å finne en riktig svar. Dette er kanskje grunnen til at en sterk elev kan dominere samtalen 
(\citeNP[s. 31]{meli07}).

\subsection*{Naturfag som prosess og produkt}
Naturvitenskapen er både et produkt og en prosess (\citeNP[s.351]{sjob04}). Det vil si på den ene siden er 
naturvitenskapen en produkt, over en lang historisk utvikling, som er satt sammen av begreper, modeller og
teorier som vi idag bruker for å forstå verden rundt oss og prosesser i oss. Ved hjelp av nye oppdagelser og funn utvikler
naturvitenskap videre som et produkt. På den andre siden kjennetegnes naturvitenskapen ved sine prosesser
og metoder. Naturvitenskapen er ikke bare å vite svar, men å søke nye problemstillinger og finne deres svar og
sist men ikke minst skape nye erkjennelser. Det er først og fremst denne nysgjerrigheten vi vil skape og kultivere
blant våre ungdommer. Demokrati argummentet baseres på at unge mennesker skal være aktive og søkende deltagere
i samfunnet. For at dette skal skje er det viktig at undervisningen vektlegger ikke bare den faglige kompetansen
hos eleven men også den sosiale og emosjonelle kompetansen. I utredningen, \emph{NOU 2015:8 Fremtidens skole}, 
anbefales det at sosial og emosjonell kompetanse står sentralt i alle kompetanseområdene (\citeNP[s. 22]{ludv15}).  

\subsection*{Differensiering}

Dersom målet med undervisningen er at alle elever skal forstå det som undervises er det da viktig å treffe
hver enkelt elev som har ulik tilnærming til stoff og gi hver elev mange erfaringer innenfor samme tema 
(\citeNP[s. 32]{froy10}).

\subsection*{Moivasjon og Vurdering}

Indre og ytre motivasjon + vilje
\newline
\newline
\cite[s. 36 - 37]{froy10} lister opp følgende komponenter for å skape indre motivasjon blant elever:
\begin{itemize}
\item konstruerer personlig mening  
\item opplever valgfrihet
\item opplever det utfordrende
\item opplever at de har kontroll
\item samarbeider om oppgaver
\item opplærer at læring har konsekvenser
\end{itemize}
Det er verdt å merke at hvis dette var en prøve så ville det ha vært viktig å skape situasjoner der 
elever kan vise mestring (\citeNP[s. 225]{abhk11}) (bruk 
flere kilder her). 
\newline
\newline
Lav fagelig selvtillit kan i en del tilfeller være et reelt hinder for å velge fysikk, spesielt blant jenter\citeNP[s. 225]{abhk11}.
\newline
\newline
Ved kartlegging av elevenes svakheter og styrker, er det passende å bruke individrelaterte kritier 
(\citeNP[s. 25]{hell07}) og koble inn kompetansemålene. Dette er passende å bruke når det vurderes uten karakter. 


\section*{Introduksjon}
I utredningen, \emph{NOU 2015:8 Fremtidens skole}, vektlegges fagovergripende kompetanser, dvs. 
for eksempel lesing, skriving, utholdenhet, motivasjon, og å kunne planlegge, 
gjennomføre og vurdere egne læringsprosesser (\citeNP[s. 66]{ludv15}) :
\begin{displayquote}
Utvalget anbefaler at fagovergripende kompetanser vektlegges i fremtidens skole. Siden det  
anbefales å integrere dem i fagene, vil informasjon om elevenes kompetanse i fag være viktig.  
Samtidig vil skoler, skoleeiere og nasjonale myndigheter ha behov for informasjon om elevenes utvikling av 
prioriterte fagovergripende kompetanser, for å kunne bidra til at de vektlegges i opplæringen. 
(Ludvigsen-utvalget 2015)
\end{displayquote}
For å vektleggge fagovergripende kompetanser, er da informasjon om elevenes kompetanse i fag viktig. 
Denne informasjonen kan akkumuleres gjennom bruk av formativ vurdering og summativ vurdering.
Gjennom underveisvurderingen, for eksempel, følges elevenes progresjon i faget over tid, og både læreren 
og elev får informasjon om oppnådd kompetanse (\citeNP[s. 204]{olma15}). Hensikten med slik vurdering er å 
gi et grunnlag for å forbedre og videreutvikle kvaliteten på opplæringen (\citeNP[s. 92]{ludv15}).

Jakt etter bevis på læring er helt grunnleggende innenfor området Vurdering for Læring (VfL)
(\citeNP[s. 1]{brbl14}). Ifølge Brevik og Blikstad-Balas, dette ``beviset'' skal kunne brukes aktivt
av læreren og elevene for å avgjøre hvor de er i sin læring, hva de bør jobbe videre med, og
hvordan de kan gå fram for å få det til. \citeA[Vurderingsforskriften]{udirFP} har i tillegg definert fire
prinsipper for hva som utgjør en god vurdering 
\begin{itemize}
\item Forstår hva de skal lære og hva som er forventet av dem.
\item Får tilbakemeldinger som forteller dem om kvaliteten på arbeidet eller prestasjonen.
\item Får råd om hvordan de kan forbedre seg.
\item Er involvert i eget læringsarbeid ved blant annet å vurdere eget arbeid og utvikling.
\end{itemize}

Gjennom min praksis har jeg laget og brukt en kartleggingsprøve i sannsynlighetsregning til å bestemme
elevenes nivå og svakheter. Elevene har fått skriftlige tilbakemeldinger på sine besvarelser.
Etter at elevene har fått tilbake sine besvarelser, har de fått muntlige tilbakemeldinger. Dette utgjør da 
den kvantitative dataen. Gjennom disse tilbakemeldinger har de fått anledning til å reflektere
de skriftlige tilbakemeldingene og fått anledning til å ytre sine meninger og stilt spørsmål. Jeg har også
observert de videre og stilt egne spørsmål, spørsmål som er relevant for deres besvarelser og spørsmål
rundt deres misoppfattelser. Dette utgjør da den kvalitative dataen. Gjennom disse samtalene har jeg dannet 
en profesjonsetisk evaluering av min egen praksis, og stiller derfor følgende spørsmål til min 
\mbox{problemstilling :}
\newline
\newline
\textbf{Hvordan kan kartleggingsprøver brukes i vurderingsarbeidet til å fremme læring i 
        sannsynslighetsregning for 10. trinn?}
\newline
\newline
Med vurderingsarbeidet mener jeg følgende :
\begin{itemize}
\item Kartlegging av elevers og klassens forståelse og nivå
\item Bruk av tilbakemeldinger og fremovermeldinger
\item Konkretisering av mål, både kortsiktig og langsiktig mål
\item Elevdeltagelse i egen vurdering
\end{itemize}
Det siste punktet kan også tolkes under paragraf \S 3-4 i opplæringsloven. Her står det \mbox{følgende :}
\begin{displayquote}
Elevane, lærlingane, praksisbrevkandidatane og lærekandidatane skal vere aktivt med i opplæringa.
(Kapittel 3, lovdata.no)
\end{displayquote}
Dette blir enda stekere vektlagt for elever som har rett til spesial undervisning. I paragraf \S 5-1, står det :
\begin{displayquote}
I vurderinga av kva for opplæringstilbod som skal givast, skal det særleg leggjast vekt på utviklingsutsiktene til eleven.
(Kapittel 5, lovdata.no)
\end{displayquote}
Her kan jeg som lærer bidra til dette gjennom vudering for elevens læring. Sammen med eleven kan vi sette konkrete mål,
og jobbe mot disse målene. Ved å ta dette som utgangspunkt, vil jeg nå redegjøre for hva litteraturen sier.
Jeg vil deretter trekke frem noen besvarelser og tilbakemeldinger. Disse vil jeg drøfte i lys av fagdidaktisk og 
pedagogisk teori, og tilslutt vil jeg se hvordan jeg kan bruke dette videre i mitt undervisningsarbeid.

Den britiske pedagogen Dylan William, skriver at det er ikke 
tilbakemeldingen som er viktig men prosessen som iverksettes som følge av tilbakemeldingen (\citeNP[s. 138]{will10}). 
Dermed søker også jeg etter Williams synspunkt, å iverksette prosessen som hjelper eleven å utvikle seg i faget og bli 
mer selvstendig. For at jeg skal danne et godt  bilde av både mine enkelte elever og hele klassen som helhet, tilbyr 
kartlegginsprøver en god oversikt. Her kan jeg som tilrettelegger endre min praksis for å imøtekomme elevene hvor de er i 
sin læring.

\end{document}
