\documentclass[main.tex]{subfiles} 
\begin{document}

\section*{Teoretisk bakgrunn}

Naturvitenskapen er både et produkt og en prosess (\citeNP[s.351]{sjob04}). Det vil si på den ene siden er naturvitenskapen en produkt, over en lang historisk utvikling, som er satt sammen av begreper, modeller og teorier som vi idag bruker for å forstå verden rundt oss og prosesser i oss. Ved hjelp av nye oppdagelser og funn utvikler naturvitenskap videre som et produkt. På den andre siden kjennetegnes naturvitenskapen ved sine prosesser og metoder. Naturvitenskapen er ikke bare å vite svar, men å søke nye problemstillinger og sist men ikke minst skape nye erkjennelser. Det er først og fremst denne nysgjerrigheten vi vil skape og kultivere hos våre ungdommer slik at de kan være aktive deltagere i samfunnet (\citeNP[s. 153]{mang13}). For at dette skal skje er det viktig at undervisningen vektlegger relevans og fremprovoserer nygsjerrighet blant unge mennesker. Videre er det viktig at undervisning tilrettelegges for å oppfylle hver enkelt elevs behov slik at alle elever i en klasse føler at de er deltagende og aktive gjennom undervisningen.

\subsection*{Differensiering}

En av sentrale styringsrammene for norsk utdanningspolitikk og skolepraksis er prinsippet om tilpasset opplæring. Opplæringsloven slår fast at opplæringen skal tilpasses evnene og forutsetningene til den enkelte elev (\citeNP[s. 128]{tang08}). Opplæringen skal ivareta sentrale verdier som inkludering, variasjon, sammenheng, relevans, verdsetting, medvirkning og erfaringer. Det innebærer at innenfor rammen av ordinære undervisningen, så langt som mulig skal det prøves å tilpasse opplæringen til den enkelte elev. Dette skal operasjonaliseres av undervisere gjennom differensiering (\citeNP[s. 423]{foss14}) og individualisering (\citeNP[s. 129]{tang08}). Undervisningen må defor tilfredsstille alle elevenes tilretteleggingsbehov i klassen, fra elever med vansker i faget til høytpresterende elever. En viktig intensjon for opplæringsreformen, Kunnskapsløftet (LK06), var nemlig å gi bedre tilpasset opplæring og å styrke elevenes grunnleggende ferdigheter (\citeNP[s. 135]{tang08}; \citeNP[s. 427]{foss14}), som inkluderer blant annet å kunne lese.\footnote{Å kunne lese i naturfag er å forstå og bruke naturfaglige begreper, symboler og figurer. Dette innebærer å kunne identifisere, tolke og bruke informasjon fra lærebøker og digitale kilder (\citeNP{udirLP}).}
\newline\newline
En av vanskelighetene som ligger i gjennomføring av differensiert undervisning er de praktiske forholdene som differensiering og tilpasset opplæring skal gjennomføres i. Blant rammefaktorene som er lagt til rette for at tilpasset undervisning lar seg realisere er et læringsmiljø som er godt utstyrt, både med læringsmateriell og muligheter til å samarbeide og danne grupper. I tillegg må læreren ha tilstrekkelig kompetanse til å kunne organisere undervisningen  slik at elever får de faglige utfordringene som er tilpasset deres forutsetninger (\citeNP[s. 161]{engh11}).
\newline\newline
Ifølge \citeA[s. 422]{foss14} kan differensiert undervisning deles i to kategorier: \emph{pedagogisk differensiert undervisning} og \emph{organisatorisk differensiert undervisning}. \underline{Permanent} nivådelt undervisning strider mot det overordnede prinsippet tilpasset opplæring (\citeNP[s. 423]{foss14}):
\begin{displayquote}
\textelp{}. Til vanlig skal organisering ikke skje etter faglig nivå, kjønn eller etnisk tilhørlighet. (Opplæringsloven)
\end{displayquote}
Fosse referer til metastudie (\citeNP{hatt09}) når hun skriver at de flinke elevene kan dra nytte av organisatorisk differensiert undervisning, men det har ikke ønsket effekt for elever som strever med faget (\citeNP[s. 423]{foss14}). 

\subsection*{Motivasjon}

Gjennom min praksis erfaring har jeg hatt elever som har problemer med motivasjon. Dette har ofte resultert i at deres innsats og utbytte i timen har vært ``minimalistisk''. Smith fremhever at først må elevens vilje være til stede før det kan jobbes med elevens motivasjon (\citeNP[s. 26]{smit09}). Dermed må det arbeides med å styrke elevens tro på seg selv før presset blir lagt på det faglige innholdet. Klarer læreren å skape individuell faglig interesse hos eleven 
\begin{displayquote}
\textelp{} vil en stor del av læringsprosessen bli styrt av eleven selv. Lærerens rolle blir da hovedsakelig å stake ut veien for elevens læring, og det er mindre behov for at læreren må passe på at eleven arbeider med sin egen læring. \newline (\citeNP[s. 26]{smit09})
\end{displayquote} 
Motivasjon blir ofte kategorisert som indre og ytre motivasjon (\citeNP[s. 162]{mang13}; \citeNP[s. 26 - 27]{smit09}). Frøyland lister opp følgende komponenter for å skape indre motivasjon blant elever (\citeNP[s. 36 - 37]{froy10}). Ifølge henne dannes ytre motivasjon hos eleven når, eleven:
\begin{itemize}
\item konstruerer personlig mening  
\item opplever valgfrihet
\item opplever det utfordrende
\item opplever at de har kontroll
\item samarbeider om oppgaver
\item opplærer at læring har konsekvenser
\end{itemize}
Forholdet mellom elev og lærer har et vesentlig bidrag til elevenes resultater og skolefaglige interesser (\citeNP[s. 70]{hobo11}). En god relasjon mellom lærer og elev avhenger i hvilken grad elevene føler at de blir forstått og lyttet til. 
Viktig å skape situasjoner der elever kan vise mestring. Lav faglig selvtillit kan i en del tilfeller være et reelt hinder for at elever senere velger realfag i videregående utdanning, spesielt blant jenter\citeNP[s. 225]{abhk11}.

\end{document}