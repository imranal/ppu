\documentclass[main.tex]{subfiles} 
\begin{document}

\section*{Teoretisk bakgrunn}

Naturvitenskapen er både et produkt og en prosess (\citeNP[s.351]{sjob04}). Det vil si på den ene siden er naturvitenskapen en produkt, over en lang historisk utvikling, som er satt sammen av begreper, modeller og teorier som vi idag bruker for å forstå verden rundt oss og prosesser i oss. Ved hjelp av nye oppdagelser og funn utvikler naturvitenskap videre som et produkt. På den andre siden kjennetegnes naturvitenskapen ved sine prosesser og metoder. Naturvitenskapen er ikke bare å vite svar, men å søke nye problemstillinger og finne deres svar og sist men ikke minst skape nye erkjennelser. Det er først og fremst denne nysgjerrigheten vi vil skape og kultivere blant våre ungdommer. Demokrati argummentet baseres på at unge mennesker skal være aktive deltagere i samfunnet. For at dette skal skje er det viktig at undervisningen vektlegger relevans og nygsjerrighet blant unge mennesker. For at alle elever i en klasse føler at de er deltagende og aktive gjennom undervisning er det viktig at undervisning tilrettelegges for å oppfylle elevenes behov.

\subsection*{Differensiering}

En av sentrale styringsrammene for norsk utdanningspolitikk og skolepraksis er prinsippet om tilpasset opplæring.
Opplæringen skal ivareta sentrale verdier som inkludering, variasjon, sammenheng, relevans, verdsetting, medvirkning og 
erfaringer. Dette skal operasjonaliseres av undervisere gjennom differensiering (\citeNP[s. 423]{foss14}).
Undervisningen må, ved hjelp av differensiering, tilfredsstille alle elevenes tilretteleggingsbehov i klassen, fra 
elever med vansker i faget til evnerike elever.

Permanent nivådelt undervisning strider mot det overordnede prinsippet tilpasset opplæring (\citeNP[s. 426]{foss14}). 
Fosse kaller dette for organisatorisk differensiering:
\begin{displayquote}
\textelp{}. Til vanlig skal organisering ikke skje etter faglig nivå, kjønn eller etnisk tilhørlighet. (Opplæringsloven)
\end{displayquote}
Hun referer til metastudie (\citeNP{hatt09}) når hun skriver at de flinke elevene kan dra nytte av organisatorisk
differensiert undervisning, men det har ikke ønsket effekt for elever som strever med faget (\citeNP[s. 423]{foss14}). 

PISA undersøkelsen\footnote{I  PISA-undersøkelsen blir norske 15-åringer sammenliknet med jevnaldrende ungdommer i 
andre OECD-land innen tre sentrale kompetanseområder: matematikk, lesing og naturfag.}. 
Her kan resultatene ha noe å si om hvordan elevenes prestasjoner i ulike fagområder henger sammen med ulike 
bakgrunnsvariabler (\citeNP[s. 11]{klor04}). Begrepet \emph{literacy} er sentralt i PISA og knyttet til til alle de tre
fagområdene matematikk, naturfag og lesing.


I artikelen \citeA[s.431]{foss14}, skriver hun at utstrakt bruk av individuell veiledning kan føre
til manglende inkludering. Siden individualiserte metoder krever stor selvstendighet hos elevene, vil
elever som har vansker med selvregulært læring falle utenfor. 


I paragraf \S 3-4 i opplæringsloven står det \mbox{følgende:}
\begin{displayquote}
Elevane, lærlingane, praksisbrevkandidatane og lærekandidatane skal vere aktivt med i opplæringa.
(Kapittel 3, lovdata.no)
\end{displayquote}

Sett fra det relasjonelle perspektivet i sosiokulturell teori, består det i å veilede elevene i den 
nærmeste utviklingssonen. Den \emph{nærmeste utviklingssonen} beskriver en sone som ligger i mellom en elevs kognitive 
ferdigheter, dvs. hva de kan oppnå selvstendig uten hjelp, og elevens potensielle utvikling, dvs. 
hva en elev kan få til eller forstå gjennom veiledning (\citeNP[s. 125]{bta98}; \citeNP[s. 75]{salj13}). 
Bruk av ''scaffolding`` eller stillasbygging (\citeNP{bta98}) er da viktig for å knytte fagbegreper og 
teori til elevenes forkunnskaper. Gjennom f.eks. elevsamtale kan elev ikke bare få 
tilbakemelding om hvor hen står faglig, men også gi veiledning som kan hjelpe eleven videre i faglig utvikling.
Ifølge \citeA[s.96]{tang10} vil da elevsamtalen være en form for formativ evaluering. \citeA{engh11} utdyper at
\begin{displayquote}
\textelp{} formativ elevvurdering innebærer at læreren veileder det videre arbeidet mot høyere måloppnåelse,
bl.a. med utgangspunkt i elevens  bruk av læringsstrategier.
(\citeNP[s. 162]{engh11}) 
\end{displayquote}
Vurderingsarbeidet vil derfor også gi den forskende lærer (\citeNP[s. 19]{hell07}) verdifull informasjon om sin egen didaktiske tilrettelegging.


\citeA[s. 56]{froy10} beskriver hvordan forståelse kan nivådeles fra lav nivå 
til høy nivå gjennom følgende fire dimensjoner
\begin{itemize} 
\item å forstå et kunnskapsområde (kunnskap)
\item å forstå metodene som ble brukt for å komme fram til kunnskapen (metode)
\item å forstå meningen med kunnskapen og hva den har å gjøre med deg (hensikt)
\item å forstå hvordan du skal formidle kunnskapen til andre (form)
\end{itemize}
I den siste dimensjonen blir elever vurdert etter hvordan de presenterer sin kunnskap,
i hvilken grad de tar i bruk symbolsystemer og i hvilken de grad viser hensyn til 
den situasjonen foregår i.
\end{document}
