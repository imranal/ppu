\documentclass[main.tex]{subfiles} 
\begin{document}

\section*{Teoretisk bakgrunn}

Naturvitenskapen er både et produkt og en prosess (\citeNP[s.351]{sjob04}). Det vil si på den ene siden er naturvitenskapen en produkt, over en lang historisk utvikling, som er satt sammen av begreper, modeller og teorier som vi idag bruker for å forstå verden rundt oss og prosesser i oss. Ved hjelp av nye oppdagelser og funn utvikler naturvitenskap videre som et produkt. På den andre siden kjennetegnes naturvitenskapen ved sine prosesser og metoder. Naturvitenskapen er ikke bare å vite svar, men å søke nye problemstillinger og sist men ikke minst skape nye erkjennelser. Det er først og fremst denne nysgjerrigheten vi vil skape og kultivere hos våre ungdommer slik at de kan være aktive deltagere i samfunnet (\citeNP[s. 153]{mang13}). For at dette skal skje er det viktig at undervisningen vektlegger relevans og fremprovoserer nygsjerrighet blant unge mennesker. Videre er det viktig at undervisning tilrettelegges for å oppfylle hver enkelt elevs behov slik at alle elever i en klasse føler at de er deltagende og aktive gjennom undervisningen.

\subsection*{Differensiering}

En av sentrale styringsrammene for norsk utdanningspolitikk og skolepraksis er prinsippet om tilpasset opplæring. Opplæringsloven slår fast at opplæringen skal tilpasses evnene og forutsetningene til den enkelte elev (\citeNP[s. 128]{tang08}). Opplæringen skal ivareta sentrale verdier som inkludering, variasjon, sammenheng, relevans, verdsetting, medvirkning og erfaringer. Det innebærer at innenfor rammen av ordinære undervisningen, så langt som mulig skal det prøves å tilpasse opplæringen til den enkelte elev. Dette skal operasjonaliseres av undervisere gjennom differensiering (\citeNP[s. 423]{foss14}) og individualisering (\citeNP[s. 129]{tang08}), gjennom blant annet nivå-, mengde- og/eller tempodifferensiering. Undervisningen må defor tilfredsstille alle elevenes tilretteleggingsbehov i klassen, fra elever med vansker i faget til høytpresterende elever. En viktig intensjon for opplæringsreformen, Kunnskapsløftet (LK06), var nemlig å gi bedre tilpasset opplæring og å styrke elevenes grunnleggende ferdigheter (\citeNP[s. 135]{tang08}; \citeNP[s. 427]{foss14}), som inkluderer blant annet å kunne lese.\footnote{Å kunne lese i naturfag er å forstå og bruke naturfaglige begreper, symboler og figurer. Dette innebærer å kunne identifisere, tolke og bruke informasjon fra lærebøker og digitale kilder (\citeNP{udirLP}).}
\newline\newline
En av vanskelighetene som ligger i gjennomføring av differensiert undervisning er de praktiske forholdene som differensiering og tilpasset opplæring skal gjennomføres i. Blant rammefaktorene som er lagt til rette for at tilpasset undervisning lar seg realisere er et læringsmiljø som er godt utstyrt, både med læringsmateriell og muligheter til å samarbeide og danne grupper. I tillegg må læreren ha tilstrekkelig kompetanse til å kunne organisere undervisningen  slik at elever får de faglige utfordringene som er tilpasset deres forutsetninger (\citeNP[s. 161]{engh11}).
\newline\newline
Ifølge \citeA[s. 422]{foss14} kan differensiert undervisning deles i to kategorier: \emph{pedagogisk differensiert undervisning} og \emph{organisatorisk differensiert undervisning}. \underline{Permanent} nivådelt undervisning strider mot det overordnede prinsippet tilpasset opplæring (\citeNP[s. 423]{foss14}):
\begin{displayquote}
\textelp{} Til vanlig skal organisering ikke skje etter faglig nivå, kjønn eller etnisk tilhørlighet. (Opplæringsloven)
\end{displayquote}
Fosse referer til metastudie (\citeNP{hatt09}) når hun skriver at de flinke elevene kan dra nytte av organisatorisk differensiert undervisning, men det har ikke ønsket effekt for elever som strever med faget (\citeNP[s. 423]{foss14}). 


\subsection*{Høytpresterende og evnerike elever}

I opplæringsloven \S 1-3 angis det at opplæringen skal tilpasses til den enkelt elevs ``evner og forutsetninger''.  Elever med stort læringspotensial er også en elevgruppe som må ivaretas i klasserommet. Dette er elever som viser faglig sammenhenger, er reflekterte og har god formuleringsevne. Fra lærerens perspektiv er det ofte ikke nødvendig med motivasjonsarbeid, derimot kan slike elever kjede seg hvis de ikke blir faglig utfordret (\citeNP{brgu16}). Slike elever har nytte av tilbakemeldinger som hjelper de med å jobbe målrettet. Det må eksistere en aksept for å være flink i klasserommet, dette skaper trygghet. Et slikt aksept blir oppfostret gjennom god klassemiljø og gjennom lærers egen oppmerksomhet til slike elever. Slike elever har også gode rutiner med å være klar til undervisningssekvenser, her kan jeg som lærer bruke deres forkunnskaper til å engasjere disse elevene i en dialog. Det ligger innenfor lærerens profesjonsetisk ansvar å ivareta slike elever fra presset med å alltid være best. 
\newline\newline
Jeg kan tilpasse opplæringen gjennom å tilby høytpresterende elever nivå delte oppgaver og oppgaver som er kognitiv utfordrende. I slike tilfeller kan rike åpne oppgaver brukes til å la slike elever ha mulgiheten til å gå dypere i fagstoff og trekke faglig sammenhenger.  Derfor er det viktig å være bevisst på hvor mange frihetsgrader elever skal få (\citeNP[s. 29]{knai11}). Jo flere beslutninger eleven må ta selv, jo åpnere er oppgaven.
\newline\newline 
I denne diskursen har jeg snakket kortfattet om høytpresterende elever. I litteraturen skilles det mellom høytpresterende elever og evnerike elever (\citeNP{brgu16}; \citeNP[s. 208]{kolb14}). Kolberg beskriver de evnerike elevene, fra et snevert definisjon uttrykt ved elever med en IQ på over 130 til en mangfoldig gruppe, hvor elevene har sterke intellektuelle evner, sterke kreative og/eller kunstneriske evner, sterke lederegenskaper, eller som har særlige evner innenfor bestemte fagområder. Også denne elevgruppen har behov for tilpasset opplæring. Manglende tilpasset opplæring kan medføre en negativ holdning til skolen til generelt, og til læring i skoleregi spesielt (\citeNP[s. 208]{kolb14}). Her kan mulige tiltak være bruk av både organisatorisk differensiering og pedagogisk differensiering. Til vanlig skal ikke organisatorisk differensiering brukes som et fast virkemiddel, men forskning viser at slike elever har behov for å tilbringe tid med andre på samme intellektuell nivå (\citeNP[s. 215]{kolb14}). Gjennom pedagogisk differensiert undervisning kan tiltak som tempo brukes til å forsere eleven i pensum, fag og trinn. Et slikt tiltak tillatter at eleven forblir en del av mangfoldet og prinsippet om tilpasset opplæring ivaretas.


\subsection*{Elever med minoritetsspråklig bakgrunn og svaktpresterende elever}
I et studie (\citeNP{hegn13}) analyseres en longitudinell surveyundersøkelse som følger overgangen for Oslo-ungdommer fra ungdomstrinnet til videregående skole. Resultatene av studiet viser til at sammenlignet med ungdom med norskfødte foreldre opplever ungdom med innvandringsbakgrunn en sterkere nedgang i blant annet skoleengasjement, trivsel på skolen og fag selvforståelse, særlig i studiespesialiserende studieretning. Årsak til nedgangen i skoletrivsel skyldes at enkelte påbegynner utdanningen med noe svakere resultater fra grunnskolen, men vel så mye at de opplever en nedgang i vennenettverk og faglig sosial støtte fra lærerene i videregående skole (\citeNP[s. 49]{heng13}). Mange ungdommer opplever å få mindre faglig støtte, faglig interesse, ros og forventninger fra læreren/lærerne sine i videregående skole enn på ungdomsskolen. Manglende opplevd støtte i relasjonen til læreren på videregående er en klar risikofaktor for svakere skoleresultater og redusert motivasjon for elever med innvandringsbakgrunn (\citeNP[s. 70]{heng13}).

Gi foreldre gode råd om hvordan de kan aktiv følge opp skolegangen og dermed bidra til støtten hjemme.
Derimot:
\begin{displayquote}
Den praktiske oppfølgingen av hjemmearbeid og leksehjelp er ikke like sterk når elevene når lengere opp i utdanningsløpet.
\end{displayquote}
Lærerne kan bidra til å arrangere leksehjelp en gang i uken, og oppfordre elever til deltagelse.
\begin{itemize}
\item Stille tydelig krav og forventninger fra eleven
\item Bruke formativ vurdering underveis til å følge opp eleven helt fra starten
\item Gi gode tilbakemeldinger (2 wishes and a star)
\item Viktigst av alt: Danne gode relasjoner med eleven
\end{itemize}
\begin{displayquote}
Lærernes støtte og omsorg var viktig for både gutter og jenter, men mens jentene foretrakk å vende seg til venner og foreldre når de hadde det vanskelig, fortalte flere av guttene at de vendte seg til (mannlige) lærere.
\end{displayquote}

\subsection*{Motivasjon}

Gjennom min praksis erfaring har jeg hatt elever som har problemer med motivasjon. Dette har ofte resultert i at deres innsats og utbytte i timen har vært ``minimalistisk''. Smith fremhever at først må elevens vilje være til stede før det kan jobbes med elevens motivasjon (\citeNP[s. 26]{smit09}). Dermed må det arbeides med å styrke elevens tro på seg selv før presset blir lagt på det faglige innholdet. Klarer læreren å skape individuell faglig interesse hos eleven 
\begin{displayquote}
\textelp{} vil en stor del av læringsprosessen bli styrt av eleven selv. Lærerens rolle blir da hovedsakelig å stake ut veien for elevens læring, og det er mindre behov for at læreren må passe på at eleven arbeider med sin egen læring. \newline (\citeNP[s. 26]{smit09})
\end{displayquote} 
Motivasjon blir ofte kategorisert som indre og ytre motivasjon (\citeNP[s. 162]{mang13}; \citeNP[s. 26 - 27]{smit09}). Frøyland lister opp følgende komponenter for å skape indre motivasjon blant elever (\citeNP[s. 36 - 37]{froy10}). Ifølge henne dannes ytre motivasjon hos eleven når, eleven:
\begin{itemize}
\item konstruerer personlig mening  
\item opplever valgfrihet
\item opplever det utfordrende
\item opplever at de har kontroll
\item samarbeider om oppgaver
\item opplærer at læring har konsekvenser
\end{itemize}
Forholdet mellom elev og lærer har et vesentlig bidrag til elevenes resultater og skolefaglige interesser (\citeNP[s. 70]{hobo11}). En god relasjon mellom lærer og elev avhenger i hvilken grad elevene føler at de blir forstått og lyttet til. 
Det er samt viktig å skape situasjoner der elever kan vise mestring. Lav faglig selvtillit kan i en del tilfeller være et reelt hinder for at elever senere velger realfag i videregående utdanning, spesielt blant jenter (\citeNP[s. 225]{abhk11}).

\end{document}