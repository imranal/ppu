\documentclass[main.tex]{subfiles} 
\begin{document}

\section*{Metode}
\label{sec:2}
Nordahl skriver at utfordringen er ikke at skolen mangler data, men at data ofte i lite grad blir
systematisk analysert og senere aktivt brukt for å forbedre praksisen (\citeNP[s. 9]{hell07}).
I denne oppgaven er dataen innsamling av elevbesvarelser og mine egne skriftlige tilbakemeldinger.
Jeg har valgt å fremheve noen av disse skriftlige tilbakemeldinger for å analysere min egen praksis.
Dataen er også ment til å brukes i læringsrettet kontekst, der elevene kan få
individuelle tilbakemeldinger og fremovermeldinger. Henikten er å få oversikt over elevers 
ferdigheter i sannsynlighetsregning og hjelpe de bli flinkere i sannsynlighetsregning.
\newline
\newline
I denne oppgaven har jeg valgt å også benytte meg av kvalitativ forskning og metode. Ved
kvalitative metoder får en ofte anledning til å gå mere i dybden på materialet og man kommer
tett på subjektene, men derfor er metoden også mere ressurskrevende og man må derfor
begrense antall forsøksobjekter. Forskerne kaller det å "mette" materialet, noe som vil si 
flere intervjuer neppe vil avdekke noe avgjørende nytt (\citeNP{hoff13}). Jeg har hatt
personlige samtaler med 6 av 28 elever som tok kartleggingstesten. Elevene hadde ulike
resultater og derfor også ulike former for tilbakemeldinger og fremovermeldinger.

\section*{Gjennomføring}
\label{sec:3}
Når elever skal vurderes så kan dette gjøres på flere måter:
\begin{itemize}
\item Normalfordeling : da blir vurderingsgrunnlaget \emph{de andre elevenes prestasjoner}. 
Dette blir også referert som relativ vurdering. Vurdering av en individ avhenger da av de andre
elevenes prestasjoner.
\item Fast poengsum : innebærer at det er etter poeng oppnåelse elevene blir vurdert. Da vil
karakterene avgjøre ut ifra hvor mye peongsum elevene klarte å oppnå.
\item Individrelatert kriterier : da vurderes eleven utelukkende i forhold til sine egne forutsetninger
og tidligere prestasjoner. I grunnskolen skal vurderingen uten karakterer i hovedsak være 
individrelatert (\citeNP[s. 25]{hell07}).
\end{itemize}
Jeg er nok enig i at bruken av individrelatert vurdering er en god vurderingsgrunnlag i situasjoner
der karakterer ikke brukes. Denne vurderingsformen oppfyller kriterier for god vurdering,
siden den brukes til å fortelle eleven hvor hen befinner seg i sitt studieløp/progresjon.
Når lærer og elev sammen setter individuelle mål, både nærliggende og langsikte, da vil eleven
gjennom et slikt vurderingsformat få konkete tilbakemeldinger og fremovermeldinger som 
fokuserer på nettopp elevens prestasjon og målsettinger som hen har laget sammen med
læreren.
\newline
\newline
Jeg brukte derimot ikke et slikt vurderingsgrunnlag siden min hensikt var å kartlegge elevenes
svakheter. Da er det passende å bruke en fast poengsum som vurderingskriterie for å 
fortelle elevene hvor de er i sitt utdanningsløp og hva de må jobbe med fremover. 

\end{document}
