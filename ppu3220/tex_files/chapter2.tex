\documentclass[main.tex]{subfiles} 
\begin{document}

\section*{Metode}
\label{sec:2}

Kvalitativ forskning skal blant annet:
\begin{enumerate}
\item Sammenlignes med og forholdes til litteraturen på området: Stemmer konklusjonene noenlunde med det man vet fra før?
\item Bruke teoretisk kunnskap til å analysere dataene: Viser de hva man kunne forvente?
\item Bruke ulike metoder – for eksempel både fokusgrupper, individuelle intervjuer og observasjoner. Når Arne Astrup kritiserer studien av Ny nordisk hverdagsmat for å bare bygge på seks personer, gir det ifølge Bente Halkier et misvisende bilde, fordi intervjuene er supplert av andre metoder.
\item Snakke med tilstrekkelig mange personer til at man har avdekket de viktigste poengene. Forskerne kaller det å «mette» materialet, noe som vil si flere intervjuer neppe vil avdekke noe avgjørende nytt.
\end{enumerate}
Hvis de fire områdene er dekket skikkelig, er kriteriene for god, kvalitativ forskning oppfylt.
\newline
\newline
Kilde: http://forskning.no/sosiologi/2013/09/hva-kan-vi-bruke-kvalitativ-forskning-til

\section*{Gjennomføring}
\label{sec:3}

\end{document}
