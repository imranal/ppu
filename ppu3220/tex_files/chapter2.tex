\documentclass[main.tex]{subfiles} 
\begin{document}

\section*{Metode}
\label{sec:2}
Nordahl skriver at utfordringen er ikke at skolen mangler data, men at data ofte i lite grad blir
systematisk analysert og senere aktivt brukt for å forbedre praksisen (\citeNP[s. 9]{hell07}).
I denne oppgaven er dataen innsamling av elevbesvarelser og mine egne skriftlige tilbakemeldinger.
Jeg har valgt å fremheve noen av disse skriftlige tilbakemeldinger for å analysere min egen praksis.
Dataen er også ment til å brukes i læringsrettet kontekst, der elevene kan få
individuelle tilbakemeldinger og fremovermeldinger. Henikten er å få oversikt over elevers 
ferdigheter i sannsynlighetsregning og hjelpe de bli flinkere i sannsynlighetsregning.
\newline
\newline
I denne oppgaven har jeg valgt å også benytte meg av kvalitativ forskning og metode. Ved
kvalitative metoder får en ofte anledning til å gå mere i dybden på materialet og man kommer
tett på subjektene, men derfor er metoden også mere ressurskrevende og man må derfor
begrense antall forsøksobjekter. Forskerne kaller det å "mette" materialet, noe som vil si 
flere intervjuer neppe vil avdekke noe avgjørende nytt (\citeNP{hoff13}). Jeg har hatt
personlige samtaler med 6 av 28 elever som tok kartleggingstesten. Elevene hadde ulike
resultater og derfor også ulike former for tilbakemeldinger og fremovermeldinger.

\section*{Gjennomføring}
\label{sec:3}

\end{document}
