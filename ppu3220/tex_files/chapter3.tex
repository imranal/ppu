\documentclass[main.tex]{subfiles} 
\begin{document}

\section*{Resultater}
\label{sec:4}

Siden jeg hadde valgt å burke poengsum som vurderingskriterie for elevenes besvarelser var det 
uheldig i denne sammenhengen. Jeg hadde vektlagt de ``vanskelige'' oppgavene mye mer enn de 
``enkle''. Nesten alle elever på tvers av nivå og ferdigheter hadde 
problemmer med å løse disse oppgavene og veldig få klarte å gå over en score på 5 ut av 10. Dermed fikk jeg ikke 
veldig mye informasjon om elevene gjennom poengsum. Oftest var det de faglige sterke i klassen som tangerte
mot 5 i poengsum. Ut ifra dette konkluderte jeg sammen med veileder at prøven var nok litt vanskeligere enn
det burde ha vært. Videre nå vil jeg snakke om individuelle oppgaver fra kartleggingsprøven, og diskutere 
elevenes feiltolninger.

\subsection*{Elevenes feiltolkninger}
Dessuten var det en deloppgave i prøven mange elever feiltolket, og her kan jeg godt
akseptere at det var lett å feiltolke hva oppgaven spør om :
\par
\begin{figure}[h!]
\centering
\includegraphics[scale = 0.7]{../figures/oppgave4b.png}
\caption{Oppgave 4}
\label{fig:oppgave4}
\end{figure}
I oppgave b står det \emph{Hva er sannsynligheten for å vinne på det tredje kastet?}. Her var hensikten at
elevene skulle oppfatte det som \emph{Hva er sannsynligheten for å tape på to runder på rad og deretter vinne på 
det tredje kastet?}, men mange oppfattet det som å vinne på tredje kastet uavhengig av hva som forekommer på de 
første to kast. Dermed ville svaret til neste del av deloppgaven, \emph{Er det mer sannsynlig å få til 17 eller 
høyere på første kast enn på det tredje kast?},  være ``like sannsynlig'' og det var ofte det elevene besvarte. 
Det var heller ikke hjelpsomt når denne oppgaven telte 15\% (dvs. 1.5 poeng) av hele prøven.  

\begin{figure}
\centering
\includegraphics[scale = 0.4]{../figures/maryam.png}
\caption{Oppgave 1}
\label{fig:maryam}
\end{figure}

\begin{figure}
\centering
\includegraphics[scale = 0.4]{../figures/maryam2.png}
\caption{Oppgave 3}
\label{fig:maryam2}
\end{figure}

\begin{figure}
\centering
\includegraphics[scale = 0.4]{../figures/maria.png}
\caption{Oppgave 4}
\label{fig:maria}
\end{figure}

\begin{figure}
\centering
\includegraphics[scale = 0.4]{../figures/mohsin.png}
\caption{Oppgave 5}
\label{fig:mohsin}
\end{figure}

\begin{figure}
\centering
\includegraphics[scale = 0.4]{../figures/mohsin2.png}
\caption{Tilbakemelding og fremovermelding}
\label{fig:mohsin2}
\end{figure}

William redegjør for hvorfor tilbakemeldinger noen ganger kan føre til senking 
i elvenes ytelse. Han referer til Kluger og DeNisi (1996), når han summerer opp 
\begin{displayquote}
\textelp{} feedback was least effective when it focused on the task in hand, 
and more effective when it focused on the details at hand, and most effective 
when it focused on the details of the task and involved goal-setting.
(\citeNP[s. 140]{will10})
\end{displayquote}

\end{document}
