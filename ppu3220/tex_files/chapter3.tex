\documentclass[main.tex]{subfiles} 
\begin{document}

\section*{Resultater}
\label{sec:4}

Før resultatene til kartleggingsprøven hadde jeg fått vite av min veileder at klassensnitt midt mellom
karakter 2 og 3. (utfyll)

Siden jeg hadde valgt å burke poengsum som vurderingskriterie for elevenes besvarelser var det 
uheldig i denne sammenhengen. Jeg hadde vektlagt de ``vanskelige'' oppgavene mye mer enn de 
``enkle''. Nesten alle elever på tvers av nivå og ferdigheter hadde 
problemmer med å løse disse oppgavene og veldig få klarte å gå over en score på 5 ut av 10. Dermed fikk jeg ikke 
veldig mye informasjon om elevene gjennom poengsum. Ofte var det de over middelssterke elevene i klassen som tangerte
mot 5 i poengsum. Kun en elev klarte å oppnå en score på 8.5, hvor den eneste ``feilen'' eleven gjorde var å
feiltolke oppgave 4.b (dette blir diskutert i neste seksjon). Siden eleven demonstrerte så pass sterke ferdigheter 
burde det kanskje ha vært rom for å gi eleven uttelling for oppfattelsen hen hadde dannet om deloppgaven. 
Uansett konkluderte jeg sammen med veileder at prøven var nok litt vanskeligere enn det burde ha vært. Videre nå vil 
jeg snakke om individuelle oppgaver fra kartleggingsprøven, og diskutere elevenes feiltolninger.

\end{document}
