\documentclass[main.tex]{subfiles} 
\begin{document}

\section*{Teoretisk bakgrunn}
Gjennom underveisvurderingen følges elevenes progresjon i faget over tid, og læreren får informasjon
om oppnådd kompetanse.

Det er ikke bare den faglige vurderingen som skal evalureres gjennom en skolegang. 
I utredning \emph{NOU 2015:8 Fremtidens skole} står det blant at at den sosiale og 
emosjonelle utviklingen bør også ha større plass i vurderingsgrunnlaget enn det har idag, 
særskilt som forutsetninger for den kompetansen elevene viser i faget. :
\begin{displayquote}
Utvalget fremhever betydningen av et bredt kompetansebegrep,
og at skolen mer systematisk enn
i dag skal støtte elevenes sosiale og emosjonelle
læring og utvikling i fagene. For eksempel skal
elevene utvikle nysgjerrighet, selvregulering og
respekt for andres synspunkter. Sosiale og emosjonelle
kompetanser er ikke vektlagt systematisk
i dagens læreplaner, og det er derfor en endring
sammenlignet med i dag når dette blir en tydeligere
del av kompetansemålene i fagene. Dette
gir noen utfordringer som må håndteres på en
god måte i bestemmelser for vurdering og i lærernes
praksis. (Ludvigsen-utvalget 2015)
\end{displayquote}

\begin{figure}[h!]
\includegraphics[scale = 0.1]{../figures/vurderingslinjen.png}
%\caption{Oversikt over naturfaglærernes undervisningstilbud til elevene fra PISA+ studie. Kilde: 
%\protect\citeA{odeg10}.}
%\label{fig:odeg10}
\end{figure}

Skriv om : Summativ vs formativ vurdering
\newline
\newline
I læringsrettet vurdering stilles det strengere krav til lærerers evner som evaluator. Da er det viktig å se på lærerens læringssyn. 
Her er to eksempeler på forskjellige læringssyn :
\newline
\newline
\textbf{Læring som overføring av kunnskap}
\newline
I behavioristisk læringsteori foregår læring ved overføring av kunnskap, uavhengig av relasjonen mellom lærer 
og elev. Elev blir anskuet som et tomt kar, som det er lærerens jobb å fylle med kunnsakp. 
I et slikt læringssyn er vurdering i seg selv relativt ukomplisert, siden da gjelder det å 
formulere sine tilbakemeldinger på en så presis og elevtilpasset måte som mulig.
Derimot forventes det da at eleven tar til seg tilbakemeldingene og bruker dem til å rette seg etter.
Tilbakemeldingene vil da være begrenset til spesifikke svakheter relatert til faget eller kompetansemål.
Sentralt i behaviorismens syn på læring er betinging \citeNP[s. 74]{salj13}. Ved ønsket atferd belønnes 
handlingen. Dette referes som forsterking \citeNP[s. 22]{hell07}. Innenfor vurderingskontekten er
tilsiktet hensikt å motivere elevene. Dermed er karakteren en forsterker for noen og straff for andre.
\newline
\newline
\textbf{Læring som en relasjonell prosess}
\newline
Sett fra det relasjonelle perspektivet består det i å veilede elevene i den nærmeste utviklingssonen.
Den \emph{nærmeste utviklingssonen} beskriver en sone som ligger i mellom en elevs kognitive 
ferdigheter, dvs. hva de kan oppnå selvstendig uten hjelp, og elevens potensielle utvikling, dvs. 
hva en elev kan få til eller forstå gjennom veiledning (\citeNP[s. 125]{bta98}; \citeNP[s. 75]{salj13}). 
Bruk av ``scaffolding'' eller stillasbygging (\citeNP{bta98}) er da viktig for å knytte fagbegreper og teori til elevenes 
forkunnskaper. Vurderingsarbeidet vil derfor også gi den forskende lærer \citeA[s. 19]{hell07} verdifull informasjon om sin egen didaktiske tilrettelegging.
Jeg vil komme tilbake til disse læringssyn når jeg evaluerer min egen praksis gjennom FoU arbeidet.
\newline
\newline
En av sentrale styringsrammene for norske utdanningspolitikk og skolepraksis er prinsippet om tilpasset opplæring.
Opplæringen skal ivareta sentrale verdier som inkludering, variasjon, sammenheng, relevans, verdsetting, medvirkning og erfaringer.
Dette skal operasjonaliseres av undervisere gjennom differensiering.  Undervisningen må, ved hjelp av differensiering, 
tilfredsstille alle elevenes tilretteleggingsbehov i klassen, fra elever med matematikkvansker (dyskalkuli) til evnerike elever. Når en lærer jobber 
med elever hvor spennet er så pass stort, med andre ord at elevene utgjør en heterogen gruppe, da kan heller ikke undervisningen 
være homogenisert. Jeg kan derfor allerede nå påstå at en behavioristisk tilnærming til vurdering for læring vil tydeligvis ikke oppfylle prinsippet om tilpasset opplæring.
\newline
\newline
Før vi forsetter videre er det lurt å bemerke noe termer jegg vil bruke videre i denne oppgaven.
Vurdering av læring
Vurdering for læring

\end{document}