
\documentclass[main.tex]{subfiles} 
\begin{document}

\setlength{\epigraphwidth}{0.8\textwidth}
\epigraph{``How is error possible in mathematics? 
A sane mind should not be guilty of a logical fallacy, 
and yet there are very fine minds who do not trip in brief reasoning 
such as occurs in the ordinary doings of life, 
and who are incapable of following or repeating without error 
the mathematical demonstrations which are longer, 
but which after all are only an accumulation of brief reasonings 
wholly analogous to those they make so easily. 
Need we add that mathematicians themselves are not infallible?''}
{\textit{Jules Henri Poincaré}}

\section*{Problemstilling}
Sannsynlighetsregning vil bli gjennomgått i løpet av en uke. Elevenes kunnskap vil utredes ved
slutten av uken ved hjelp av en kartleggingsprøve. Deretter vil de få skriftlige tilbakemeldinger
og fremovermeldinger som de kan bruke til å jobbe videre og forberede seg til en skriftlig heldagsprøve. 
Det vil også gis muntlig tilbakemeldinger underveis til elever mens de forbereder seg til heldagsprøven. 
Etter at prøven har blitt avlagt, vil elevenes (mangel av) bruk av tilbakemeldinger bli kartlagt. 

\end{document}
