
\documentclass[main.tex]{subfiles} 
\begin{document}

\setlength{\epigraphwidth}{0.8\textwidth}
\epigraph{``\textelp{} part of the feedback given to pupils in class is like so many 
bottles thrown out to sea. No one can be sure that the message they contain will one day 
find a receiver.''}
{\textit{Philippe Perrenoud}}

\section*{Introduksjon}
Jakt etter bevis på læring er helt grunnleggende innenfor området Vurdering for Læring (VfL)
(\citeNP{brbl14}). Brevik skriver videre at dette ``beviset'' skal kunne brukes aktivt
av læreren og elevene for å avgjøre hvor de er i sin læring, hva de bør jobbe videre med, og
hvordan de kan gå fram for å få det til. \citeA[Vurderingsforskriften]{udirFP} har i tillegg definert fire
prinsipper for hva som utgjør en god vurdering. Ut fra disse prinsipper vil jeg se kritisk på min
egen praksis og evaluere om jeg har klart å følge prinsippene for god vurdering. Sammen med
mine observasjoner fra mine samtaler med elever og min praksisveileder, vil jeg reflektere i lys
av pedagogikk og fagdidaktisk teori over hvordan jeg kan videre utforme kartleggingsprøver og bruke de til å veilede
elever. \citeA{will10} skriver at det er ikke 
tilbakemeldingen som er viktig men prosessen som iverksettes som følge av tilbakemeldingen. Dermed søker
også jeg etter Williams synspunkt, å iverksette prosessen som hjelper eleven å utvikle seg i faget 
og bli mer selvstendig. For at jeg skal danne et godt bilde av både mine enkelte elever og hele klassen som
helhet, tilbyr kartlegginsprøver en god oversikt. Her kan jeg som tilrettelegger endre min praksis for å
imøtekomme elevene hvor de er i sin læring.
\newline

Gjennom min praksis har jeg laget og brukt en kartleggingsprøve i sannsynlighetsregning til å bestemme
elevenes nivå og svakheter. Elevene har fått skriftlige tilbakemeldinger på sine besvarelser.
Etter at elevene har fått tilbake sine besvarelser, har de fått muntlige tilbakemeldinger. Dette utgjør da 
den kvanitative dataen. Gjennom disse tilbakemeldinger har de fått anledning til å reflektere
de skriftlige tilbakemeldingene og fått anledning til å ytre sine meninger og stilt spørsmål. Jeg har også
observert de videre og stilt egne spørsmål, spørsmål som er relevant for deres besvarelser og spørsmål
rundt deres misoppfattelser. Dette utgjør da den kvalitative dataen. Gjennom disse samtalene har jeg dannet 
en profesjonsetisk evaluering av min egen praksis, og stiller derfor følgende spørsmål til min \mbox{problemstilling :}
\newline
\newline
\textbf{Hvordan kan kartleggingsprøver brukes i vurderingsarbeidet til å fremme læring i 
        sannsynslighetsregning for 10. trinn?}
\newline
\newline
Med vurderingsarbeidet mener jeg følgende :
\begin{itemize}
\item Kartlegging av elevers og klassens forståelse og nivå
\item Bruk av tilbakemeldinger og fremovermeldinger
\item Bruk av formativ vurdering
\item Konkretisering av mål, både kortsiktig og langsiktig mål
\item Elevdeltagelse i egen vurdering
\end{itemize}

Det siste punktet kan også tolkes under paragraf \S 3-4 i opplæringsloven. Her står det \mbox{følgende :}
\begin{displayquote}
Elevane, lærlingane, praksisbrevkandidatane og lærekandidatane skal vere aktivt med i opplæringa.
(Kapittel 3, lovdata.no)
\end{displayquote}
Dette blir enda stekere vektlagt for elever som har rett til spesial undervisning. I paragraf \S 5-1, står det :
\begin{displayquote}
I vurderinga av kva for opplæringstilbod som skal givast, skal det særleg leggjast vekt på utviklingsutsiktene til eleven.
(Kapittel 5, lovdata.no)
\end{displayquote}
Her kan jeg som lærer bidra til dette gjennom vudering for elevens læring. Sammen med eleven kan vi sette konkrete mål,
og jobbe mot disse målene. Ved å ta dette som utgangspunkt, vil jeg nå redegjøre for hva litteraturen sier.
Jeg vil deretter trekke frem noen besvarelser og tilbakemeldinger. Disse vil jeg drøfte i lys av fagdidaktisk og 
pedagogisk teori.
\end{document}
