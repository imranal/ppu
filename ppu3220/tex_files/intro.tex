
\documentclass[main.tex]{subfiles} 
\begin{document}

\setlength{\epigraphwidth}{0.8\textwidth}
\epigraph{``How is error possible in mathematics? 
A sane mind should not be guilty of a logical fallacy, 
and yet there are very fine minds who do not trip in brief reasoning 
such as occurs in the ordinary doings of life, 
and who are incapable of following or repeating without error 
the mathematical demonstrations which are longer, 
but which after all are only an accumulation of brief reasonings 
wholly analogous to those they make so easily. 
Need we add that mathematicians themselves are not infallible?''}
{\textit{Jules Henri Poincaré}}

\section*{Introduksjon}
Jakt etter bevis på læring er helt grunnleggende innenfor området Vurdering for Læring (VfL)
(\citeNP{brbl14}). Brevik skriver videre at dette ``beviset'' skal kunne brukes aktivt
av læreren og elevene for å avgjøre hvor de er i sin læring, hva de bør jobbe videre med, og
hvordan de kan gå fram for å få det til. \citeA[Vurderingsforskriften]{udirFP} har i tillegg definert fire
prinsipper for hva som utgjør en god vurdering. Ut fra disse prinsipper vil jeg se kritisk på min
egen praksis og evaluere om jeg har klart å følge prinsippene for god vurdering. Sammen med
mine observasjoner fra mine samtaler med elever og min praksisveileder, vil jeg reflektere i lys
av pedagogikk og fagdidaktikk over hvordan jeg kan videre utforme skriftlige tilbakemeldinger,
slik at elever kan oppnå best mulig læringsutbytte. Det vil derfor være naturlig å stille spørsmål :
hva utgjør en god tilbakemelding, og er det tilbakemeldingen som er viktig eller prosessen som
iverksettes som følge av tilbakemeldingen (\citeNP{will10})?  
\newline
\newline
Gjennom min praksis har jeg laget og brukt en kartleggingsprøve i sannsynlighetsregning til å bestemme
elevenes nivå og svakheter. Elevene har fått skriftlige tilbakemeldinger på sine besvarelser.
Dette utgjør den kvanitative dataen. Etter at elevene har fått tilbake sine besvarelser, har de fått
muntlige tilbakemeldinger. Gjennom disse tilbakemeldinger har de fått anledning til å reflektere
de skriftlige tilbakemeldingene og fått anledning til å ytre sine meninger og stilt spørsmål. Dette utgjør
da den kvalitative dataen. Gjennom disse samtalene har jeg dannet en profesjonsetisk
evaluering av min egen praksis, og stiller derfor følgende spørsmål til min \mbox{problemstilling :}
\newline
\newline
\textbf{Hvordan kan kartleggingsprøver brukes i vurderingsarbeidet til å fremme læring i 
        sannsynslighetsregning for 10. trinn?}



\end{document}
