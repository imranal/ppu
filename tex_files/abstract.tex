\documentclass[main.tex]{subfiles} 

\begin{document}

\section*{Problemstilling}    
I alle studier som diskuterer hva som kjennetegner \emph{god undervisning}, knyttes 
dette ofte til tre dimensjoner\cite[Klette 2013, side 142]{klette}; som den 
engelskspråkelige litteraturen kaller 
\begin{itemize}
\item emotional support (emosjonell støtte),
\item organisational support (organisatorisk støtte),
\item instructional support (undervisningsmessig støtte).
\end{itemize}
I korte trekke sammenfatter emosjonell støtte klasseromsarbeidet som knytter
seg til de sosialse og emosjonelle rammene for læringsarbeidet \cite[Klette 2013, side 143]{klette},
organisatorisk støtte viser til fysisk organisering og klasseledelse \cite[Klette 2013, side189]{klette}
og undervisningsmessig støtte retter fokus mot lærerens sentrale rolle i elevenes kunnskapstilegnelse; ''det 
læreren gjør av faktisk undervisningshandlinger i klasserommet som bidrar til læring'' 
\cite[Klette 2013, side 143 og 146]{klette}.
\newline

Undervisningsmessig støtte kan igjen fordeles blant 4 dimensjoner\cite[Klette 2013, side 146]{klette}
\begin{enumerate}
\item klare læringsmål og intensjoner
\item relevante kognitive utfordringer
\item kvaliteten på klassesamtalene
\item støttende klima.
\end{enumerate}

Ifølge Ungdata\cite[legg til referanse ungdata]{ungdata}\cite[Klette, side 144]{klette} har Norge på nasjonal 
plan siden 1992 hatt et løft når det gjelder elev-lærer 
relasjon, og det psyko-sosiale miljøet på skolen har merkant 
forbedret seg. Det er fære elever som melder at de gruer seg til å 
gå på skolen og fære skulker. Generelt har trivsel 
blant elever økt, og det er etablert et godt læringsmiljø. Derimot er det 
forsatt rom for forbedring når det gjelder hvor effektivt elever tar imot 
instruksjoner og hvorvidt de blir kognitiv utfordret. 
\newline

Problemstillingen jeg har valgt i denne oppgaven fokuserer  på tiltak 
lærere kan ta for å formidle klare instrukser i forbindelse med utføring av en 
naturfag time med utforskende arbeidsmåter\cite[legg til referanse 1 og 2]{utforsk}. 
\newline

Under observasjoner og 
utføring av undervisningssekvens har fokus vært på hvordan 
lærere kan bli flinkere til å delegere oppgaver og formidle 
informasjon. Hvis instrukser ikke er tydelige nok, vil elevene bruke unødvendig 
lengere tid til å komme i gang med undervisningsaktiviteten. Det 
er grunn til å tro at effektiv formidling av instrukser kan i 
helhet spare tid som igjen kan brukes i andre klasseaktiviteter. Den mest selvsagte måte
å rette på dette er at lærer krever at ingen praktiske
spørsmål kan stilles etter at instrukser har blitt formidlet. Da gjenstår
det kun rom for faglige spørsmål. Dette kan derimot kvele engasjement og er
rett og slett ikke et godt nok løsning. Det er derimot viktigere at læreren
gir gode instrukser og forsatt tillater rom for spørsmål rundt instruksene.
Dermed faller denne oppgaven til læreren som må tydeligere
etablere lederrollen og foreta tiltak for å formidle instrukser 
effektivt.

\subsection*{Undervisningsopplegget}

\subsection*{Klassen}
\newpage\null

\end{document}
