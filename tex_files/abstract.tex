\documentclass[main.tex]{subfiles} 

\begin{document}

\section*{Problemstilling}      
Norge som nasjon har siden 1992 hatt et l\o ft i forhold til elev-l\ae rer 
relasjon, og det psyko-sosiale milj\o et p\aa\hspace{1mm} skolen har merkant 
forbedret seg. Det er f\ae re elever som melder at de gruer seg \aa\hspace{1mm} 
g\aa\hspace{1mm} p\aa\hspace{1mm} skolen og f\ae re skulker. Generelt har trivsel 
blant elever \o kt, og det er etablert et godt l\ae ringsmilj\o . Derimot er det 
forsatt rom for forbedring i forhold til hvor effektivt elever assimilerer 
instruksjoner og hvorvidt de blir kognitiv utfordret. Denne oppgaven fokuserer 
p\aa\hspace{1mm} den f\o rst nevnte problemstillingen. Under observasjoner og 
utf\o ring av undervisningssekvens har fokus v\ae rt p\aa\hspace{1mm} hvordan 
l\ae rere kan bli flinkere til \aa\hspace{1mm} delegere oppgaver og formidle 
informasjon. Hvis instrukser ikke er tydelige nok, vil elevene bruke un\o dvendig 
lengere tid til \aa\hspace{1mm} komme i gang med undervisningsaktiviteten. Det 
er grunn til \aa\hspace{1mm} tro at effektiv formidling av instrukser kan i 
helhet spare tid som igjen kan brukes i andre klasseaktiviteter. En m\aa te
\aa\hspace{1mm} rette p\aa\hspace{1mm} dette er at l\ae rer krever at ingen
sp\o rsm\aa l kan stilles etter at instrukser har blitt formidlet. Da gjenst\aa r
det kun rom for faglige sp\o rsm\aa l. Dette kan derimot kvele engasjement og er
rett og slett ikke et godt nok l\o sning. Det er derimot viktigere at l\ae reren
gir gode instrukser og forsatt tillater rom for sp\o rsm\aa l rundt instruksene.
Dermed faller denne oppgaven til l\ae reren som m\aa\hspace{1mm} tydeligere
etablere lederrollen.
\newline
\newline
\hspace{-7mm}\textbf{N\o kkelord}: undervisningsst\o tte : 4-dimensjoner , 
instructional clarity,       

\newpage\null

\end{document}
