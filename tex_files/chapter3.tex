\documentclass[main.tex]{subfiles} 
\begin{document}

\section*{Refleksjon}
\label{sec:3}

Ifølge \citeA{ludv15} vil læringsprosesser som fører til forståelse bidra til å styrke elevenes 
motivasjon og opplevelse av mestring og relevans i skolehverdagen. Men, var dette tilfellet for 8. 
klassen og hvordan kunne undervisningsopplegget forbedres?
\newline
\newline
\citeA[s. 162]{mang13} innleder motivasjon som en trengsel for å ha lyst på noe eller ønske om å 
utføre en aktivitet. Han avslutter med følgende sitat
\begin{displayquote}
\guillemotleft Motivasjon for å læra inneber noko meir enn lyst til å læra. Det handler om den 
mentale innsatsen til eleven. Å lese ein tekst ti gonger kan indikera at eleven held ut, men  
læringsmotivasjon viser seg mellom anna gjennom meir aktive studiestrategiar, slik som 
oppsummeringar, refleksjon over dei grunnleggjande ideane i faget og sammenfattingar av ideane med 
eigne ord.\guillemotright 
\end{displayquote}
Hos Vygotsky (\citeNP[s. 130]{bta98}), motivasjon ligger i å skape meningsfulle læringsbetingelser 
både ved å tilrettelegge undervisningen som passer elevens aktuelle og potensielle nivå, dvs. 
de ytre rammene til den approksimale sonen, og ved å tydeliggjøre nytteverdien av det gitte 
lærestoffet. 
\newline
\newline
\citeA[s. ~136]{klet13} beskriver en god undervisningsseksens der lærere klarer å balansere mellom 
tilegnelses-, utprøvings-, og konsolideringssituasjoner. Ifølge Klette har norske klasserom ensidige 
tendenser i bruken av variert arbeidsmåter. Slik det kan ses fra figur \ref{fig:odeg10}, er det for 
eksempel lite konsolideringssituasjoner. Lærernes metalæringsaktiviteter regnes som særlig 
avgjørende for å sikre elevenes læring (\citeNP[s. 186]{klet13}). Derimot å bruke dette som et fast
organiserende prinsipp, blir sjelden gjennomført (\citeNP[s. 26]{odeg10}). Gjennom timen har 
aktivering av forkunnskaper, gjennom repitisjon og gjenbruk av begreper og gjennomgang av 
lekser, bæret preg av konsolideringssituasjoner/metalæringsaktiviteter. Det var ingen 
appetittvekkere, og dette er noe som burde ha blitt inkludert.
\newline
\newline
Gjennom helklassesamtalene ble elevene spurt om det de har hatt til lekse.
Siden de blir engasjert i samtaler rundt lekser de skal ha utført, har de forutsetning for å kunne 
respondere til lærer initiativ. Det er ønskelig å få bekreftet at elevene innehar en overordnet 
forståelse. Det kan derfor være nødvendig å utpeke noen elever som ikke viser aktiv deltagelse i 
timen og frembringe deres respons. Hvis elevene ikke klarer å respondere på lærer initiativ, kan 
utspørringen av elevene vise hull i deres kunnskap. Derimot har utpeking av elever også noen 
negative implikasjoner. For eksempel vil noen elever føle ubehag av å bli utpekt. Det er ønskelig å 
trene elevene i å aktive delta i undervisningen, men det er også lurt å ikke forsterke negativ 
assosiasjoner til slik deltagelse. Hvis svake elever blir engasjert, bør de få muligheten til å 
kunne demonstrere sin mestring om temaer de er fortrolig og godtkjent med. Det finnes også andre 
metoder for å redegjøre om elevene har gjort sine lekser. Dette kan være at enkelt elever blir 
inspisert. Dermed vil læreren være klar over hvilke elever som ikke har gjort sine lekser, og da er 
det ikke nødvendig å initiere disse elevene til helklassesamtalen.
\newline
\newline
Bruken av revoicing, se \citeA[s. ~175]{klet13}, til å gjenta og forsterke elevenes forslag og 
begrepsbruk ble ikke brukt tilstrekkelig gjennom den første timen. For å kunne bruke revoicing mest 
mulig effektivt, må læreren raskt og effektivt bestemme om elevens repons har validitet og om det er 
relevant. Gjennom prasiserfaringen har revoicing vært vanskelig å utføre og krever veldig god grep 
på det \citeA{batp08} kaller Content Specific Knowledge, CSK. Ifølge Klette, viser fravær av slike 
eksplisitte innramminger fra lærerens side at eleven blir sittende med et uklart kunnskapsinnhold og 
i verste fall feil begrepsforståelse, \citeA[s. ~175-176]{klet13}. 
\newline
\newline
Øvelsen med tokolonnnenotatet (se vedlegg \ref{sec:tokolonnenotat}) hadde flere styrker, men den 
hadde flere organisatoriske svakheter. Det ble brukt for mye tid til å fordele elever i grupper, 
dette kunne gjerne ha blitt planlagt på forhånd. Dessuten var instruksjonene ikke helt klare, 
tydelighet i instruksjoner ville ha spart tid som kunne da brukes av elever i faglig aktivitet. 
Ifølge \citeA[s. ~189]{klet13}, faktorer som har direkte effekt på elevenes læring, fremheves an en 
gjennomtenkt undervisningsopplegg som muliggjør at de bruker minimalt tid på ikke-faglige 
aktiviteter. For tokolonnenotatet og mikroskopøvelsen er det også viktig å være klar over hvor mange 
frihetsgrader elever skal få (\citeNP{knai11}). Jo flere beslutninger eleven må ta selv, jo åpnere 
er oppgaven. Den først-nevnte øvelsen hadde hensikt å skape dypere forståelse av faglig begreper, 
mens den sist-nevnte øvelsen hadde til hensikt å gi erfaring og innsikt i utforskende arbeidsmåter 
som prosess og motivere elevene. Begge øvelsene var delvis lærerstyrt, men hadde stor grad av 
åpenhet rundt resultatene/produktet og kunnskapsutbytte.
\newline
\newline
Når naturfag rettferdiggjøres som et fag i skolen bruker man ofte to typer argumenter, som blir
omtalt som produkt-argumentet og prosess argumentet, \citeA[s. ~351]{sjob04}. Produkt-argumentet går 
ut på at naturfaglige kunnskaper, begreper og teorier er viktige både for eleven i skolehverdagen og 
senere i arbeidslivet. Prosess-argumentet går ut på at det er naturvitenskapens prosesser, 
arbeidsmåter og metoder som rettferdiggjør fagets plass i skolen. \iffalse \citeauthor{sjob04} \else 
Sjøberg \fi skriver at selv om det er noe \emph{pedagogisk tidsmessig og tiltrekkende} ved det synet 
at det er prosessene som er det vesentlige, må det understrekes at produktorientert syn trenger ikke 
å medføre \emph{en autoritær og doserende metodisk tilnærming} når dette produktet skal formidles 
til elevene. \iffalse \citeauthor{sjob04} \else Han \fi skriver videre at det er viktig at 
vitenskapens egenart ikke automatisk dikterer en metodisk tilnærming, eller motsatt, at man lar et 
syn på læring definere hva som skal oppfattes som vitenskapens egenart. Undervisningsopplegget har 
hatt en preg av begge disse syn på vitenskapens vesen. Innføring av nye begreper har styrket 
elevenes syn på naturfag som et produkt, mens deres observasjoner i laboratoriet og skriving av 
rapport har forsterket deres syn på naturfag som en prosess. 
\newline
\newline
En overordnet ramme for arbeid med Forskerspiren er at elevene skal praktisere en vitenskapelig 
metode. På 1960-tallet i USA og England kom læreplaner som blir omtalt
for \emph{discover-learning}, \citeA[s. ~31]{knai11}. Her skulle elevene lære naturvitenskapelig 
kunnskap gjennom aktiviteter som skulle ligne naturvitenskapelig forskning. I følge \iffalse 
\citeauthor{knai11} \else Knain \fi  er det flere svakheter ved denne retningen. En av dem var 
tanken at barn lærer naturfaglig begrepskunnskap gjennom induksjon, det vil si ved å trekke 
sluttninger fra erfaringer. \iffalse \citeauthor{knai11} \else Knain \fi skriver videre at
\begin{displayquote}
\guillemotleft Som Hodson påpeker:
\begin{displayquote}
Du kan ikke oppdage noe som du mangler begreper om. Du vet ikke hvor du skal se, hvordan du skal se 
eller hvordan du skal gjenkjenne det når du har funnet det (Hodson 1996, s. 118). \guillemotright
\end{displayquote}
\end{displayquote}
\citeA[s. ~77]{solv92} skriver at forståelse er aktivert kunnskap. Det vil si hver gang vi utsettes 
for en utfordring blir vårt eget \emph{kunnskapsreservoar} tappet. Dermed aktiverer vi kunnskap. 
Elevenes kunnskaper utgjør en av forutsetningene for de nye kunnskapene vi tilfører dem. Disse 
kunnskapene, sammen med elevenes erfaringer, utgjør det eleven kan møte nye utfordringer med. Dette 
betegnes også som kognitiv struktur av \citeauthor{solv92} og kan deles opp i delstrukturer. Piaget 
kaller slike delstrukturer for skjemaer \citeA[s. ~78]{solv92}. En elev har for eksempel ett skjema 
for celler og ett for organsystemmer. Det som er karakterisktisk for slike skjemaer er at de kan 
operere sammen. Hvis eleven behersker begrepene celler og organsystemmer, kan eleven danne koblingen
mellom disse skjemaene og dermed danne andre assosiasjoner til dyr. På denne måten konstruerer 
eleven ny kunnskap ved hjelp av den kunnskap hun har. Hver elev vil ha sine skjemaer til å møte 
undervisning med.

Utforskende samtaler må først innlæres i en klasse slik at elevene kan få mest mulig
utbytte av sine felles diskusjoner og samtaler. \citeA[s. 57]{meli07} beskriver dette som kjernen i 
praksisen:
\begin{displayquote}
\guillemotleft
At the heart of the approach is the negotiation by each teacher and class of a set of 'ground
rules' for talking and working together. These ground rules then become established as a set of 
principles for how the children will collaborate in groups.
\guillemotright
\end{displayquote}
Slike regler bør derfor etableres ved et tidlig stadie for en gitt klasse. Elevene bør rutineres
i å tillate rom for alternative løsninger, uten å true gruppens solidaritet eller individets 
identitet. 

Valg av grupper med forskjellige permutasjoner er noe som bør utprøves. For eksempel elever
som viser tydelige leder egenskaper bør jevnt fordeles i forskjellige grupper. Dette kan bidra til
å skape tydelige roller i grupper. Det som er viktig med denne tankegangen er at alle elever
i en gruppe bør føle at de har en unik rolle i gruppen, og at sammen kan de utforme et felles
produkt.

Te viktige ting : Ikke tydelige nok instrukser for den første del av tokolonneøvelsen. 
Det kan ha ført noen elever til å samarbeide isteden for å kollaborere.
Mangel av en ferdig produkt som kan leveres på OneNote. Konsolidering av 
hele øvelsen. Ferdig produkt ville vise felles forståelsen elevene hadde dannet i egne
tekster.

\end{document}