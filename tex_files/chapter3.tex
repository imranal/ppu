\documentclass[main.tex]{subfiles} 
\begin{document}

\section*{Refleksjon}
\label{sec:3}

Ifølge \citeA{ludv15} vil læringsprosesser som fører til forståelse bidra til å styrke elevenes 
motivasjon og opplevelse av mestring og relevans i skolehverdagen. Men, var dette tilfellet for 8. 
klassen og hvordan kunne undervisningsopplegget forbedres?
\newline
\newline
\citeA[s. 162]{mang13} innleder motivasjon som en trengsel for å ha lyst på noe eller ønske om å 
utføre en aktivitet. Han avslutter med følgende sitat
\begin{displayquote}
Motivasjon for å læra inneber noko meir enn lyst til å læra. Det handler om den 
mentale innsatsen til eleven. Å lese ein tekst ti gonger kan indikera at eleven held ut, men  
læringsmotivasjon viser seg mellom anna gjennom meir aktive studiestrategiar, slik som 
oppsummeringar, refleksjon over dei grunnleggjande ideane i faget og sammenfattingar av ideane med 
eigne ord.
\end{displayquote}
Hos Vygotsky (\citeNP[s. 130]{bta98}), motivasjon ligger i å skape meningsfulle læringsbetingelser. 
Dette kan oppnås ved å tilrettelegge undervisningen som passer elevens aktuelle og potensielle nivå, 
dvs. de ytre rammene til den approksimale sonen, og ved å tydeliggjøre nytteverdien av det gitte 
lærestoffet.
\newline
\newline
\citeA[s. ~136]{klet13} beskriver en god undervisningsseksens der lærere klarer å balansere mellom 
tilegnelses-, utprøvings-, og konsolideringssituasjoner. Ifølge Klette har norske klasserom ensidige 
tendenser i bruken av variert arbeidsmåter. Slik det kan ses fra figur \ref{fig:odeg10}, er det for 
eksempel lite konsolideringssituasjoner. Lærernes metalæringsaktiviteter regnes som særlig 
avgjørende for å sikre elevenes læring (\citeNP[s. 186]{klet13}). Derimot å bruke dette som et fast
organiserende prinsipp, blir sjelden gjennomført (\citeNP[s. 26]{odeg10}). Gjennom timen har 
aktivering av forkunnskaper, gjennom repitisjon og gjenbruk av begreper og gjennomgang av 
lekser, bæret preg av konsolideringssituasjoner/metalæringsaktiviteter. Det var ingen 
appetittvekkere, og dette er noe som kunne ha blitt inkludert.
\newline
\newline
I helklassesamtalene ble elevene spurt om det de har hatt til lekse.
Siden de blir engasjert i samtaler rundt lekser de skal ha utført, har de forutsetning for å kunne 
respondere til lærer initiativ. Det er ønskelig å få bekreftet at elevene innehar en overordnet 
forståelse. Det kan derfor være nødvendig å utpeke noen elever som ikke viser aktiv deltagelse i 
timen og frembringe deres respons. Hvis elevene ikke klarer å respondere på lærer initiativ, kan 
utspørringen av elevene vise hull i deres kunnskap. Derimot har utpeking av elever også noen 
negative implikasjoner. For eksempel vil noen elever føle ubehag av å bli utpekt. Det er ønskelig å 
trene elevene i å aktive delta i undervisningen, men det er også lurt å ikke forsterke negativ 
assosiasjoner til slik deltagelse. Hvis svake elever blir engasjert, bør de få muligheten til å 
kunne demonstrere sin mestring om temaer de er fortrolig og godtkjent med. Det finnes også andre 
metoder for å redegjøre om elevene har gjort sine lekser. Dette kan være at enkelt elever blir 
inspisert. Dermed vil læreren være klar over hvilke elever som ikke har gjort sine lekser, og da er 
det ikke nødvendig å initiere disse elevene til helklassesamtalen.
\newline
\newline
Øvelsen med tokolonnnenotatet hadde flere styrker, men den 
hadde flere organisatoriske svakheter. Det ble brukt for mye tid til å fordele elever i grupper, 
dette kunne gjerne ha blitt planlagt på forhånd. Dessuten var instruksjonene ikke helt klare, 
tydelighet i instruksjoner ville ha spart tid som kunne da brukes av elever i faglig aktivitet. 
Ifølge \citeA[s. ~189]{klet13}, faktorer som har direkte effekt på elevenes læring, fremheves en 
gjennomtenkt undervisningsopplegg som muliggjør at de bruker minimalt tid på ikke-faglige 
aktiviteter. For tokolonnenotatet er det også viktig å være klar over hvor mange 
frihetsgrader elever skal få (\citeNP{knai11}). Jo flere beslutninger eleven må ta selv, jo åpnere 
er oppgaven.
% \newline
% \newline
% Når naturfag rettferdiggjøres som et fag i skolen bruker man ofte to typer argumenter, som blir
% omtalt som produkt-argumentet og prosess argumentet, \citeA[s. ~351]{sjob04}. Produkt-argumentet går 
% ut på at naturfaglige kunnskaper, begreper og teorier er viktige både for eleven i skolehverdagen og 
% senere i arbeidslivet. Prosess-argumentet går ut på at det er naturvitenskapens prosesser, 
% arbeidsmåter og metoder som rettferdiggjør fagets plass i skolen. \iffalse \citeauthor{sjob04} \else 
% Sjøberg \fi skriver at selv om det er noe ``pedagogisk tidsmessig og tiltrekkende'' ved det synet 
% at det er prosessene som er det vesentlige, må det understrekes at produktorientert syn trenger ikke 
% å medføre \emph{en autoritær og doserende metodisk tilnærming} når dette produktet skal formidles 
% til elevene. \iffalse \citeauthor{sjob04} \else Han \fi skriver videre at det er viktig at 
% vitenskapens egenart ikke automatisk dikterer en metodisk tilnærming, eller motsatt, at man lar et 
% syn på læring definere hva som skal oppfattes som vitenskapens egenart. Undervisningsopplegget har 
% hatt en preg av begge disse syn på vitenskapens vesen. Innføring av nye begreper har styrket 
% elevenes syn på naturfag som et produkt, mens deres observasjoner i laboratoriet og skriving av 
% rapport har forsterket deres syn på naturfag som en prosess. 
% \newline
% \newline
% En overordnet ramme for arbeid med Forskerspiren er at elevene skal praktisere en vitenskapelig 
% metode. På 1960-tallet i USA og England kom læreplaner som blir omtalt
% for \emph{discover-learning}, \citeA[s. ~31]{knai11}. Her skulle elevene lære naturvitenskapelig 
% kunnskap gjennom aktiviteter som skulle ligne naturvitenskapelig forskning. I følge \iffalse 
% \citeauthor{knai11} \else Knain \fi  er det flere svakheter ved denne retningen. En av dem var 
% tanken at barn lærer naturfaglig begrepskunnskap gjennom induksjon, det vil si ved å trekke 
% sluttninger fra erfaringer. \iffalse \citeauthor{knai11} \else Knain \fi skriver videre at
% \begin{displayquote}
% Som Hodson påpeker:
% \begin{displayquote}
% Du kan ikke oppdage noe som du mangler begreper om. Du vet ikke hvor du skal se, hvordan du skal se 
% eller hvordan du skal gjenkjenne det når du har funnet det (Hodson 1996, s. 118).
% \end{displayquote}
% \end{displayquote}

% Selv om kognitiv forståelse alltid vil være avgjørende viktig i naturfag, så kan vi her spørre om det ikke nettopp er økt evne til deltagelse som bør være 
% begrunnelsen for å utvikle elevenes begrepsforståelse. Fokuset på forståelse må derfor suppleres med vektlegging av lesing for at naturfaget skal forberede til 
% demokratisk deltagelse \citeA[s. ~72]{kols09}.
% \citeA[s. 67]{roen15} referer til Kolstø når han argumenter om elever som aktive samfunnsborgere.
% \begin{displayquote}
% Kolstø (2009) peker også på at forståelse av begrepene vil være nødvendig med tanke på deltakelse 
% i samfunnet for øvrig. Slike begreper må være forstått for å kunne ta del i diskusjoner og argumentasjon, 
% og er dermed en del av en naturfaglig allmenndannelse.
% \end{displayquote}
% Demokrati er avhengig av 

Utforskende samtaler må først innlæres i en klasse slik at elevene kan få mest mulig
utbytte av sine felles diskusjoner og samtaler. \citeA[s. 57]{meli07} beskriver dette som kjernen i 
praksisen:
\begin{displayquote}
At the heart of the approach is the negotiation by each teacher and class of a set of ``ground
rules'' for talking and working together. These ground rules then become established as a set of 
principles for how the children will collaborate in groups.
\end{displayquote}
Slike regler bør derfor etableres ved et tidlig stadie for en gitt klasse, noe \citeA[s. 151]{ogd09}
også understøtter. Elevene bør rutineres i å tillate rom for alternative løsninger, uten å true 
gruppens solidaritet eller individets identitet. Disse reglene kan innøves gjennom flere 
anledninger: helklassesamtaler, gruppesamtaler, og par samtaler. Sist nevnte anledningen er passende
for 8. klassen, siden alle elever har en læringspartner. Ved å bruke bord plasseringen som allerede 
er på plass frigjør dette organiseringstid som isteden kan brukes mot fagrettet læring. 
\citeA{klet13} legger vekt på effektive instrukser som bidrar til mer fagrettet undervisning og 
større fokus på kognitive utfordringer.
\newline
\newline
Valg av grupper med forskjellige permutasjoner er noe som bør utprøves. For eksempel elever
som viser tydelige leder egenskaper bør jevnt fordeles i forskjellige grupper. Dette kan bidra til
å skape tydelige roller i grupper. Det som er viktig med denne tankegangen er at alle elever
i en gruppe bør føle at de har en unik rolle i gruppen, og at sammen kan de utforme et felles
produkt.

Tre viktige ting (under konstruksjon) : 

Ikke tydelige nok instrukser for den første del av tokolonneøvelsen. 
Det kan ha ført noen elever til å samarbeide isteden for å kollaborere. Design av 
gruppe oppgaven bør utformes slik at elevene er nødt til å jobbe sammen. Oppgaven
bør ikke være for enkel, at elevene kan jobbe individuelt med oppgavene, slik at
det ikke er noen nødvendighet for elevene å jobbe sammen. Tilsvarende gjelder for høy
vanskelighetsgrad, at de klarer ikke å danne forståelse eller mening. En gruppeoppgave
er da en oppgave som individet ikke klarer å utføre alene og er en oppgave som krever
kollaborasjon.  Åpne oppgaver er bedre tilegnet enn lukkede hvor fokuset er å finne 
en riktig svar. Dette er kanskje grunnen til at en sterk elev kan dominere samtalen 
(\citeNP[s. 31]{meli07}). Først og fremst er villigheten til deltagende til å dele
sin forståelse og ideer, og forsette med dette til tross for uenigheter i mellom,
faktor til en vellykket utforskende samtale. Positive relasjoner mellom elever
er derfor avgjørende for å skape et klima for kollaborasjon. \citeA{klet13}
kategoriserer dette som en underkategori i undervisningsmessigstøtte: støttende
klima - et klassemiljø preget av respekt, toleranse og engasjement (\citeNP[s. 191]{klet13}). 
Ifølge \citeA{ungd15} virker støttende klima til å være godt 
ivaretatt i norske klasserom.

Gruppe samtaler mister sin potens når det gjelder 
læringsutbytte hvis de har et preg av samarbeid men ingen kollaborasjon. 
Individets største utbytte fra utforskende samtaler er at ved endt kollaborasjon ender 
han opp med en ny oppfatning. En oppfatning som er farget av bidrag fra andre elever 
gjennom samtalene. Mangel av en ferdig produkt som kan leveres på OneNote. Konsolidering av 
hele øvelsen. Ferdig produkt ville vise felles forståelsen elevene hadde dannet i egne
tekster.

\end{document}