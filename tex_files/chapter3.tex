\documentclass[main.tex]{subfiles} 

\begin{document}
\section*{Refleksjon}
\label{sec:3}

I følge \citeA{ludv15} vil læringsprosesser som fører til forståelse bidra til å styrke elevenes motivasjon og 
opplevelse av mestring og relevans i skolehverdagen. Men, var dette tilfellet for vår klasse og hvordan
kunne undervisningsopplegget forbedres?

\citeA{mang13} innleder motivasjon som en trengsel for å ha lyst på noe eller ønske om å utføre en aktivitet.
Men han avslutter med følgende sitat
\begin{displayquote}
Motivasjon for å læra inneber noko meir enn lyst til å læra. Det handler om den mentale innsatsen til eleven.
Å lese ein tekst ti gonger kan indikera at eleven held ut, men læringsmotivasjon viser seg mellom anna gjennom
meir aktive studiestrategiar, slik som oppsummeringar, refleksjon over dei grunnleggjande ideane i faget og 
sammenfattingar av ideane med eigne ord.
\end{displayquote}
I dybdelæring bygger undervisningen på at studentene viser engasjement i faget. De er interessert i faget som sådan, er motivert til å komme til bunns i emner og har interaksjon med innholdet.
Bruken av tokolonnenotatet i første timen

Blooms taksonomi er et hierarki i seks nivåer: kunnskap, forståelse, anvendelse, analyse, syntese og evaluering. Dybdelæring forekommer blant de øverste niveåene
i Blooms taksonomi. Hvis elever skal lære gjennom de øverste nivåene i hierarkiet, må de selvsagt mestre de lavere nivåene.

Bruken av revoicing, se \citeA[s. ~175]{klet13}, til å gjenta og forsterke elevenes forslag og begrepsbruk ble ikke brukt tilstrekkelig gjennom den første timen.
For å kunne bruke revoicing mest mulig effektivt, må læreren raskt og effektivt bestemme om elevens repons har validitet og om det er relevant.
Gjennom personlig erfaring har revoicing vært vanskelig å utføre og krever veldig god grep på det \citeA{batp08} kaller Content Specific Knowledge, CSK.
I følge Klette, viser fravær av slike eksplisitte innramminger fra lærerens side at eleven blir sittende med et uklart kunnskapsinnhold og i verste fall feil
begrepsforståelse, \citeA[s. ~175-176]{klet13}. Klette referer til en annen studie \citeA[s. ~176]{klet13} når hun viser til viktigheten av at lærerne legger 
til rette for 
\emph{systematisk trening, øvelse og bruk av naturfaglige begreper for å utvikle elevenes naturfaglige forståelse, inkludert repitisjon av sentrale begreper.}

Når naturfag rettferdiggjøres som et fag i skolen bruker man ofte to typer argumenter, som blir
omtalt som produkt-argumentet og prosess argumentet, \citeA[s. ~351]{sjob04}. Produkt-argumentet går ut på at
naturfaglige kunnskaper, begreper og teorier er viktige både for eleven i skolehverdagen og senere i
arbeidslivet. Prosess-argumentet går ut på at det er naturvitenskapens prosesser, arbeidsmåter og metoder
som rettferdiggjør fagets plass i skolen. \iffalse \citeauthor{sjob04} \else Sjøberg \fi skriver at selv om det er noe \emph{pedagogisk 
tidsmessig og tiltrekkende} ved det synet at det er prosessene som er det vesentlige, må det understrekes at 
produktorientert syn trenger ikke å medføre \emph{en autoritær og doserende metodisk tilnærming} når dette produktet 
skal formidles til elevene. \iffalse \citeauthor{sjob04} \else Han \fi skriver videre at det er viktig at vitenskapens egenart ikke automatisk 
dikterer en metodisk tilnærming, eller motsatt, at man lar et syn på læring definere hva som skal oppfattes som 
vitenskapens egenart. Undervisningsopplegget har hatt en preg av begge disse syn på vitenskapens vesen. 
Innføring av nye begreper har styrket elevenes syn på naturfag som et produkt, mens deres observasjoner i 
laboratoriet og skriving av rapport har forsterket deres syn på naturfag som en prosess. 
\newline
\newline
En overordnet ramme for arbeid med Forskerspiren er at 
elevene skal praktisere en vitenskapelig metode. På 1960-tallet i USA og England kom læreplaner som blir omtalt
for \emph{discover-learning}, \citeA[s. ~31]{knai11}. Her skulle elevene lære naturvitenskapelig kunnskap gjennom aktiviteter som
skulle ligne naturvitenskapelig forskning. I følge \iffalse \citeauthor{knai11} \else Knain \fi  er det flere svakheter ved denne
retningen. En av dem var tanken at barn lærer naturfaglig begrepskunnskap gjennom induksjon, det vil si ved
å trekke sluttninger fra erfaringer. \iffalse \citeauthor{knai11} \else Knain \fi skriver videre at
\begin{displayquote}
Som Hodson påpeker:
\begin{displayquote}
Du kan ikke oppdage noe som du mangler begreper om. Du vet ikke hvor du skal se, hvordan du skal se eller
hvordan du skal gjenkjenne det når du har funnet det (Hodson 1996, s. 118).
\end{displayquote}
\end{displayquote}


% Totalt fravær av appetittvekker :
% Didaktiske prinsipper som å gi en faglig appetittvekker og introduksjon til et emne og å foreta en
% oppsummering av timens faginnhold og aktiviteter forekommer sjelden. Å bruke dette som et fast
% organiserende prinsipp, blir sjelden gjennomført 
\citeA{odeg10}.

% Opprinnelig hadde jeg planlagt å utføre innhenting av nødvendig materialer for å gjennomføre 
% mikroskop øvelsen ved å få elevene til å utføre innsamlingen gjennom en skoleutflukt. Utflukten var 
% en del av valgfaget friluftsliv. Fra vår klasse var det 12 elever som deltok i utflukten. Vår hensikt
% var opprinnelig å få disse elvene til å samle inn døde planter og vann. Derimot endte vi (dvs. 
% lærerstudentene) med å gjøre innsamlingen selv grunnet sen planlegging av timen. 


\citeA[s. ~77]{solv92} skriver at forståelse er aktivert kunnskap. Det vil si hver gang vi utsettes for en utfordring blir vårt eget \emph{kunnskapsreservoar}
tappet. Dermed aktiverer vi kunnskap. Elevenes kunnskaper utgjør en av forutsetningene for de nye kunnskapene vi tilfører dem. Disse kunnskapene, sammen med
elevenes erfaringer, utgjør det eleven kan møte nye utfordringer med. Dette betegnes også som kognitiv struktur av \citeauthor{solv92} og kan deles opp i 
delstrukturer. Piaget kaller slike delstrukturer for skjemaer \citeA[s. ~78]{solv92}. En elev har for eksempel ett skjema for celler og ett for organsystemmer. 
Det som er karakterisktisk for slike skjemaer er at de kan operere sammen. Hvis eleven behersker begrepene celler og organsystemmer, kan eleven danne koblingen
mellom disse skjemaene og dermed danne andre assosiasjoner til dyr. På denne måten konstruerer eleven ny kunnskap ved hjelp av den kunnskap hun har.
Hver elev vil ha sine skjemaer til å møte undervisning med.


% Selv om kognitiv forståelse alltid vil være avgjørende viktig i naturfag, så kan vi her spørre om det ikke nettopp er økt evne til deltagelse som bør være 
% begrunnelsen for å utvikle elevenes begrepsforståelse. Fokuset på forståelse må derfor suppleres med vektlegging av lesing for at naturfaget skal forberede til 
% demokratisk deltagelse 
\citeA[s. ~72]{kols09}

% Slike begreper må være forstått for å kunne ta del i diskusjoner og argumentasjon, og er dermed en del av en naturfaglig allmenndannelse. En slik 
% begrepsforståelse må ses i sammenheng med opplæring i lesing av slike tekster. Ellers kan det være vanskelig å bruke slik kunnskap i nye situasjoner. 
\citeA[s. ~67]{roen15}

\end{document}