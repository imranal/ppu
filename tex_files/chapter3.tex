\documentclass[main.tex]{subfiles} 

\begin{document}
\section*{Refleksjon}
\label{sec:3}

I følge \citeA{ludv15} vil læringsprosesser som fører til forståelse bidra til å styrke elevenes motivasjon og 
opplevelse av mestring og relevans i skolehverdagen. Men, var dette tilfellet for vår klasse og hvordan
kunne undervisningsopplegget forbedres?

\citeA{mang13} innleder motivasjon som en trengsel for å ha lyst på noe eller ønske om å utføre en aktivitet.
Men han avslutter med følgende sitat
\begin{displayquote}
Motivasjon for å læra inneber noko meir enn lyst til å læra. Det handler om den mentale innsatsen til eleven.
Å lese ein tekst ti gonger kan indikera at eleven held ut, men læringsmotivasjon viser seg mellom anna gjennom
meir aktive studiestrategiar, slik som oppsummeringar, refleksjon over dei grunnleggjande ideane i faget og 
sammenfattingar av ideane med eigne ord.
\end{displayquote}

% Naturfag som prosess og produkt \citaA{sjob04}\citeA{knai11}

% Totalt fravær av appetittvekker :
% Didaktiske prinsipper som å gi en faglig appetittvekker og introduksjon til et emne og å foreta en
% oppsummering av timens faginnhold og aktiviteter forekommer sjelden. Å bruke dette som et fast
% organiserende prinsipp, blir sjelden gjennomført 
\citeA{odeg10}.

% Opprinnelig hadde jeg planlagt å utføre innhenting av nødvendig materialer for å gjennomføre 
% mikroskop øvelsen ved å få elevene til å utføre innsamlingen gjennom en skoleutflukt. Utflukten var 
% en del av valgfaget friluftsliv. Fra vår klasse var det 12 elever som deltok i utflukten. Vår hensikt
% var opprinnelig å få disse elvene til å samle inn døde planter og vann. Derimot endte vi (dvs. 
% lærerstudentene) med å gjøre innsamlingen selv grunnet sen planlegging av timen. 


\citeA[s. ~77]{solv92} skriver at forståelse er aktivert kunnskap. Det vil si hver gang vi utsettes for en utfordring blir vårt eget \emph{kunnskapsreservoar}
tappet. Dermed aktiverer vi kunnskap. Elevenes kunnskaper utgjør en av forutsetningene for de nye kunnskapene vi tilfører dem. Disse kunnskapene, sammen med
elevenes erfaringer, utgjør det eleven kan møte nye utfordringer med. Dette betegnes også som kognitiv struktur av \citeauthor{solv92} og kan deles opp i 
delstrukturer. Piaget kaller slike delstrukturer for skjemaer \citeA[s. ~78]{solv92}. En elev har for eksempel ett skjema for celler og ett for organsystemmer. 
Det som er karakterisktisk for slike skjemaer er at de kan operere sammen. Hvis eleven behersker begrepene celler og organsystemmer, kan eleven danne koblingen
mellom disse skjemaene og dermed danne andre assosiasjoner til dyr. På denne måten konstruerer eleven ny kunnskap ved hjelp av den kunnskap hun har.
Hver elev vil ha sine skjemaer til å møte undervisning med.


% Selv om kognitiv forståelse alltid vil være avgjørende viktig i naturfag, så kan vi her spørre om det ikke nettopp er økt evne til deltagelse som bør være 
% begrunnelsen for å utvikle elevenes begrepsforståelse. Fokuset på forståelse må derfor suppleres med vektlegging av lesing for at naturfaget skal forberede til 
% demokratisk deltagelse 
\citeA[s. ~72]{kols09}

% Slike begreper må være forstått for å kunne ta del i diskusjoner og argumentasjon, og er dermed en del av en naturfaglig allmenndannelse. En slik 
% begrepsforståelse må ses i sammenheng med opplæring i lesing av slike tekster. Ellers kan det være vanskelig å bruke slik kunnskap i nye situasjoner. 
\citeA[s. ~67]{roen15}

\end{document}