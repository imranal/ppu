% !TEX spellcheck = en
\documentclass[main.tex]{subfiles} 

\begin{document}

\section*{Konklusjon}
\label{sec:4}

8. klassen gjennomgikk gamle og nye begreper gjennom helklassesamtaler og utforskende 
samtaler. Det viste seg at å kalle et samarbeid mellom elever for utforskende samtale
er problematisk. Et fåtall grupper demonstrerte kvalitet på gruppesamtaler som kan
klassifiseres som kollaborasjon. For å danne gode vaner blant elever, er det nødvendig 
å innføre klare regler og rutiner. Elevene må opplæres i hvordan de skal kollaborere 
med hverandre. I tillegg må lærere tilrettelegge gode oppgaver der slutt resultatet 
ikke er entydig. Gruppesamtaler mister sin potens når det gjelder læringsutbytte hvis 
de har et preg av samarbeid men ingen kollaborasjon. Individets største utbytte fra 
utforskende samtaler er at ved endt kollaborasjon ender han opp med en ny oppfatning. 
En oppfatning som er farget av bidrag fra andre elever gjennom samtalene.
I motsetning til helklassesamtaler, tilbyr utforskende samtaler et bedre innblikk
inn i elevenes begrepsforståelse. Her kan lærer lettere observere elevene og veilede
dem i den nærmeste utviklingssonen.
\newline
\newline
Det er ikke et spørsmål om utforskende samtaler kan danne god begrepsforståelse, men 
hvordan.



\end{document}
