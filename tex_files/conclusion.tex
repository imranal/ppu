% !TEX spellcheck = en
\documentclass[main.tex]{subfiles} 

\begin{document}

\section*{Konklusjon}
\label{sec:4}

8. klassen gjennomgikk gamle og nye begreper gjennom helklassesamtaler og utforskende 
samtaler. Det viste seg at å kalle et samarbeid mellom elever for utforskende samtale
er problematisk. Et få tall grupper demonstrerte kvalitet på gruppesamtaler som kan
klassifiseres som kollaborasjon. For å danne gode vaner blant elever, er det nødvendig 
å innføre klare regler og rutiner. Elevene må opplæres i hvordan de skal kollaborere 
med hverandre. I tillegg må lærere tilrettelegge gode oppgaver der slutt resultatet 
ikke er entydig. Jeg oppdaget også at tydelige læringsmål er nødvendig for å skape 
læringssituasjoner som er fokusert på det faglige. Når det er sagt, hadde 
undervisningsopplegget preg av mange konsolideringssituasjoner. Undervisningen hadde 
også en del varisjon i arbeidsmåter. 

Det er ikke et spørsmål om utforskende samtaler kan danne god begrepsforståelse, men 
hvordan. Hvordan???

\end{document}