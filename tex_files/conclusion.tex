% !TEX spellcheck = en
\documentclass[main.tex]{subfiles} 

\begin{document}

\section*{Konklusjon}
\label{sec:6}

I introduksjonen ble det blant annet nevnt at ifølge \citeA{klet13} er det to dimensjoner i 
undervisningsmessigstøtte som bør fokuseres på for å øke læringstrykket, det vil si effektivisere 
timene slik at elevene kan ha mest mulig fokus på det faglige. Problemstillingen i denne oppgaven har 
vært å fokusere på en av punktene i undervisningsmessigstøtte, nemlig \emph{klare læringsmål og 
intensjoner}. Hvis klare læringsmål og intensjoner kan brukes effektivt, bidrar det tid til å fokusere 
på blant annet \emph{relevante kognitive utfordringer}. Det gjenstår å finne ut av hvordan 
underviseren kan øke kvaliteten på oppgaver og dermed utfordre sine elever kognitivt. En av de 
viktigste fokuseringsområder til Ludvigsen-utvalget er dybdelæring. For at elevene
skal nå kompetansemål som utfyller muligheten for dybdelæring vil det være sterk behov å lage og -bruke 
oppgaver som bidrar til relevante kognitive utfordringer. 
\newline
\newline
\emph{
Utvalget mener at mer dybdelæring i skolen vil bidra til at elevene behersker sentrale elementer
i fagene bedre og lettere kan overføre læring fra ett fag til et annet. Forståelse av det eleven har
lært, er en forutsetning for og en konsekvens av dybdelæring. Skoler som legger bedre til rette for
læringsprosesser som fører til forståelse, kan bidra til å styrke elevenes motivasjon og opplevelse
av mestring og relevans i skolehverdagen.} - \citeA{ludv15}

\end{document}