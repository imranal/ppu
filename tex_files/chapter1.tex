\documentclass[main.tex]{subfiles} 
\begin{document}

\section*{Undervisningsopplegget}
\label{sec:1}
Fokuset i undervisningen jeg vil utføre i 8. klassen vil være rundt begrepene celler og celledeling. 
I tillegg skal elevene instrueres i å skrive en rapport til et eksperiment de skal utføre relatert 
til disse begrepene. Hensikten med opplegget er å formidle til elever vanskelige begreper fra 
naturfag slik at de kan lettere se sammenhenger mellom temaer. Temaer som forøvrig blir memorisert 
og forstått på et lavt nivå, i henhold til nivåene som er definert utfra kompetansemålene. 
\newline
\newline
Undervisningen er fordelt på 3 skoletimer over 2 uker. Opplegget (se vedlegg \ref{sec:plan}) 
utførte jeg alene, med veileder og en medstudent som observatører. De bidro også i blant 
med å gi veiledning når elevene jobbet enten selvstendig eller sammen i grupper. I denne 
oppgaven velger jeg å utdype den første timen. Først vil jeg gjøre rede for undervisningsopplegget 
og deretter kommer jeg til å analysere opplegget i lys av teori i pedagogikk og naturfagdidatikk.
\newline
\newline
Timen starter med en oppsummering av det elevene har lært hittil om celler og mikroskop og en 
gjennomgang av deres lekser. Helklassesamtalen foregår som en dialog med tavle som hjelpemiddel. 
Elevene initieres til å reflektere over temaer og begreper de har lært. Ettersom elevene gjennom 
helklassesamtalen har blitt ``varmet'' opp kognitivt, er de mottagelige for å lære om et nytt tema. 
Dermed innføres temaet encellede organismer. Innføringen av temaet er satt opp på en slik 
måte at overgangen fra repetisjon til det nye temaet blir naturlig og flytende. Hensikten med 
innføringen er tredelt : å gjøre elevene bevisst om at det finnes forskjellige 
type organismer, forberede de for den neste timen hvor flercellede organismer blir introdusert, og 
til slutt i den siste timen studere encellede organismer gjennom et mikroskop\footnote{Det kan sies 
at den naturlige rekkefølgen ville ha vært å studere de encellede organismene i den andre timen. 
Siden organismene som skulle studeres måtte vokses frem i laboratoriet over en ukestid, var det ikke 
mulig å koordinere det bedre.}.
\newline
\newline
I den siste delen av timen utføres en øvelse der elevene skal jobbe sammen med tokolonnenotatet i 
grupper (se vedlegg \ref{sec:tokolonnenotat}), hvor de skal bli enige med hverandre om hva som er 
viktig å formidle videre om deres felles temaer. Deretter fordeles de i nye grupper slik at hver 
gruppe har minst en elev som har forbredt sitt sett med begreper. Under hele denne prosessen er jeg 
tilgjengelig til å gå rundt for å høre elevene diskutere begreper, først sammen i grupper, og 
deretter individuelt når de fremfører sine konklusjoner med medelever. Hvis det observeres at en 
elev har problemer med å gi tilstrekkelig respons på et gitt tema, initieres eleven i en dialog 
hvor vi forsøker å sammen konstruere en mer utdypet forståelse av begrepene.
Tokolonnenotatet øvelsen hadde hensikt å skape dypere forståelse av faglig begreper, 
gjennom repitisjon og muntlig bruk av naturfaglig begreper. Øvelsen var delvis lærerstyrt, 
men hadde stor grad av åpenhet rundt produktet og kunnskapsutbytte.
\newline
\newline
Dermed er det naturlig å dele timen i tre deler:
\begin{enumerate}
\item Aktivering av forkunnskaper  
\item Innføring av nytt tema
\item Gruppe samtaler
\end{enumerate}
Vi kommer nå til å analysere disse punktene.

\end{document}