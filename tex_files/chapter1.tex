\documentclass[main.tex]{subfiles} 

\begin{document}

\section*{Undervisningsopplegget}
\label{sec:1}
Fokuset i undervisningen jeg vil utføre i 8. klassen vil være rundt begrepene celler og celledeling. 
I tillegg skal elevene instrueres i å skrive en rapport til et eksperiment de skal utføre relatert 
til disse begrepene. Hensikten med opplegget er å formidle til elever vanskelige begreper fra 
naturfag slik at de kan lettere se sammenhenger mellom temaer. Temaer som forøvrig blir memorisert 
og forstått på et lavt nivå, i henhold til nivåene som er definert utfra kompetansemålene. 
\newline
\newline
Undervisningen er fordelt på 3 skoletimer over 2 uker. Opplegget (se vedlegg \ref
{sec:plan}) utførte jeg alene, med veileder og en medstudent som observatører. De bidro 
også i blant med å gi veiledning når elevene jobbet enten selvstendig eller sammen i grupper. I 
denne oppgaven velger jeg å utdype den første timen. Først vil jeg gjøre rede for 
undervisningsopplegget og deretter kommer jeg til å analysere opplegget i lys av teori i pedagogikk 
og naturfagdidatikk.
\newline
\newline
Timen starter med en oppsummering, gjennom helklassesamtale, av det elevene har hittil lært om 
mikroskop og cellestrukturen. Helklassesamtalen foregår som en dialog med tavle som hjelpemiddel. 
Elevene initieres til å reflektere over temaer og begreper de har lært og hatt lekser om. Siden 
elevene gjennom helklassesamtalen har blitt "varmet" opp kognitivt, er de mottagelige for å 
lære om et nytt tema. Innføringen av nytt tema er bevisst satt opp på en slik måte at overgangen 
fra repetisjon til det nye temaet blir naturlig og flytende. I timene hvor de har hatt en innføring 
om celler, har de lært om basale strukturer. I denne timen går de litt dypere ved å få en innføring 
om encellede organismer. Hensikten med innføringen er tredelt : å gjøre elevene bevisst om at det 
finnes forskjellige type organismer, forberede de for den neste timen hvor flercellede organismer 
blir introdusert, og til slutt i den siste timen studere encellede organismer gjennom et mikroskop
\footnote{Det kan sies at den naturlige rekkefølgen ville ha vært å studere de encellede organismene 
i den andre timen. Siden organismene som skulle studeres måtte vokses frem i laboratoriet over en 
ukestid, var det ikke mulig å koordinere det bedre. Prøvene ble dessuten samlet i forbindelse med en 
klassetur til en skog gjennom valgfaget FUTT : Friluftsliv, Uteaktivitet, Trivsel og Turmat.}.
\newline
\newline
I den siste delen av timen utføres en øvelse der elevene skal jobbe sammen med tokolonnenotatet i 
grupper (se vedlegg \ref{sec:tokolonnenotat}), hvor de skal bli enige med hverandre om hva som er 
viktig å formidle videre om deres felles temaer. Deretter fordeles de i nye grupper slik at hver 
gruppe har minst en elev som har forbredt sitt sett med begreper. Under hele denne prosessen er jeg 
tilgjengelig til å gå rundt for å høre elevene diskutere begreper, først sammen i grupper, og 
deretter individuelt når de fremfører sine konklusjoner med medelever. Hvis det observeres at en 
elev har problemer med å gi tilstrekkelig respons på et gitt tema, initieres eleven i en dialog 
hvor vi forsøker å sammen konstruere en mer utdypet forståelse av begrepene.

\end{document}
