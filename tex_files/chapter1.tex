\documentclass[main.tex]{subfiles} 
\begin{document}

\section*{Undervisningssituasjonen}
Skolen hvor undervisningsopplegget ble utført befinner seg i et område hvor det er gode 
sosioøkonomiske forhold. Klassen som vi, praksisstudentene, observerte var en 8. klasse, som består 
av 13 gutter og 11 jenter. I klassen sitter elevene to-og-to sammen ved sine pulter i et rutenett. 
Annenhver uke byttes plasseringen til elevene. Elevene blir fordelt sammen med det skolen kaller 
læringspartnere. Hensikten med læringspartnere er at de kan snakke sammen når de jobber med oppgaver 
eller når de blir bedt om å diskutere noe.  Det er generelt ingen sosiale problemer 
eller konflikter i klassen, og elevene pleier å samarbeide med hverandre uten store problemer. Tavlen 
brukes sjelden siden lystavlen er plassert i alle klasserom rett foran tavlen. OneNote brukes isteden 
for tavlen, og OneNote brukes også til planleggingen av undervisningen.
\newline
\newline
Jeg og en annen lærerstudent observerte elevene fra 8. klassen i både naturfagstimer og 
matematikktimer. Elevenes faglige forutsetninger er varierende, klassen har en jevn fordeling av 
faglig sterke og faglig svake elever. I en naturfagstime observerte vi at elevene brukte mikroskop 
for å studere diverse celleprøver, blant annet fra deres egen munn og deres egne hårstrå. Timen 
startet med repetisjon av begreper om celler og mikroskop. Elevene ble fordelt i grupper på 3-4 
stykker, og læreren gikk rundt og veiledet alle gruppene. Noen av gruppene fikk 
hjelp fra læreren med å innstille mikroskopene slik at de endte opp med riktig fokus. Deretter brukte 
læreren et mikroskop som var koblet til en datamaskin. Bildet fra mikroskopet ble projisjert på 
lystavlen i laboratoriet. Hensikten med denne øvelsen var å gi elevene en pekepinn på 
størrelsesordner for celler og demonstrere bruk av mikroskop. Etter timen bemerket læreren at 
elevene forsatt ikke har lært å skrive en rapport. Dette inspirerte meg til å bruke et tilsvarende 
opplegg til å strukturere mine egene undervisningstimer, og innføre en avsluttende rapport 
slik at elevene kan begynne å danne gode vaner for å skrive om sine observasjoner og resultater.

\section*{Undervisningsopplegget}
\label{sec:1}
Undervisningen er fordelt på 3 skoletimer over 2 uker. Opplegget (se vedlegg \ref{sec:plan}) 
utførte jeg alene, med veileder og en medstudent som observatører. De bidro også i blant 
med å gi veiledning når elevene jobbet enten selvstendig eller sammen i grupper. I denne 
oppgaven velger jeg å utdype den første timen. 
\newline
\newline
Timen startet med en oppsummering av det elevene hittil hadde lært om celler og mikroskop og en 
gjennomgang av deres lekser. Helklassesamtalen foregikk som en dialog med tavle og OneNote som hjelpemiddel. 
Elevene ble initiert til å reflektere over temaer og begreper de hadde tidligere lært. Ettersom elevene gjennom 
helklassesamtalen hadde blitt ``varmet'' opp kognitivt, var de mottagelige for å lære om et nytt tema. 
Dermed ble temaet encellede organismer innført. Innføringen av temaet var satt opp på en slik 
måte at overgangen fra repetisjon til det nye temaet ble naturlig og flytende. Hensikten med 
innføringen var tredelt : å gjøre elevene bevisst om at det finnes forskjellige type organismer, 
forberede de for den neste timen hvor flercellede organismer blir introdusert, og 
til slutt i den siste timen studere encellede organismer gjennom et mikroskop. \footnote[2]{Det kan sies 
at den naturlige rekkefølgen ville ha vært å studere de encellede organismene i den andre timen. 
Siden organismene som skulle studeres måtte vokses frem i laboratoriet over en ukes tid, var det ikke 
mulig å koordinere det bedre.}
\newline
\newline
I den siste delen av timen ble en øvelse utført der elevene jobbet sammen med tokolonnenotatet i 
grupper (se vedlegg \ref{sec:tokolonnenotat}). I denne øvelsen skulle elevene bli enige med hverandre om hva som er 
viktig å formidle videre om deres felles temaer. Deretter ble de fordelt i nye grupper slik at hver 
gruppe hadde minst en elev som hadde forbredt sitt sett med begreper. Under hele denne prosessen var jeg 
tilgjengelig til å gå rundt for å høre elevene diskutere begreper, først sammen i grupper, og 
deretter individuelt når de fremførte sine oppsummeringer med medelever. Hvis jeg observerte at en 
elev hadde problemer med å gi tilstrekkelig respons på et gitt tema, initierte jeg eleven i en dialog 
hvor vi forsøkte å sammen konstruere en mer utdypet forståelse av begrepene.
Tokolonnenotat øvelsen hadde til hensikt å skape dypere forståelse av naturfaglige begreper, 
gjennom repetisjon og muntlig bruk av begrepene. Øvelsen var delvis lærerstyrt, 
men hadde stor grad av åpenhet (se også frihetsgrader: \citeNP{knai11}) rundt produktet, det vil
si frihet rundt det elevene kunne skrive i sine notater, og kunnskapsutbyttet de endte opp med.
\newline
\newline
Dermed er det naturlig å dele timen i tre deler:
\begin{enumerate}
\item Aktivering av forkunnskaper  
\item Innføring av nytt tema
\item Gruppesamtaler
\end{enumerate}
Hvordan ble undervisningen lagt opp for å skape god begrepsforståelse i naturfagstimen, 
og hvordan bidro gruppesamtalene til dette? For å svare på dette, la oss se nærmere på hele 
undervisningssekvensen. Jeg vil nå drøfte hver av disse tre punktene, og hvordan utforskende
samtaler bidro til å skape god begrepsforståelse blant elevene.

\end{document}
