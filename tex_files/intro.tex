\documentclass[main.tex]{subfiles} 

\begin{document}

\section*{Problemstilling}    
I alle studier som diskuterer hva som kjennetegner \emph{god undervisning}, knyttes 
dette ofte til tre dimensjoner\cite[Klette 2013, side 142]{klette}; som den 
engelskspråkelige litteraturen kaller 
\begin{itemize}
\item emotional support (emosjonell støtte),
\item organisational support (organisatorisk støtte),
\item instructional support (undervisningsmessig støtte).
\end{itemize}
I korte trekke sammenfatter emosjonell støtte klasseromsarbeidet som knytter
seg til de sosialse og emosjonelle rammene for læringsarbeidet \cite[Klette 2013, side 143]{klette},
organisatorisk støtte viser til fysisk organisering og klasseledelse \cite[Klette 2013, side189]{klette}
og undervisningsmessig støtte retter fokus mot lærerens sentrale rolle i elevenes kunnskapstilegnelse; ''det 
læreren gjør av faktisk undervisningshandlinger i klasserommet som bidrar til læring'' 
\cite[Klette 2013, side 143 og 146]{klette}.
\newline

Undervisningsmessig støtte kan igjen fordeles blant 4 dimensjoner\cite[Klette 2013, side 146]{klette}
\begin{enumerate}
\item klare læringsmål og intensjoner
\item relevante kognitive utfordringer
\item kvaliteten på klassesamtalene
\item støttende klima.
\end{enumerate}

Ifølge Ungdata\cite[legg til referanse ungdata]{ungdata}\cite[Klette, side 144]{klette} har Norge på nasjonal 
plan siden 1992 hatt et løft når det gjelder elev-lærer 
relasjon, og det psyko-sosiale miljøet på skolen har merkant 
forbedret seg. Det er fære elever som melder at de gruer seg til å 
gå på skolen og fære skulker. Generelt har trivsel 
blant elever økt, og det er etablert et godt læringsmiljø. Derimot er det 
forsatt rom for forbedring når det gjelder hvor effektivt elever tar imot 
instruksjoner og hvorvidt de blir kognitiv utfordret. 
\newline

Problemstillingen jeg har valgt i denne oppgaven fokuserer på tiltak lærere kan ta for å
formidle klare instrukser i forbindelse med utføring av en naturfag time, ved å formidle instrukser 
både muntlig, skriftlig og visuelt.
\newline

Under observasjoner og 
utføring av undervisningssekvens har fokus vært på hvordan 
lærere kan bli flinkere til å delegere oppgaver og formidle 
informasjon. Hvis instrukser ikke er tydelige nok, vil elevene bruke unødvendig 
lengere tid til å komme i gang med undervisningsaktiviteten. Det 
er grunn til å tro at effektiv formidling av instrukser kan i 
helhet spare tid som igjen kan brukes i andre klasseaktiviteter. Den mest selvsagte måte
å rette på dette er at lærer krever at ingen praktiske
spørsmål kan stilles etter at instrukser har blitt formidlet. Da gjenstår
det kun rom for faglige spørsmål. Dette kan derimot kvele engasjement og er
rett og slett ikke et godt nok løsning. Det er derimot viktigere at læreren
gir gode instrukser og forsatt tillater rom for spørsmål rundt instruksene.
Dermed faller denne oppgaven til læreren som må tydeligere
etablere lederrollen og foreta tiltak for å formidle instrukser 
effektivt.

\subsection*{Undervisningsopplegget}
\textbf{Kompetansemål i læreplanen}
\newline
Forskerspiren :
\newline
formulere testbare hypoteser, planlegge og gjennomføre undersøkelser av dem og diskutere observasjoner og 
resultater i en rapport
\newline

Mangfold i naturen :
\newline
beskrive oppbygningen av dyre- og planteceller og forklare hovedtrekkene i fotosyntese og celleånding
gjøre rede for celledeling og for genetisk variasjon og arv
\newline

Mål for dette undervisningsopplegget 
\newline
Gjøre rede for celledeling og DNA, beskrive oppbygningen av celler, gjøre rede for encellede -og flercellede organismer
og deres oppbygging. Innhente prøver av planter fra en dam og oppbevare de i laboratoriet i en ukes tid for å vokse 
fram mikroorganismer. Bruke mikroskop til å studere mikroorganismer; hvordan de ser ut og beveger seg, og skrive en 
rapport om forsøket.


\subsection*{Klassen}
Skolen er lokalisert i et godt sosioøkonomisk område, deriblant har foreldrene til elevene høy utdanningsbakgrunn. 8.klassen består av 13 gutter og 11 jenter. En skoletime varer i 50 minutter, efterfulgt av en 10 minutter lang pause. Elevene ved skolen har i gjennomsnitt 27.6 timer i uka. I klassen sitter elevene to-og-to sammen ved sine pulter i et rutenett. Hver andre uke byttes plasseringene til elevene. Elevene blir fordelt sammen med det skolen kaller læringspartnere. Læreren printer et nytt klassekart som han/hun har tilgjengelig på sin kateter/podium. Elever pleier å legge fra sine mobiler i en hylleplass eller deres bokskap. Når en time starter, står elevene opp i sine stoler og hilser på læreren før de får lov til sitte. Tavlen brukes sjelden, siden lystavlen er ofte plassert i alle klasserom foran tavlen. Onenote brukes flittig gjennom undervisning og til planleggingen av undervisningen. Elevene har også blitt velkjent med Onenote ved å se lærere bruke den, og selv bruke den i sine delingstimer. Lekser blir ført i It’s Learning plattformen. I klassen vi observerte kommer det 3 elever fra velkomstklassen som deltar i undervisning torsdag og fredag hver uke. Disse elevene har ofte problemmer med å forstå norsk, men de er flinkere til å lese og skrive. I blant bruker deres kontaktlærer engelsk for å formidle informasjon. Men som regel blir helklasse undervisningen ført i norsk. Det er generelt ingen sosiale problemmer eller konflikter i klassen, og elevene pleier å samarbeide med hverandre uten store problemmer. Skolen har en del problemmer med elever som trenger en eller annen form for tilrettelegging. I trinnmøter til 8.trinn blir det i blant tatt opp spørsmål om hvem som skal ha tilpasning og hvordan det skal utføres. Fokuset til skolen er å tilby sine elever et godt psykososial læringsmiljø.
\newpage\null

\end{document}
