\documentclass[main.tex]{subfiles} 

\begin{document}

\section{Problemstilling}
Ifølge \citeA{ungd15} har Norge på nasjonal plan siden 1992 hatt et løft når det gjelder elev-lærer 
relasjon, og det psykososiale miljøet på skolen har merkant forbedret seg. Det er fære elever 
som melder at de gruer seg til å gå på skolen og fære skulker. Generelt har trivsel 
blant elever økt, og det er etablert et godt læringsmiljø. Derimot er det forsatt rom for forbedring 
når det gjelder hvor effektivt elever tar imot instruksjoner og hvorvidt de blir kognitiv utfordret. 
\newline

I de fleste studier som diskuterer hva som kjennetegner \emph{god undervisning}, knyttes 
dette ofte til tre dimensjoner \citeA[s. 142]{klet13}, som den engelskspråkelige litteraturen kaller 
\begin{itemize}
\item emotional support (emosjonell støtte),
\item organisational support (organisatorisk støtte),
\item instructional support (undervisningsmessig støtte).
\end{itemize}
I korte trekk sammenfatter emosjonell støtte klasseromsarbeidet som knytter
seg til de sosiale og emosjonelle rammene for læringsarbeidet, %\citeA[s. ~143]{klet13},
organisatorisk støtte viser til fysisk organisering og klasseledelse %\citeA[s. ~189]{klet13}
og undervisningsmessig støtte retter fokus mot lærerens sentrale rolle i elevenes kunnskapstilegnelse, 
"\emph{det læreren gjør av faktisk undervisningshandlinger i klasserommet som bidrar til læring}" 
\citeA{klet13}.
%\citeA[s. ~143 og s. ~146]{klet13}".
\newline
Undervisningsmessig støtte kan igjen deles i 4 dimensjoner, \citeA[s. 146]{klet13},
\begin{enumerate}
\item klare læringsmål og intensjoner,
\item relevante kognitive utfordringer,
\item kvaliteten på klassesamtalene,
\item støttende klima.
\end{enumerate}
Her er det flere viktige faktorer som bidrar til læring. Deriblant kvaliteten på oppgaver, variasjon i
oppgavenes vanskelighetsgrad, og oppgaver som fordrer kognitivt, hvor akkurat tilstrekkelig tid\footnote{
Nærstudier av gruppeoppgaver i norsktimene i PISA og videostudien \citeA{klal13} viste at selv om oppgavene
var engasjerende og relevante, elevene fikk for god tid til å løse oppgavene, og da ble den kognitive utfordringen 
intetsigende. Foreksempel, hvis elevene fikk 20 minutter til å løse en oppgave, klarte de å utføre det på 6 
minutter og den resterende tiden ble da brukt til ikke-faglig diskusjoner.} er gitt til løsning av oppgaver. 
Dette faller derfor under punkt 2 : \emph{relevante kognitive utfordringer}. 
Introduksjon til nye fagtemaer, utvikling av elevenes synspunkter og ideer, faller under punkt 1 : 
\emph{klare læringsmål og intensjoner}. Effektive lærere bruker mer tid rettet mot faglig undervisning. 
De utøver også klar og tydelig klasseromsledelse, som igjen gir mer tid til undervisning rettet mot fag. 
Kvaliteten på klassesamtaler skaper elevengasjement og deltagelse. Forskning basert på mikroanalyser av språk
og kommunikasjoner
Klare rutiner og regler skaper et klassemiljø som er preget av respekt, toleranse og engasjement. Ungdata
tyder på at norske klasserom viser god støttendeklima for læring, og generelt blir kvaliteten på klassesamtaler
opprettholdt gjennom dialogisk samtaleform.

Problemstillingen jeg har valgt å fokusere på i denne oppgaven er det første punktet i undervisningsmessig støtte : 
\emph{klare læringsmål og intensjoner}. Derfor spør jeg følgende :
\newline
\textbf{Hvordan kan lærere forbedre sine instrukser i utføringen av en naturfagtime ved å ta i bruk både muntlige,
skriftlige og visuelle kommunikasjonsmidler ?}
\newline

Under observasjoner og utføring av undervisningssekvens har fokus vært på hvordan 
lærere kan bli flinkere til å delegere oppgaver og formidle 
informasjon. Hvis instrukser ikke er tydelige nok, vil elevene bruke unødvendig 
lenger tid på å komme i gang med undervisningsaktiviteten. Det 
er grunn til å tro at effektiv formidling av instrukser kan i 
helhet spare tid som igjen kan brukes i andre klasseaktiviteter. Den mest selvsagte måte
å rette på dette er at lærer krever at ingen praktiske
spørsmål kan stilles etter at instrukser har blitt formidlet. Da gjenstår
det kun rom for faglige spørsmål. Dette kan derimot kvele engasjement og er
rett og slett ikke en god nok løsning. Det er derimot viktigere at læreren
gir gode instrukser og forsatt tillater rom for spørsmål rundt instruksene.
Dermed faller denne oppgaven til læreren som må tydeligere
etablere lederrollen og foreta tiltak for å formidle instrukser 
effektivt. Elevene vil også forsatt ha muligheten til å kommunisere med sine 
medelever/samtalepartnere.

\subsection*{Klassen}
Skolen er lokalisert i et godt sosioøkonomisk område, deriblant har foreldrene 
til elevene høy utdanningsbakgrunn. 8.klassen består av 13 gutter og 11 jenter. 
En skoletime varer i 50 minutter, efterfulgt av en 10 minutter lang pause. Elevene 
ved skolen har i gjennomsnitt 27.6 timer i uka. I klassen sitter elevene to-og-to sammen 
ved sine pulter i et rutenett. Hver andre uke byttes plasseringene til elevene. 
Elevene blir fordelt sammen med det skolen kaller læringspartnere. Læreren printer 
et nytt klassekart som han/hun har tilgjengelig på sin kateter/podium. Elever pleier 
å legge fra sine mobiler i en hylleplass eller deres bokskap. Når en time starter, 
står elevene opp i sine stoler og hilser på læreren før de får lov til sitte. 
Tavlen brukes sjelden, siden lystavlen er ofte plassert i alle klasserom foran 
tavlen. Onenote brukes flittig gjennom undervisning og til planleggingen av 
undervisningen. Elevene har også blitt velkjent med Onenote ved å se lærere 
bruke den, og selv bruke den i sine delingstimer\footnote{En time der elever 
fra forskjellige klasser får felles undervisning, og deretter deles de i 
grupper og jobber sammen til å løse oppgaver på OneNote. Deres progressjon 
blir overvåket av en lærer som kan se live all input.} Lekser blir ført i 
It’s Learning plattformen. I klassen vi observerte kommer det 3 elever fra 
velkomstklassen som deltar i undervisning torsdag og fredag hver uke. 
Disse elevene har ofte problemmer med å forstå norsk, men de er flinkere 
til å lese og skrive. I blant bruker deres kontaktlærer engelsk for å 
formidle informasjon. Men som regel blir helklasse undervisningen ført 
i norsk. Det er generelt ingen sosiale problemmer eller konflikter i 
klassen, og elevene pleier å samarbeide med hverandre uten store problemmer. 
Skolen har en del problemmer med elever som trenger en eller annen 
form for tilrettelegging. I trinnmøter til 8.trinn blir det i blant 
tatt opp spørsmål om hvem som skal ha tilpasning og hvordan det skal 
utføres. Fokuset til skolen er å tilby sine elever et godt psykososial 
læringsmiljø.
\newline

\end{document}
