
\documentclass[main.tex]{subfiles} 
\begin{document}

\setlength{\epigraphwidth}{0.8\textwidth}
\epigraph{``You can know the name of a bird in all the languages of the world, 
but when you're finished, you'll know absolutely nothing whatever about the bird...
So let's look at the bird and see what it’s doing - that's what counts. I learned 
very early the difference between knowing the name of something and knowing something.''}
{\textit{Richard Feynman}}

\section*{Problemstilling}

I naturfag er det veldig mange begreper elever skal mestre. For at de skal kunne danne en god 
overordnet forståelse for faget, er det da viktig at de kan gå fra enkeltstående begreper til 
koblinger mellom begreper og være klar over de logiske sammenhengene. Det er derfor viktig fra 
læringspersktivet at undervisningen er forståelsesorientert, fremfor fakta-orientert. I \citeA
{ludv15} står det blant at
\begin{displayquote}
Skoler som legger bedre til rette for læringsprosesser som fører til forståelse, kan bidra til å 
styrke elevenes motivasjon og opplevelse av mestring og relevans i skolehverdagen. 
(Ludvigsen-utvalget 2015)
\end{displayquote}
Dermed trekker utvalget en kobling mellom forståelse og elevenes motivasjon og opplevelse av
mestring og relevans i skolehverdagen. \citeA[s. 176]{klet13} viser til viktigheten av at 
lærerere legger til rette for ``systematisk trening, øvelse og bruk av naturfaglige begreper 
for å utvikle elevenes naturfaglige forståelse''. I den sosiokulturelle tradisjonen rettes 
fokus mot læring i felleskap før kunnskap blir internalisert på individnivå. Blant annet 
inkluderer dette arbeid i grupper. Samtalekvaliteten på gruppearbeid kan ha et stort spenn. 
\citeA[s. 58-59]{meli07} definerer tre distinkte klassifiseringer for slike samtaler:
``Disputational'', ``Cumulative'' og ``Exploratory''. Den sist nevnte klassifikasjonen,
også kalt utforskende samtaler, utgjør gruppearbeid som har en preg av kollaborasjon og dermed
regnes som den mest ønskelige samtaleformen. Det kan derfor tenkes at utforskende samtaler 
kan bidra til å skape god begrepsforståelse i naturfag. For å undersøke dette vil jeg se 
på en undervisningssekvens jeg utførte i en ungdomskole for en 8. klasse.
\newline
\newline
Derfor er min problemstilling følgende:
\newline
\newline
\textbf{Hvordan bidrar utforskende samtaler til å skape god begrepsforståelse i en naturfagstime 
for 8. trinn?}
\newline
\newline
Fokuset i undervisningen jeg utførte i 8. klassen var rundt begrepene celler og celledeling. 
I tillegg ble elevene instruert i å skrive en rapport til et eksperiment de utførte relatert 
til disse begrepene. Hensikten med opplegget var både å formidle og la elevene selv bruke vanskelige 
begreper fra naturfag slik at de lettere kan se sammenhenger mellom temaer. Temaer som ellers
blir memorisert og forstått på et lavere nivå av elevene, når de lærer begreper som de skal
gjengi, i henhold til kompetansenivåene som er definert utfra kompetansemålene.
\newline
\newline
Undervisningsopplegget som jeg operasjonaliserte var basert på følgende kompetansemål i 
læreplanen
\begin{displayquote}
Forskerspiren:
\begin{itemize}
\vspace{-2mm}
\item formulere testbare hypoteser, planlegge og gjennomføre undersøkelser 
av dem og diskutere observasjoner og resultater i en rapport
\end{itemize}
Mangfold i naturen:
\begin{itemize}
\vspace{-2mm}
\item beskrive oppbygningen av dyre- og planteceller og forklare hovedtrekkene i fotosyntese 
og celleånding
\vspace{-2mm}
\item gjøre rede for celledeling og for genetisk variasjon og arv
\end{itemize}
\end{displayquote} 
Fra kompetansemålene i \emph{Mangfold i naturen} blir verbene \emph{beskrive} og \emph{gjøre rede 
for} brukt for relativt vanskelige begreper, som for eksempel celler og celledeling. I Blooms 
taksonomi\footnote[1]{Blooms taksonomi er et klassifiseringssystem for ulike læremål som lærere 
setter for sine elever.} utgjør derfor disse kompetansemålene det nederste trinn. Ved å koble til 
kompetansemålet fra forskerspiren kan det rettferdiggjøres at elevene skal kunne bruke begrepene 
i en videre forstand, danne sammenhenger og trekke egne slutninger. Det som gjenstår da er hvordan 
undervisningen kan legges opp slik at elevene kan danne gode forbindelser til begrepene og bruke 
de i undervisningen. Først vil jeg gjøre rede for undervisningssiuasjonen og undervisningsopplegget, 
og deretter kommer jeg til å analysere opplegget i lys av teori i pedagogikk og naturfagdidatikk.

\end{document}
