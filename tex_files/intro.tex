\documentclass[main.tex]{subfiles} 

\begin{document}

\section*{Problemstilling}
Ifølge \citeA{ungd15} har Norge på nasjonaltplan siden 1992 hatt et løft når det gjelder elev-lærer 
relasjon, og det psykososiale miljøet på skolen har markant forbedret seg. Det er færre elever 
som melder at de gruer seg til å gå på skolen og færre skulker. Generelt har trivsel 
blant elever økt, og det er etablert et godt læringsmiljø. Det er derimot forsatt rom for
forbedring når det gjelder hvor effektivt elever tar imot instruksjoner og hvorvidt de blir 
kognitiv utfordret. 
\newline

I de fleste studier som diskuterer hva som kjennetegner \emph{god undervisning}, knyttes 
dette ofte til tre dimensjoner, \citeA[s. 142]{klet13}, som den engelskspråkelige litteraturen 
kaller 
\begin{itemize}
\item emotional support (emosjonell støtte),
\item organisational support (organisatorisk støtte),
\item instructional support (undervisningsmessig støtte).
\end{itemize}
I korte trekk sammenfatter emosjonell støtte klasseromsarbeid som knytter
seg til de sosiale og emosjonelle rammene for læringsarbeid, %\citeA[s. ~143]{klet13},
organisatorisk støtte viser til fysisk organisering og klasseledelse %\citeA[s. ~189]{klet13}
og undervisningsmessig støtte retter fokus mot lærerens sentrale rolle i elevenes 
kunnskapstilegnelse, det \citeA{klet13} fremhever som "\emph{det læreren gjør av faktiske 
undervisningshandlinger i klasserommet som bidrar til læring}".
%\citeA[s. ~143 og s. ~146]{klet13}".
Internasjonal forksning peker på at det er spesielt lærerens kompetanse i undervisningsmessig 
støtte som trengs å systematiseres og videreutvikles. \citeA{klet13} referer til amerikanske 
studie, når hun skriver
\newline
\newline
"\emph{lærerne utviste et bredt handlingsrepertoar når det gjaldt emosjonell støtte og 
organisatorisk støtte, mens den undervisningsmessige støtten knyttet til å lære elever temaer som 
algebra, sannsynlighetsregning og andre utfordrende temaer i matematikk, var svakt utviklet hos 
de observerte lærerene (Schoenfeld 2011).}"
\newline
\newline
Ifølge \citeA{klet13}, kan undervisningsmessig støtte igjen deles i 4 dimensjoner,
\begin{enumerate}
\item klare læringsmål og intensjoner,
\item relevante kognitive utfordringer,
\item kvaliteten på klassesamtalene,
\item støttende klima.
\end{enumerate}
Her er det flere viktige faktorer som bidrar til læring. Introduksjon til nye fagtemaer faller 
under punkt 1 : \emph{klare læringsmål og intensjoner}. Effektive lærere bruker mer tid mot 
fagligrettet undervisning. De utøver også klar og tydelig klasseromsledelse, som igjen gir mer tid 
til undervisning rettet mot fag. Kvaliteten på oppgaver, variasjon i oppgavenes vanskelighetsgrad, 
og oppgaver som fordrer kognitivt, hvor akkurat nok tid\footnote{
Nærstudier av gruppeoppgaver i norsktimene i PISA og videostudien \citeA{klal13} viste at selv om 
oppgavene var engasjerende og relevante, elevene fikk for god tid til å løse oppgavene, og da ble 
den kognitive utfordringen intetsigende. Foreksempel, hvis elevene fikk 20 minutter til å løse en 
oppgave, klarte de å utføre det på 6 minutter og den resterende tiden ble da brukt til ikke-faglig 
diskusjoner.} 
er gitt til løsning av oppgaver, faller under punkt 2 : \emph{relevante kognitive utfordringer}. 
Kvaliteten på klassesamtaler skaper elevengasjement og deltagelse, og er med på å 
utvikle elevenes synspunkter og ideer. Forskning basert på mikroanalyser av språk og kommunikasjon 
har fått fram kunnskap om hvordan lærerens kommunikasjon med elevene følger det fagtradisjonen 
(jamfør \citeA{klet13}) kaller et IRE/F-mønster. Lærer starter dialog, m.a.o lærer tar 
initiativ(I), elev responderer(R) og responsen blir evaluert(E) og/eller kommentert(F) av læreren.
Klare rutiner og regler skaper et klassemiljø som er preget av respekt, toleranse og engasjement. 
Ungdata (jamfør \citeA{ungd15}) tyder på at norske klasserom viser god støttendeklima for læring, 
og generelt dominerer dialogisk samtaleform (IRE/F) norske klasserom.
\newline

Problemstillingen jeg har valgt å fokusere på i denne oppgaven er det første punktet i 
undervisningsmessig støtte : \emph{klare læringsmål og intensjoner}. Derfor spør jeg 
\newline
\newline
\textbf{Hvordan kan lærere demonstrere tydelige læringsmål og intensjoner, og dermed holde høy
læringstrykk ?}
\newline

Under observasjoner og utføring av undervisningssekvens har fokus vært på hvordan lærere kan bli 
flinkere til å delegere oppgaver og formidle informasjon. Hvis instrukser ikke er tydelige nok, vil 
elevene bruke unødvendig lenger tid på å komme i gang med undervisningsaktiviteten. Det er grunn 
til å tro at effektiv formidling av instrukser kan i helhet spare tid som igjen kan brukes i andre 
klasseaktiviteter. Den mest selvsagte måte å rette på dette er at lærer krever at ingen praktiske
spørsmål kan stilles etter at instrukser har blitt formidlet. Da gjenstår det kun rom for faglige 
spørsmål. Dette kan derimot kvele engasjement og er rett og slett ikke en god nok løsning. Det er 
derimot viktigere at læreren gir gode instrukser og forsatt tillater rom for spørsmål rundt 
instruksene. Dermed faller denne oppgaven til læreren som må tydeligere etablere lederrollen og 
foreta tiltak for å formidle instrukser effektivt. Elevene vil også forsatt ha muligheten til å 
kommunisere med sine medelever/læringspartnere.

\subsection*{Klassen}
Skolen befinner seg i et område hvor det er gode sosioøkonomiske forhold, deriblant har foreldrene 
til elevene høy utdanningsbakgrunn. Klassen som vi, praksisstudentene, observerte var en 8. klasse, 
som består av 13 gutter og 11 jenter. På skolen varer en skoletime i 50 minutter, etterfulgt av en 
10 minutter lang pause. Elevene ved skolen har i gjennomsnitt 27.6 timer i uka. I klassen sitter 
elevene to-og-to sammen ved sine pulter i et rutenett. Annenhver uke byttes plasseringen til 
elevene. Elevene blir fordelt sammen med det skolen kaller læringspartnere. Læreren printer 
et nytt klassekart som han/hun har tilgjengelig på sitt podium. Elever pleier å legge fra sin 
mobiler i sin hylleplass i klassen, eller bokskap utenfor klassen. Når en time starter, står 
elevene opp fra sine stoler og hilser på læreren før de får lov til sitte. Tavlen brukes sjelden 
siden lystavlen er plassert i alle klasserom rett foran tavlen. OneNote brukes flittig gjennom 
undervisning og til planleggingen av undervisningen. Elevene har også blitt vel kjent med OneNote 
ved å se lærere bruke den, og selv bruke den i sine delingstimer\footnote{En time der elever fra 
forskjellige klasser får felles undervisning, og deretter deles de i grupper og jobber sammen til å 
løse oppgaver på OneNote. Deres progressjon blir overvåket av en lærer som kan se all inntasting i 
sanntid.}. Lekser blir ført i It’s Learning plattformen. I klassen er det 3 elever fra 
velkomstklassen som deltar i faglig undervisning torsdag og fredag hver uke. 
Disse elevene har ofte problemmer med kommunikasjon, men ifølge kontaktlæreren er de flinkere til å 
lese og skrive. I blant bruker deres kontaktlærer engelsk for å formidle informasjon. 
Helklasseundervisningen blir alltid ført på norsk. Det er generelt ingen sosiale problemer eller 
konflikter i klassen, og elevene pleier å samarbeide med hverandre uten store problemer. Skolen har 
en del problemer med elever som trenger en eller annen form for tilrettelegging. Ifølge 
skoleadministrasjon får hver tredje elev en eller annen form for tilrettelegging. I trinnmøter til 
8. klasse blir det iblant tatt opp spørsmål om hvem som skal ha tilpasning og hvordan det skal 
utføres. Fokuset til skolen er å tilby sine elever et godt psykososial læringsmiljø.

\end{document}