\documentclass[main.tex]{subfiles} 

\begin{document}

\section*{Problemstilling}
En av de viktigste fokuseringsområder til Ludvigsen-utvalget er dybdelæring. For at elevene
skal nå kompetansemål som utfyller muligheten for dybdelæring vil det være sterk behov for å lage 
og bruke oppgaver som bidrar til relevante kognitive utfordringer og ikke minst fokusere på 
forståelse vs. fakta. I \citeA{ludv15} står det blant at

\begin{displayquote}
Utvalget mener at mer dybdelæring i skolen vil bidra til at elevene behersker sentrale elementer
i fagene bedre og lettere kan overføre læring fra ett fag til et annet. Forståelse av det eleven har
lært, er en forutsetning for og en konsekvens av dybdelæring. Skoler som legger bedre til rette for
læringsprosesser som fører til forståelse, kan bidra til å styrke elevenes motivasjon og opplevelse
av mestring og relevans i skolehverdagen.
\end{displayquote}
I naturfag er dette spesielt viktig siden det er veldig mange begreper elever skal mestre. For at de 
skal kunne danne et godt overordnet forståelse for faget, er det viktig at de kan gå fra 
enkeltstående begreper til koblinger mellom begreper og være klar over de logiske sammenhengene. 
Koblingen mellom dybdelæring og begrepsforståelse i naturfag er derfor tett tilknyttet.
\newline
\newline
Derfor er min problemstilling :
\newline
\newline
\textbf{Hvordan bør undervisningen legges opp for å skape god begrepsforståelse i en naturfagstime 
for 8. trinn og dermed styrke elevenes motivasjon og opplevelse av mestring og relevans i 
skolehverdagen ?}
\newline
\newline
Undervisningsopplegget jeg har forberedt har til hensikt å utfylle følgende kompetansemål  i 
læreplanen

\begin{displayquote}
\emph{Forskerspiren} :
\begin{itemize}
\vspace{-2mm}
\item formulere testbare hypoteser, planlegge og gjennomføre undersøkelser 
av dem og diskutere observasjoner og resultater i en rapport
\end{itemize}
\emph{Mangfold i naturen} :
\begin{itemize}
\vspace{-2mm}
\item beskrive oppbygningen av dyre- og planteceller og forklare hovedtrekkene i fotosyntese 
og celleånding
\vspace{-3mm}
\item gjøre rede for celledeling og for genetisk variasjon og arv
\end{itemize}
\end{displayquote} 
Fra kompetansemålene i \emph{Mangfold i naturen} blir verbene \emph{beskrive} og \emph{gjøre rede 
for} brukt 
for relativt vanskelige begreper. I Blooms taksonomi\footnote{Blooms taksonomi er et 
klassifiseringssystem for ulike læremål som lærere setter for sine elever.} utgjør disse 
kompetansemålene det nederste trinn. Celle og cellestruktur er relativt vanskelige begreper. Ved å 
koble til kompetansemålet fra forskerspiren kan det rettferdiggjøres at elevene skal kunne bruke 
begrepene i en videre forstand, danne sammenhenger og trekke egne slutninger.
Det som gjenstår da er hvordan undervisningen kan legges opp slik at 
elevene kan danne gode forbindelser til begrepene og bruke de i undervisningen og dagligtale.

\subsection*{Undervisningssituasjonen}
Skolen hvor undervisningsopplegget ble utført befinner seg i et område hvor det er gode 
sosioøkonomiske forhold. Klassen som vi, praksisstudentene, observerte var en 8. klasse, som består 
av 13 gutter og 11 jenter. I klassen sitter elevene to-og-to sammen ved sine pulter i et rutenett. 
Annenhver uke byttes plasseringen til elevene. Elevene blir fordelt sammen med det skolen kaller 
læringspartnere. Hensikten med læringspartnere er at de kan snakke sammen når de jobber med oppgaver 
eller når de blir bedt om å diskutere noe.  Det er generelt ingen sosiale problemer 
eller konflikter i klassen, og elevene pleier å samarbeide med hverandre uten store problemer. Tavlen 
brukes sjelden siden lystavlen er plassert i alle klasserom rett foran tavlen. OneNote brukes isteden 
for tavlen, og OneNote brukes også til planleggingen av undervisningen.
\newline
\newline
Jeg og en annen lærerstudent observerte elevene fra 8. klassen i både naturfagstimer og 
matematikktimer. Elevenes faglige forutsetninger er varierende, klassen har en jevn fordeling av 
fagelig sterke og faglig svake elever. I en naturfagstime observerte vi at elevene brukte mikroskop 
for å studere diverse celleprøver, blant annet fra deres egen munn og deres egne hårstrå. Timen 
startet med repetisjon av begreper om celler og mikroskop. Elevene ble fordelt i grupper på 3-4 
stykker, og læreren gikk rundt og veiledet alle gruppene. Noen av gruppene fikk 
hjelp fra læreren med å innstille mikroskopene slik at de endte opp med riktig fokus. Deretter brukte 
læreren et mikroskop som var koblet til en datamaskin. Bildet fra mikroskopet ble projisjert på 
lystavlen i laboratoriet. Hensikten med denne øvelsen var å gi elevene en pekepinne på 
størrelsesordner for celler og demonstrere bruk av mikroskop. Etter timen bemerket læreren at 
elevenehar forsatt ikke lært å skrive en rapport. Dette inspirerte meg til å bruke et tilsvarende 
opplegg til å strukturere mine egene undervisningstimer, og innføre en avsluttende rapport 
slik at elevene kan begynne å danne gode vaner for å skrive om sine obeservasjoner og resultater.

% og faller under det John Dewey  kaller utforskende arbeidsmåter, \citeA{rogs13}, \citeA[kap. 1]{knai11}.

\end{document}
