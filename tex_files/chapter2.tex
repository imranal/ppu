\documentclass[main.tex]{subfiles} 

\begin{document}
\section*{Analyse}
\label{sec:2}
Hvordan ble undervisningen lagt opp for å danne begrepsforståelse i naturfagstimene?

Lærer starter dialog, m.a.o lærer tar initiativ(I), elev responderer(R) og responsen blir
evaluert(E) og/eller kommentert(F) av læreren. I følge \citeA{klet13} dominerer IRE/F 
metoden klasseromsinteraksjonen. Til den første timen rekker elevene opp hånda for å 
respondere. Det viser seg at det er noen få elever, som viser trygghet og kontroll når 
de responderer til lærer initiert dialog. 

Ved å være bevisst på at alle elevene skal ha kjennskap til 
begrepene som blir tatt opp og repetert, er det da nødvendig å få bekreftet at elevene innehar en 
overordnet forståelse. Det kan derfor være nødvendig å utpeke noen elever som ikke viser aktiv 
deltagelse i timen og frembringe deres respons. Hvis elevene ikke klarer å respondere på lærer 
initiativ, kan utspørringen av elevene vise hull i deres kunnskap. I 2. timen ble denne formen for
utspørringen anvendt til å frembringe respons. 

% En viktig del av den sosiale utprøvingen av ideer og begreper innebærer å sammenlikne egne forestillinger
% med andres forestillinger i tillegg til naturvitenskapens forklaringer 
\citeA{odeg10}\citeA{dals94}

% Mortimer og Scott (2003) beskriver for eksempel læring som både individuell meningsskaping
% hvor man rekonstruerer gamle og nye ideer, og dialogisk meningsskaping hvor ideer gis et språk
% i en sosial sammenheng. Her skapes mening ved at man får forståelse av faglig kunnskap; i første
% rekke begrepsforståelse. 
\citeA{odeg10}

\citeA[s. ~136]{klet13} beskriver en god undervisningseksens hvor lærere klarer å balansere mellom tilegnelses-,
utprøvings-, og konsolideringssituasjoner. I følge Klette har norske klasserom ensidige tendenser i bruken av 
variert arbeidsmåter. Slik det kan ses fra figur \ref{fig:odeg10}, er det for eksempel lite konsolideringssituasjoner.
Lærernes metalæringsaktiviteter regnes som særlig avgjørende for å sikre elevenes læring, \citeA[s. ~186]{klet13}.
Gjennom alle timene har aktivering av forkunnskaper, gjennom repitisjon og gjenbruk av begreper og gjennomgang av 
lekser, bruk av appetittveker, som i dette tilfellet kan være bruken av en anatomisk modell og observasjon av
encellede organismer gjennom et mikroskop, og tilslutt oppsummering av timen med gjentagelse av prosessen
for flercellede organismer i motsatt rekkefølge, fra organismer med organsystemmer til encellede organismer, har 
alle timene bæret preg av bevisst fokus på bruk av konsolideringssituasjoner/metalæringsaktiviteter. Timene har 
dermed hatt en god fordeling av ny fagkunnskap.


\begin{figure}[h!]
\includegraphics[scale = 0.6]{../figures/undervisnings_aktivitet.png}
\caption{Oversikt over naturfaglærernes undervisningstilbud til elevene i prosent av kodet tid. Kilde: \protect\citeA{odeg10}.}
\label{fig:odeg10}
\end{figure}

% gode fagsentrerte samtaler mellom elever hvor elever brukte egne erfaringer
% og språket for å oppnå faglig forståelse, eller faglige samtaler med lærer som hjelper til å skape
% bro mellom praksis og teori 
\citeA{odeg10}


%  ”inquiry-based science teaching” 
\citeA{knai11}

%Siden resterende del av timen skal brukes til repetisjon, er det ikke nødvendig å 
%prøve å finne svakheter i elevenes respons gjennom helklassesamtalen. For å finne slike svakheter 
%ble gruppesamtalene en bedre plattform. I den forbindelse ble tokolonnenotatet tatt i bruk (se 
%vedlegg : \ref{sec:tokolonnenotat}).

% Timen 2. starter på tilsvarende vis som den første timen. Derimot i denne timen er oppsettet 
% forskjellig. Hensikten med timen er å repetere leksene elevene har fått til timen, om celletyper og
% utvikling av celler fra enkeltceller til flercelledeorganismer. Etter å konsultert med veilederen
% var jeg nå klar over at alle elevene hadde forutsetning til å kunne respondere til våre spørsmål, 
% så lenge de var relatert til leksene. Etter den første timen var jeg nå bevisst på at elevenes 
% respons var avhengig av deres trygghet med et gitt tema. 

% Til den siste timen hadde vi innsamlet prøver fra en utflukt og lagret de i laboratoriet. 
% Gjennom tilstrekkelige forhold hadde vi klart å vokse fram encellede organismer, deriblant tøffeldyr 
% (en organisme som er oppkalt etter sko fordi dens utseende ligner på tøfler).

Et premiss for dybdelæring er at elevene får anvendt kunnskapen, og dermed opplever de større grad av faglig utvikling, \citeA{beer14}.


\end{document}
