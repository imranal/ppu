% !TEX spellcheck = en
\documentclass[main.tex]{subfiles} 

\begin{document}
\section*{Analyse}
\label{sec:2}
Hvordan ble undervisningen lagt opp for å danne begrepsforståelse i naturfagstimene?

%IRE/F metoden <legg til forklaring> brukes, der elevene som rekker opp hånda blir spurt. Det viser seg at det 
%er noen få elever, som viser trygghet og kontroll når de responderer til lærer initiert dialog. 

% En av de viktigste egenskapene en lærer kan utvise er evnen til å tilpasse seg overfor 
% klassen, en gruppe eller på individnivå.<må begrunnes fra litteratur> 
% Ved å være bevisst på at alle elevene skal ha kjennskap til 
% begrepene som blir tatt opp og repetert, er det da nødvendig å få bekreftet at elevene innehar en 
% overordnet forståelse. Det kan derfor være nødvendig å utpeke noen elever som ikke viser aktiv 
% deltagelse i timen og frembringe deres respons. Dette er problematisk hvis det viser seg at 
% de ikke har forutsetninger for å kunne respondere. Da settes de i en vanskelig situasjon læreren 
% må lede de ut av, for eksempel ved hjelp av ledende spørsmål. Hvis det derimot 
% er forventet at det er en del av forutsetningene at elevene skal kunne respondere på lærer 
% initiativ, kan utspørringen av elevene vise hull i deres kunnskap. I 2. time blir en 
% annen form for lærerinitiativ brukt til å frembringe respons. 

%Siden resterende del av timen skal brukes til repetisjon, er det ikke nødvendig å 
%prøve å finne svakheter i elevenes respons gjennom helklassesamtalen. For å finne slike svakheter 
%ble gruppesamtalene en bedre plattform. I den forbindelse ble tokolonnenotatet tatt i bruk (se 
%vedlegg : \ref{sec:tokolonnenotat}).

% Timen 2. starter på tilsvarende vis som den første timen. Derimot i denne timen er oppsettet 
% forskjellig. Hensikten med timen er å repetere leksene elevene har fått til timen, om celletyper og
% utvikling av celler fra enkeltceller til flercelledeorganismer. Etter å konsultert med veilederen
% var jeg nå klar over at alle elevene hadde forutsetning til å kunne respondere til våre spørsmål, 
% så lenge de var relatert til leksene. Etter den første timen var jeg nå bevisst på at elevenes 
% respons var avhengig av deres trygghet med et gitt tema. 

% Til den siste timen hadde vi innsamlet prøver fra en utflukt og lagret de i laboratoriet. 
% Gjennom tilstrekkelige forhold hadde vi klart å vokse fram encellede organismer, deriblant tøffeldyr 
% (en organisme som er oppkalt etter sko fordi dens utseende ligner på tøfler).


\section*{Refleksjon}
\label{sec:3}
I følge \citeA{ludv15} vil læringsprosesser som fører til forståelse bidra til å styrke elevenes motivasjon og 
opplevelse av mestring og relevans i skolehverdagen. Men, var dette tilfellet for vår klasse og hvordan
kunne undervisningsopplegget forbedres?

% Opprinnelig hadde jeg planlagt å utføre innhenting av nødvendig materialer for å gjennomføre 
% mikroskop øvelsen ved å få elevene til å utføre innsamlingen gjennom en skoleutflukt. Utflukten var 
% en del av valgfaget friluftsliv. Fra vår klasse var det 12 elever som deltok i utflukten. Vår hensikt
% var opprinnelig å få disse elvene til å samle inn døde planter og vann. Derimot endte vi (dvs. 
% lærerstudentene) med å gjøre innsamlingen selv grunnet sen planlegging av timen. 

\end{document}
